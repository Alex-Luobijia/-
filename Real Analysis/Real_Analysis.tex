\documentclass[lang=en, 12pt]{elegantbook}


\title{{\Huge{\textbf{Real Analysis}}}}
\author{Alex Luo}
\date{\today}
\cover{cover}

\newcommand{\vt}[1]{\textit{\textbf{#1}}}
\newcommand{\RR}{\mathbb{R}}
\newcommand{\QQ}{\mathbb{Q}}
\newcommand{\ZZ}{\mathbb{Z}}
\newcommand{\NN}{\mathbb{N}}
\newcommand{\lmf}[1]{$\mathcal{M} \mathfrak{F} \{#1\}$}

\begin{document}
\frontmatter
\maketitle
\tableofcontents

\mainmatter
    \chapter{Why We Need Lebesgue Integral}
        In the previous sections, we have already developed the theory of Riemann integral. But Riemann integral dose not 
    close after taking limit. If ${f_n}$ is a sequence of Riemann integrable functions, and it's limit function may not be Riemann integrable.   
    \begin{example}
        Let ${f_n}$ be a sequence of integrable function over $[0,1]$ and $f$ be its limit function. 
    \begin{enumerate}
        \item We already know that if $f_n$ converge uniformly, then $f$ must be integrable and 
        \begin{equation}
        \lim_{n \to \infty} \int_a^b f_n = \int_a^b \lim_{n \to \infty}f_n = \int_a^b f \label{limint}
        \end{equation}
        by the theory of uniform Convergence.
        \item ${f_n}$ pointwise converge to $f$ but $f$ isn't integrable.
        Let ${r_k}$ be the sequence which consits of all the rationals( $\mathbb{Q}$ is countable, so this sequence exists).
        Let $f_n(x) = 1 \mbox{ if } x = r_k $ for some $k < n$, and for other cases, $f_n(x) = 0$. For any $n$, $f_n$ has only 
        finite discontinuity, thus it is integrable. ${f_n}$ converge to the dirichlet functions, which we know isn't integrable.
        \item ${f_n}$ pointwise converge to $f$ and $f$ is integrable. Let $f_n = x^n$, then ${f_n}$ converge to 
        $f: f(x)= 1 $ if $ x = 1 ; f(x) = 0 $ if $ x \in [0,1)$. The integral exist and satisfies (\ref*{limint}).
    \end{enumerate} 
    \end{example} \par
        Riemann integral functions is in the exactly situation of rational number, if you take limit, you probably got something
    else. An integral like real number, with the completeness,  is desired.
    \par
        One of the motivation of the developing the theory of integration is to calculate the area, and the method we use basicly is to cut
    a whole thing into pieces. For Riemann integral, to calculate the area under a function, one take a partition the domain, and build up 
    a "step function". If the step function converge to the original function as partition gets finer, we say the integral exist.\par
        For Lebesgue integral, we also take partition, but instead of on it's domain, we take it on the range. Let $f:E\to \mathbb{R}$ be a real value function
    $P_{M_n} = \{\frac{2iM}{2^n} -M : i \in [0,2^n] \cap \mathbb{Z} \}$ be a partition of $f(E) \in \mathbb{R}$. The integral under this partition will be 
    look like 
    \begin{equation}
    \int_E{f(x)} = \lim_{n \to \infty} \sum_{i=1}^{2^n} [(\frac{2iM}{2^n} - M )m(E_i)] = \lim_{n \to \infty} \sum_{i=1}^{2^n} [(\frac{2(i+1)M}{2^n} - M )m(E_i)] \label{li1} 
    \end{equation}
    where $E_i=\{x:f(x) \in [\frac{2iM}{2^n} -M,\frac{2(i+1)M}{2^n}-M]\}$, $m(E_i)$ is it's lenth.\par
        One may find as $n \to \infty$, the last equal sign in \eqref{li1} is true no matter what the intergrand is, but this doesn't means all the functions
    are Lebesgue integrable. Because without defining the lenth of a set(or volume, for higher dimension), the equation above is meaningless.
    So, in order to develop the theory of Lebesgue integral, we need to find a set function $m$, which is called Lebesgue measure, mapping 
    a general set into $[0,+\infty]$. We also hope $m$, has the following properties(in order to fit in our notion of lenth):
        \begin{enumerate}
            \item (Monotonicity) If $E \subset F$, $m(E) \leq m(F)$. This is inherited from the intuitive notion of volume.
            \item $m([a,b]) = b - a$. The lenth of integral is agree with its measure.
            \item (Countable additivity) Let ${E_i}$ be a disjoint collection of set, then $$m(\bigcap_{i=1}^{\infty} E_i) = \sum_{i=1}^{\infty}m(E_i)$$
            which enable us doing integral.
        \end{enumerate} \par
        We will first discuss Lebesgue's theory on $\RR$, and for most sets that occurs will be subsets of $\RR$. In next section, we will focus 
    measure theory, which solves the problems above.

    \chapter{Lebesgue Measure}
        To define a such measure function(the Lebesgue measure) we have discussed in the previous section, we need to first define a function
    which is called \emph{Lebesgue outer measure}. This is because countable additivity is really a rigorous property. Outer measure is defined
    for all set, but is doesn't have the countable additivity. The true Lebesgue measure only is defined on a certain collection of sets, which
    we called measurable set. Basicly, we ristrict the Lebesgue outer measure on the measurable sets to obtain the Lebesgue measure. In the 
    general measure theory, there are many measures, and Lebesgue measure is just one of them. Since we will only talk about Lebesgue measure,
    for convenience, we will just call the Lebesgue as measure(also the concepts measurable).
        \section{Lebesgue Outer Measure}
            \begin{definition}[Lebesgue outer measure]
                Given a set $A$, let $\{I_n\}$ be a countable open interval cover of $A$, namely $A \subset \bigcap_{n=1}^{\infty}I_n$ and
            ${I_n}$ is a countable collection of open intervals. Let $m^*(A) = \inf\{\sum_{n=1}^{\infty}I_n:\{I_n\} \mbox{is a open interval cover of }A\}$,  
            We call the function $m^*$ Lebesgue outer measure. 
            \end{definition} % is {{I_k}: union I_k contains A} a set ?
            Since for every set such collection of number(can be infinity) is nonempty, so the outer measure of every set on $\RR$ is well defined. 
            \begin{proposition}[Properties of Lebesgue outer measure]
                Let $A,B,\{A_n\}$ be any sets and countable collection of sets. The following properties is satisfied.
            \begin{enumerate}
                \item Positivity: $m^*(A) \geq 0$.
                \item Monotonicity: $B \subset A \Rightarrow m^{*}(A) \geq m^{*}(b) $
                \item Countable Subadditivity: $\sum_{n=1}^{\infty} m^*(A_n) \geq m^*(\bigcap_{n=1}^{\infty}A_n)$
                \item Translation Invariance: For any number $y$, $m^*(A+y) = m^*(A)$ where $A+y = {a+y:a\in A}$ 
            \end{enumerate} 

            \end{proposition}
            \begin{proof}
                Positivity: Since for every open interval cover $\{I_n\}$, $\sum_{n=1}^{\infty}|I_n|$ is a positive series, $m^*(A) \geq 0$. \par    
                Monotonicity: If $B \subset A$, every open interval cover of $A$ is also that of $B$. Thus by the property of $\inf$,
                $m^*(B) \leq m^*(A)$. \par
                Countable subadditivity: Suppose $\sum_{n=1}^{\infty}m^*(A_n) < +\infty$(the $+\infty$ case is trivial). For each $\{A_n\}$,
            let $\{I_{n_k}\}$ be its open interval cover, and we can choose a $\{I_{n_k}\}$ such that 
                $$\sum_{n=1}^{\infty}\{I_{n_k}\} - m^*(A_n) \leq \frac{\epsilon}{2^n}$$ \par
                Then we know 
                $$\bigcap_{n=1}^{\infty}A_n \subset \bigcap_{n=1}^{\infty}\{I_{n_k}\},\ \sum_{n,k=1}^{\infty}|I_{n_k}| \leq \sum_{n=1}^{\infty}m^*(A_n) + \epsilon$$
                Let $\epsilon \to 0$, the desired proposition is proved.\par 
                (This is something called coset. You may have run into this concept in algebra, but if you haven't, never mind it)\par
                
                Translation invariance: For every open interval cover $\{I_n\}$ of $A$, $\{I_n+y\}$ is a open interval cover of $A+y$, and at the same time, 
                for every open cover $\{I_n\}$ of $A+y$, $\{I_n+y\}$ is a open cover of $A$. Since invervals is translation invariant,
                $$\sum_{n=1}^{\infty}|I_n| =\sum_{n=1}^{\infty}I_n+y$$ 
                Given any countable open interval cover $\{I_n\}$ of $A$,
                \begin{equation}
                    m^*(A+y) \leq \sum_{n=1}^{\infty}I_n+y = \sum_{n=1}^{\infty}I_n
                \end{equation}
                Then we can see that $m^*(A+y) \leq  m^*(A+y)$, since the former one is a lower bound and the latter one is the greatest lower bound.
                The inequality in the other way can be proved in the came way, then we conclude that $m^*(A)=m^*(A+y)$.  
            \end{proof}
            \begin{example}[The outer measure of $\varnothing$]
                Since $\varnothing \subset E$ for any set $E$, let $I_{n_k} = (0,\frac{1}{n 2^k})$, then 
            the collection $\{I_{n_k}\}_{k=1}^{\infty}$ is certain a open interval cover of $\varnothing$ and 
            $\sum_{k=1}^{\infty}|I_{n_k}|= \frac{1}{n}$. Let $n\to \infty$, then $\sum_{k=1}^{\infty}|I_{n_k}| \to 0$. 
            Thus $m^*(\varnothing) \leq 0$(remember $m^*$ of a set is a lower bound). By the positivity of outer measure, $m^*(\varnothing) = 0$
            \end{example}
            \begin{example}[The outer measure of countable set is $0$]
                Give any countable set $ A = \{a_k\}$, let $I_{n_k} = (a_k-\frac{1}{n2^{k+1}},a_k+\frac{1}{n2^{k+1}})$. For each $n$, $\{I_{n_k}\}$
            is a open interval cover of $A$, and 
                $$\sum_{k=1}^{\infty}|I_{n_k}|=\sum_{k=1}^{\infty}\frac{1}{n2^k}=\frac{1}{n}$$ \par
                Let $n \to \infty$, we arrive at the conclusion: $m^*(A) = 0$.
            \end{example}
            \begin{proposition}[The outer measure of intervals]
                The outer measure of an interval is its lenth. Let $I = [a,b]$, then $$m^*(I) = b-a $$
                The similar statements for $[a,b), (a,b], (a,b)$ are also true.\par
                
            \end{proposition}
            \begin{proof}
                For every close interval $[a,b]$, one can find an open interval $(a-\epsilon,b+\epsilon)$ cover it, which has the lenth 
            $b-a+2\epsilon$. Since episilon can be choosen freely, $m^*([a,b]) \leq b-a$. \par
                It remains to show $m^*([a,b]) \geq b-a$, which is
            equivalence to show for all countable open interval covers $\{I_n\}$, 
            \begin{equation}
                \sum_{n=1}^{\infty}I_n \geq b-a \label{tomoi1}
            \end{equation}
                Because we are dealing a
            close interval on $\RR$, which is a compat set, Heine-Borel theorem tells us it has a finite subcover for each open cover. So 
            \begin{equation}
                \sum_{n=1}^{N}I_n \geq b-a \label{tomoi2}
            \end{equation}
            will indicate \eqref{tomoi1}. \par
                Since $a$ is in $\bigcap_{n=1}^{N} I_n$, there exist a $I_{n_1}$ contains $a$. If $I_{n_1}$ contains b, $|I_{n_1}|>b-a$;
            if not, $I_{n_1} = (c,d) \ , \ c<a<d<b$, then $d$ is in $[a,b]$, which means exist  $I_{n_2}$ such that $d \in I_{n_2}$; if
            $b \in I_{n_2}$, then $|I_{n_1}|+|I_{n_2}|>b-a$; if not, $I_{2} = (f,g) \ , \ f<d,g<b$, \dots There must be a $I_{n_i}$ such that 
            $b \in I_{n_i}$, and since we have only finite $I_n$, so this process will be eventually ended. After doing this for enough times,
            one will find out that 
            $$\sum_{i=1}{N}|I_{n_i}|>b-a$$ \par
                For other intervals, since they are subsets of closed interval, by monotonicity, we have $m^*(I)\leq b-a$. By the subadditivity
            and our previous conculsion on the outer measure of countable set, $$m^*([a,b])=m^*((a,b)\cap \{a,b\})\leq m^*((a,b))+m^*(\{a,b\})=b-a$$
            The proof for $[a,b)$ and $(a,b]$ is simialr.
            \end{proof}
        \section{Lebesgue Measure}
            \begin{definition}[Measurable sets and Lebesgue measure]
                Let $E$ be a set, if for any set $A$, 
                \begin{equation}
                    m^*(A) = m^*(A \cap E) + m^* (A \cap E^c) \label{ms}
                \end{equation}
                then we say $E$ is a measuralbe set. The collection of all the measurable sets is denoted $\mathcal{M}$.\par
                If $E$ is a measurable set, we define its Lebesgue measure $m(E) := m^*(E)$. 
            \end{definition}
            \begin{example}[The sets with outer measure zero are measurable]
                First, notice that, if one wants to check whether a set $E$ is measurable, the only thing is to show is 
            $$m^*(A) \geq m^*(A \cap E) + m^* (A \cap E^c)$$
            since the other inequality is guaranteened by the subadditivity.\par 
                For the set $E$ whose outer measure is zero, and any set $A$, by monotonicity of outer measure, we have 
                \begin{equation*}
                    \begin{aligned}
                        &m^*(A \cap E = 0) \ , \ \ m^*(A \cap E^c) \leq m^*(A) \\
                        \Rightarrow &m^*(A) \geq  m^*(A \cap E) + m^* (A \cap E^c)  
                    \end{aligned}
                \end{equation*} 
                And together the subadditivity of outer measure, we have 
                $$m^*(A) = m^*(A \cap E) + m^* (A \cap E^c)$$
                which shows that $E$ is measurable.
            \end{example}
            In our expectation of the measure function, it should have the countable additivity. So we know that the set must satisfied 
        \eqref{ms}, because $(A \cap E) \cap (A \cap E^c) = \varnothing$ and $(A \cap E) \cup (A \cap E^c) = A$, which is a special case 
        of finite additivity. The Lebesgue measure define above is just the ristriction of $m^*$ on the collection of measurable sets. 
        At last, we will show that the measure defined above have all the good properties we want. Before that, we first would like to 
        prove some properties of the measurable sets.
            \begin{proposition}[Properties of measurable sets] 
                Let $E_1, E_2 \in \mathcal{M}$,and $\{E_i\} \subset \mathcal{M}$\par
                \begin{enumerate}
                    \item $\varnothing \in \mathcal{M}$
                    \item $E_1^c \in \mathcal{M}$.
                    \item $E_1 \cap E_2, \ E_1 \cup E_2, \ E_1/E_2 \in \mathcal{M}$. From this one can easily derive the closedness of finite 
                intersection and union.
                    \item If for any $i \neq j$, $E_i \cap E_j = \varnothing$, then 
                    $$m(A \cap \bigcup_{i=1}^{N}E_i) = \sum_{i=1}^{N}m(A \cap E_i)$$
                    The finite additivity among measurable sets is a special case of this proposition, taking $A=\mathbb{R}$. 
                    \item $\bigcap_{i=1}^{\infty} E_i, \ \bigcup_{i=1}^{\infty} E_i \in \mathcal{M}$
                \end{enumerate}\par
                
            \end{proposition}
            \begin{proof}
                1. This is true because we have show that measure zero sets are measurable.\par
                2. Since $(E^c)^c = E$, so 
                \begin{equation*}
                    \begin{aligned}
                        m^*(A) &= m^*(A\cap E) + m^*(A \cap E^c) \\
                               &= m^*(A\cap(E^c)^c) + m^*(A \cap E^c)
                    \end{aligned}
                \end{equation*}
                which shows that $E^c$ is measurable.\par
                3. Here we first prove the union of measurable sets is measurable. Since $E_1, \ E_2 \in \mathcal{M}$, for any set $A$
                \begin{equation*}
                    \begin{aligned}
                        m^*(A) &= m^*(A \cap E_1) + m^*(A \cap E_1^c)\\
                               &= m^*(A \cap E_1) + m^*((A \cap E_1^c) \cap E_2) + m^*((A \cap E_1^c) \cap E_2^c)  
                    \end{aligned}
                \end{equation*}
                By basic set identities,
                \begin{equation*}
                    \begin{aligned}
                        (A \cap E_1^c) \cap E_2^c = A \cap (E_1 \cup E_2) ^c\\
                        [A \cap E_1] \cup [(A \cap E_1^c) \cap E_2] = A \cap (E_1 \cup E_2)
                    \end{aligned}
                \end{equation*}
                and by the subadditivity of outer measure,
                \begin{equation*}
                    \begin{aligned}
                        m^*(A) &\geq m^*([A \cap E_1] \cup [(A \cap E_1^c) \cap E_2]) + m^*((A \cap E_1^c) \cap E_2^c)\\
                               &= m^*(A \cap (E_1 \cup E_2)) + m^*(A \cap (E_1 \cup E_2)^c)
                    \end{aligned}
                \end{equation*}
                is true. Thus the union of measurable set is measurable.\par
                4. The proof proceeds by induction on $N$. It is clear true for $N = 1$. Suppose it is true for $N-1$. By the measurability
            of $E_N$, and the fact that $\{E_n\}$ is disjoint sequence of sets, 
            \begin{equation*}
                \begin{aligned}
                m^*(A \cap (\bigcup_{n=1}^{N}E_n)) &= m^*( [A \cap (\bigcup_{n=1}^{N}E_n)] \cap E_N) + m^*([A \cap (\bigcup_{n=1}^{N}E_n)] \cap E_N^c)\\
                &= m^*(A \cap E_N) + m^*(A \cap \bigcup_{n=1}^{N-1}E_n) \\
                &= \sum_{n=1}^{N}E_n
                \end{aligned}
            \end{equation*}\par
                5. Let $E$ be the union of countable collection of measurable set, say $\{A_k\}$. By defining a new sequence of sets,
            $$ E_1 = A_1, \ \ \ E_n = A_n/E_{n-1} $$ 
            one can express $E$ as a union of disjoint sequence of measurable sets. Let $A$ be any set. Observing that $E^c \subset (\bigcup_{k=1}^N E_k)^c$
                \begin{equation*}
                    \begin{aligned}
                        m^*(A) &\geq m^*(A \cap \bigcup_{k=1}^N E_k) + m^*(A\cap (\bigcup_{k=1}^N E_k)^c\\
                               &\geq m^*(A\cap \bigcup_{k=1}^N E_k) + m^*(A\cap E^c)\\
                               &= \sum_{k=1}^N m^*(A\cap E_k) +m^*(A\cap E^c)
                    \end{aligned}
                \end{equation*}
                Since the relation is regardless of $N$
                \begin{equation*}
                    \begin{aligned}
                        m^*(A) &\geq \sum_{k=1}^{\infty} m^*(A\cap E_k) +m^*(A\cap E^c)\\
                            &\geq m^*(A\cap E) + m^*(A\cap E^c) 
                    \end{aligned}
                \end{equation*}
                Then the desired conclusion is proved.
            \end{proof}
            \begin{proposition}[Intervals are measurable]
                For every $a, \ b$ in $\mathbb{R}^*$, $[a,b], \ (a,b), \ [a,b), \ (a,b]$ are measurable.\par
                
            \end{proposition}
            \begin{proof}
                First, we show that $(a,\infty)$ is measurable. Give any set $A$, we need to show that
                $$m^*(A) \geq m^*(A\cap (a,+\infty)) + m^*(A \cap (-\infty,a))$$
            since $m^*(A)$ is a infimum, it is surffice to show
                \begin{equation}
                    m^*(A\cap (a,+\infty)) + m^*(A \cap (-\infty,a)) \leq \sum_{n=1}^{\infty} |I_{n}| \label{iim}
                \end{equation}
            where $I_n$ is a open interval cover of $A$.
                For each $n$, we define $J_n = I_n \cap (a, +\infty)$, $k_n = I_n \cap (-\infty, a)$. Then 
                \begin{equation*}
                    \begin{aligned}
                        |I_n| &= |J_n| + |K_n|\\
                        \sum_{n=1}^{\infty}|I_n| &= \sum_{n=1}^{\infty} |J_n| + |K_n| = \sum_{n=1}^{\infty} |J_n| + \sum_{n=1}^{\infty}|K_n|
                    \end{aligned}
                \end{equation*}
            Observing that $\sum_{n=1}^{\infty} |J_n|$ and $\sum_{n=1}^{\infty}|K_n|$ are open interval covers of $A \cap (a, +\infty)$ and $A \cap (-\infty, a)$,
        So \eqref{iim} is varified, then we proved that $(a,+\infty)$ is measurable. \par
            Since $\mathcal{M}$ is closed under complement, $(-\infty, a]$ is measurable; Since $\mathcal{M}$ is closed under countable,
        union, 
        $$(-\infty, a) = \bigcup_{n=1}^{\infty} (-\infty, a-\frac{1}{n}]$$ 
        is measurable, and also its complement $[a, \infty)$. Let $a \leq b$. 
        $$[a, b] = [a, \infty) \cap (-\infty, b)$$
        is measurable. Other intervals are measurable can be shown in the same manner.
        \end{proof}
            \begin{definition}[Algebra and $\sigma$ algebra]
                Let $A$ be a collection of sets, if $\forall F, \ E \in A$ 
                \begin{enumerate}
                    \item $\varnothing \in \mathcal{M}$
                    \item $F^c \in \mathcal{M}$.
                    \item $F \cap E \in \mathcal{M}$
                \end{enumerate}
                then, we say $A$ is an algebra.\par
                If a algebra $A$ is closed under countable intersection, namely if a countable collection of sets ${E_n} \in A$, then $\bigcap_{n=1}^{\infty}E_n \in A$,
            we call $A$ a $\sigma$ algebra.\par
                Given a collection of set $A$, we call the smallest algebra contains $A$(if you remove any part of it, it won't be an algebra or it won't 
            contains $A$ anymore) the algebra generate by $A$.                  
            \end{definition}
            By Demorgan identity, we can see that a algebra is also close under intersection and difference, and a $\sigma$ algebra is close under 
        countable intersection. The collection of all measurable set is a algebra. The collection of set with this particularstructure is 
        quite useful in analysis. One need to be careful not to mix up this concept with algebra in Algebra. 
            \begin{example}[Borel sets]
                We call the $\sigma$ algebra generated by the collection of open sets in $\RR$ Borel sets. Before Lebesgue, Borel had already built
            up a measure theory, but not in the purpose of defining an integral. We use $\mathcal{B}$ to denote the collection of Borel sets. 
            \end{example}
            \begin{definition}[$F_{\sigma}$ set and $G_{\delta}$ set]
                For a countable collection of close sets $\{F_n\}$, we say $F = \bigcup_{n=1}^{\infty}F_n$ is a $F_{\sigma}$ set; for any countable collection of 
            open sets $\{G_n\}$, we call the set $G = \bigcap_{n=1}^{\infty}$ a $G_{\delta}$ set. Also you can have an $F_{\sigma \delta}$ set
            and $G_{\delta \sigma}$ set, which are intersection of countable $F_{\sigma}$ sets and union of countable $G_{\delta}$ sets.   
            \end{definition}
            \begin{proposition}[The properties of Borel sets]
                \begin{enumerate}
                    \item Open sets, close sets, $G_{\delta}$ sets, $F_{\sigma}$ sets \dots, in $\mathbb{R}$ are all Borel sets.
                    \item $\mathcal{B} \subset \mathcal{M}$, the collection of Borel sets is contained in the collection of measuralbe set,
                    namely, every Borel sets are measuralbe.
                    %\item Any strickly monotonic increasing function on $\mathbb{R}$ maps Borel sets to Borel sets.
                \end{enumerate}
                
            \end{proposition}
            \begin{proof}\par
                1. By definition, all the open sets should be Borel set, and so are close sets, which are complement of open sets.
            $G_{\delta}, \ F_{\sigma}$ are countable intersection, countable union of Borel sets, so they are also Borel sets. \par
                2. By open set construction theorem, every open set $O = \bigcup_{n=1}^{\infty} I_n$, where $\{I_n\}$ is a countable collection
            of open intervals. Since $I_n$ is measurable, so is open sets. Because Borel set is the smallest $\sigma$ algebra contains
            all the open sets, thus $\mathcal{B} \subset \mathcal{M}$.  
            \end{proof}
            We end this section with the most important property of Lebesgue measure: countable additivity.
            \begin{proposition}
                Given a countable collection of disjoint measurable sets $\{E_n\}$, 
                \begin{equation}
                    m^*(\bigcup_{n=1}^{\infty} E_n) = \sum_{n=1}^{\infty} E_n \label{ca}
                \end{equation}
                
            \end{proposition}
            \begin{proof}
                We know that $\bigcup_{n=1}^{\infty} E_n$ is measurable. Since outer measure has subadditivity, it remains to show 
                \begin{equation}
                    m^*(\bigcup_{n=1}^{\infty} E_n) \geq \sum_{n=1}^{\infty} E_n \label{caa}
                \end{equation}
                For each number $N$, by finite additivity of measurable sets, 
                $$ m^*(\bigcup_{n=1}^{N} E_n) = \sum_{n=1}{N} E_n$$
                And we have $\bigcup_{n=1}^{N} E_n \subset \bigcup_{n=1}^{\infty} E_n$. By monotonicity of outer measure,
                \begin{equation*}
                    \begin{aligned}
                        m^*(\bigcup_{n=1}^{\infty} E_n) &\geq m^*(\bigcup_{n=1}^{N} E_n)\\
                        &= \sum_{n=1}{N} E_n
                    \end{aligned}
                \end{equation*}
                Let $N \to \infty$, then the proposition is proved.
            \end{proof}
        \section{Outer and Inner Approximation of Measurable Sets}
            One can find out a easy fact from the definition of measurable set, that is for a finite measuralbe set $E \subset B$ 
        we have the excision property
            \begin{equation*}
                m^*(B/E) = m^*(B) - m^*(E)
            \end{equation*}
            which is a quite useful obsevation. 

            \begin{theorem}[Use open sets, close sets, $G_{\delta}$, $F_{\sigma}$ to approximate measurable set]
                Let $E$ be a subset of $\RR$. The following assertions are equivalent to measurability of $E$.
                \begin{enumerate}
                    \item $\forall \epsilon > 0$, $\exists O \supset E$ is a open set, such that $m^*(O/E) \leq \epsilon$.
                    \item $\exists G \supset E$ is a $G_{\delta}$ set, such that $m^*(G/E) = 0$
                    \item $\forall \epsilon > 0$, $\exists C \subset E$ is a close set, such that $m^*(E/C) \leq \epsilon$.
                    \item $\exists F \subset E$ is a $F_{\sigma}$ set, such that $m^*(E/F) = 0$
                \end{enumerate} 
            \end{theorem}
            \begin{proof}\par
                1. Suppose $E$ is measurable. Since the measure of $E$ is a supremum, so given $\epsilon > 0$, there exists a open interval 
            cover, such that 
                \begin{equation}
                    m^*(E) + \epsilon \geq \sum_{n=1}^{\infty} |I_n| \geq m^*(E) 
                \end{equation}
                Let $O = \bigcup_{n=1}^{\infty} I_n$, then $O$ is a open set and $O \supset E$, it remains to show that 
                $$m^*(O/E) \leq \epsilon$$
                By definition of outer measure 
                $$m^*(O) \leq \sum_{n=1}^{\infty} |I_n| \leq m^*(E) + \epsilon$$
                which suggest that 
                $$m^*(O)-m^*(E) \leq \epsilon$$\par
                From the observation we made in the beginning of this section, we know the proposition will be true if $m^*(E) < +\infty$.
            For the case that $m^*(E) = \infty$, oen can express $E$ as countable disjoint collection of measurable sets of finite measure, say $\{E_K\}$.
            For instance, one construction of $\{E_k\}$ could be 
            $$E_{k} = [\frac{k}{2}-1, \frac{k}{2}) \cap E \mbox{ while } k=2z $$
            $$E_{k} = [-\frac{k+1}{2}, -\frac{k+1}{2}+1) \cap E \mbox{ while } k=2z+1$$
            Then we can find a sequence of open sets $\{O_k\}$ such that $O_k \supset E_k$ and $m^*(O_k/E_k) \leq \frac{\epsilon}{2^k}$.
            Since $O/E$ is a subset of $\bigcup_{k=1}^{\infty} O_k/E_k$, by monotonicity and subadditivity,
            $$m^*(O/E) \leq m^*(\bigcup_{k=1}^{\infty} O_k/E_k) \leq \sum_{k=1}^{\infty} m^*(O_k/E_k) \leq \epsilon$$
            Thus part 1 is proved.\par
                2. By 1., for every $n$, we can find a open set $O_n \supset E$ such that $m^*(O/E) \leq \frac{1}{n}$. Let 
            $G = \bigcap_{n=1}^{\infty} O_n$. It is clear that $G \supset E$, and $G$ is a $G_{\delta}$ set. Since $G/E \subset O_n/E$
            for every $n$,
            \begin{equation*}
                \begin{aligned}
                    m*(G/E) &\leq m^*(O_n /E)\\
                            &\leq \frac{1}{n}
                \end{aligned}
            \end{equation*}
            Let $n \to \infty$, we have $m^*(G/E) = 0$.\par
                Suppose 2. holds. We know $G$ and $G/E$ are measurable (one is $G_{\delta}$, one has out measure zero), so $E$ must be 
            measurable.\par
            3. $E^c$ is clearly measurable, so for every $\epsilon > 0$, we can find a $O \supset E^c$ which is a open set such that 
        $m^*(O/E^c) \leq \epsilon$. It is not hard to see that $O^c \subset E$ is a close set, and it remains to show that $m^*(E/O^c) \leq \epsilon$.
        $$E/O^c = E \cap O = O/E^c$$ Thus part 3 is proved.\par
            4. For every $n$, we can find a $C_n \subset E$ which is a close set, such that $m^*(E/C_n) \leq \frac{1}{n}$. Let 
        $F=\bigcup_{n=1}^{\infty} C_n$, then F is a $F_{\sigma}$ set in $E$. Since $E/\bigcup_{n=1}^{\infty}C_n$ is a subset of $E/C_n$,
        $m^*(E/F)=0$.\par
            Suppose 4. holds. Then $F$ and $E/F$ are measurable, so is $E$.
            \end{proof}
            This theorem indicated that every measurable set is nearly a simple set. Proving the next theorem, we push this idea futher.
            \begin{theorem}[Measurable sets are nearly finite disjoint union of open intervals]\label{LittlewoodFirstPrinciple}
                Given a measuralbe set $E$ of finite measure, for each $\epsilon > 0$, there is a finite collection of disjoint open intervals
            $\{I_n\}_{n=1}^N$, for which if $O = \bigcup_{n=1}^{N} I_n$,
                \begin{equation}
                    m^*(E/O)+m^*(O/E) < \epsilon
                \end{equation} \par
            \end{theorem}
            \begin{proof}
                $\forall \epsilon > 0$, $\exists O'$ is a open set such that $O' \supset E$ and 
                \begin{equation*}
                    m^*(O'/E) < \frac{\epsilon}{2} 
                \end{equation*}
                Every open set can be expressed as a countable collection of disjoint open intervals, so is $O'$. Let $O' = \bigcup_{n=1}^{\infty} I_n$.
            From $m^*(E)$ is finite, we can know that $m^*(O')$ is finite. So
            $$\sum_{n=1}^{\infty}|I_n|=m^*(O')$$
            is a converge positive series. Thus there exists a $N$ such that 
            $$\sum_{n=N}^{\infty}|I_n| < \frac{\epsilon}{2}$$
                We define $O =\bigcup_{n=1}^{N}I_n$. Since $O/E \subset O'/E$, and $E/O \subset O'/O$
                \begin{equation*}
                    \begin{aligned}
                        m^*(E/O)+m^*(O/E) &\leq  m^*(O'/O)+m^*(O'/E)\\
                        &< \sum_{n=N}^{\infty}|I_n| + \frac{\epsilon}{2}\\
                        &< \epsilon
                    \end{aligned}
                \end{equation*}
            \end{proof}
            Notice that we used the excision property of measurable set($m^*(B/E) = m^*(B) - m^*(E)$) in the previous theorems, thus they only
        apply to finite measure sets. One should pay attention to the difference between $m^*(O/E) < \epsilon$ and $m^*(O)-m^*(E)<\epsilon$, 
        as the later one is true for any set $E$. In the next proposition, we show that none measurable sets doesn't possess the excersion 
        property.
            \begin{proposition}
                Suppose $E$ is not measurable, and has finite out measure. Then there exist a open set $O$, such that 
                $$m^*(O/E) > m^*(O) - m^*(E)$$
            \end{proposition}
            \begin{proof}
                Since $E$ is not measurable, there exists a set $A$, such that 
                \begin{equation}\label{excision}
                    m^*(A) < m^*(E \cap A) + m^*(A / E)
                \end{equation}
                Given $\epsilon > 0$, let $I_n$ be an open interval cover of $A \cup E$, and $O = \bigcup_{n=1}^{\infty} I_n$, where
            $$m^*(A \cup E) \leq m^*(O) \leq m^*(A \cup E) + \epsilon$$ 
                We only need to show $m^*(O)<m^*(O/E) + m^*(E)$. Since \eqref{excision}, $\exists \epsilon$ such that  
                \begin{equation*}
                    \begin{aligned}
                        m^*(O) &< m^*(A \cup E) + \epsilon \\
                        &< m^*(A) + m^*(E)
                        &< m^*(O/E) + m^*(E)
                    \end{aligned}
                \end{equation*}
                Thus the proposition is proved.
            \end{proof}

    \chapter{Lebesgue Measuralbe Functions}
            Remembering that our purpose is to define Lebesgue integral, integral is an operator on functions, so the concept of measurable function
        is needed. In the first chapter, the range of the function $f$ which we want to integrate is cut into small intervals, and we want to 
        measure the size of their inverse image(of course using the Lebesgue measure). Generaly, the inverse image of these small intervals are not
        measurable, so we make compromise, by working with a specific type of function. 
            In this chapter, we would see three interesting and important properties of measurable function, called Littlewood's three principle. It
        says \textbf{measurable sets are nearly finite union of open intervals}; \textbf{measurable functions are nearly continuous functions};
        \textbf{a sequence of measurable functions which converge pointwisely is almost converge uniformly}. The first Priciple has already 
        shown by \eqref{LittlewoodFirstPrinciple}. We will show the latter two by Egoroff's theorem and Lusin's theorem. Although the 
        utterance of the principles are not in the rigorous mathematic laguagues, it is still important since it provide a intuitive 
        understanding of those theorems. In my opinion, this kind of understanding is more in need in the mathematic learning, rather than
        the rigorous proof.

        \section{Definition and basic properties}
            Although we want to have $f^{-1}[a,b]$ to be measurable, it turned out that the following condition is equivalent to that.
            \begin{definition}[Lebesgue measuralbe functions]
                Given an extended real value function $f: E \to \mathbb{R}^*$ while $E\in \mathcal{M}$, We define $$E\{f>\alpha\}:=\{x\in E:f(x) > \alpha\}$$. \par
                If $\forall \alpha, \ E\{f>\alpha\} \in \mathcal{M}$ 
                then we say $f$ is measuralbe. We denote the set of all Lebesgue measuralbe functionson a measurable set E by \lmf{E}
            \end{definition}
            \begin{proposition}
                Let the function $f$ have measurable domain. Then the following condition are equivalent to the measurability of $f$.
                \begin{enumerate}
                    \item For each $\alpha$, $E\{f \geq \alpha\}$ is measurable.
                    \item For each $\alpha$, $E\{f < \alpha\}$ is measurable.
                    \item For each $\alpha$, $E\{f \leq \alpha\}$ is measurable.
                \end{enumerate}
            \end{proposition}
            \begin{proof}
                Suppose $f:E \to \RR$ is measurable, which means $\forall \alpha \in \RR, \ E\{f>\alpha\}$ is measurable. Observing that
            $$E\{f \geq \alpha\} = \bigcap_{n=1}^{\infty} E\{f > \alpha + \frac{1}{n}\}$$
            which indicate $\forall \alpha \in \RR, \ E\{f \geq \alpha\}$ is measuralbe.\par
                Suppose $\forall \alpha \in \RR, \ E\{f \geq \alpha\}$ are measuralbe. 
                $$E\{f < \alpha\} = E\{f \geq \alpha\}^c \cap E$$
            Thus $\forall \alpha \in \RR, \ E\{f < \alpha\}$ is measurable.\par
                Suppose $\forall \alpha \in \RR, \ E\{f < \alpha\}$ are measuralbe.
                $$E\{f \leq \alpha\} = \bigcap_{n=1}^{\infty} E\{f < \alpha - \frac{1}{n}\}$$
            Thus $\forall \alpha \in \RR, \ E\{f \leq \alpha\}$ is measurable.\par 
                And by the similar trick, one can show if $\forall \alpha \in \RR, \ E\{f\leq \alpha\}$ is measurable, then 
                $$E\{f > \alpha\} = E\{f \leq \alpha\}^c \cap E$$
            is measurable. 
            \end{proof}\par
            Let's see some examples of measurable functions.
            \begin{proposition}
                $f$ is define on $E \in \mathcal{M}$. 
                \begin{enumerate}
                    \item If $f$ is continued on $E$, $f \in$ \lmf{E}.
                    \item If f is monotonic on $E$, $f \in$ \lmf{E}.
                    \item Define $\chi_E:\mathbb{R} \to \mathbb{R}$ by following ristriction: if $x \in E$, $\chi_E(x)=1$; 
                otherwise, $\chi_E(x)=0$. Then $E \in \mathcal{M} \Longleftrightarrow\ \chi_E \in$ \lmf{E}. 
                \end{enumerate}
            \end{proposition}
            \begin{proof}
                1. Since the inverse image of open set is open(relative open), given $O \in \RR$, we have $f^{-1}(O) = E \cap U$, where $U$ is open.
            Being the intersection of measurable sets, $f^{-1}(O)$ is measurable. Since each $\{x \in E:x > \alpha\}$ is open, thus $E\{f > \alpha\}$
            is measurable.\par
                2. Let $x_0 = \inf( E\{f > \alpha\})$, then for any $x>x_0, \ x \in E$, 
                $$f(x) \geq f(x_0) > \alpha$$
                Thus $E\{f> \alpha \}$ can only be $[\alpha,\infty) \cap E$ or $(\alpha,\infty) \cap E$. Both of them are measurable.\par
                3. Notice that the range of $f$ only contains two points, $1$ and $0$. If $\alpha < 0$, $E\{ f > \alpha \} = \RR$;
                if $\alpha \geq 1$, $E\{ f > \alpha \} = \varnothing$; if $ 0 \leq \alpha < 1$,$E\{ f > \alpha \} = E$, thus 
                $\chi_E$ is measurable if and only if $E$ is measurable.
            \end{proof}
            In some sense, a set with measure zero is very small. Suppose $P$ is a property, $E$ is a set and $E_0$ is a measure zero subset. 
        It turned out that the condition \textsl{for all $x \in E-E_0$, $P(x)$ is true} is strong enough to derive many desirable conclusion.  
        For conveniency, we say \textsl{$P$ is true almost everywhere on $E$} to mean that for all $x \in E$ but a set of measure zero, $P(x)$ 
        is true. And it's abbriviation is \textsl{$P$ is true on E a.e.} 
            \begin{proposition}\label{MZS}
                Let $f:E \to \RR^*$ be a extended real value function, while $E$ is a measurable set. 
                \begin{enumerate}
                    \item If $f$ is measurable and $f = g$ a.e. on $E$, then $g$ is measurable.
                    \item For a measurable subset $D \subset E$, $f$ is measurable if and only if it's restriction on $D$ and $E-D$ are 
                    measurable.
                \end{enumerate}
            \end{proposition}
            \begin{proof}
                1. Suppose $f \neq g$ on $E_0$, which is measure zero. 
                \begin{equation*}
                    \begin{aligned}
                        E\{g < \alpha\} &= (E-E_0)\{g < \alpha\} \cup E_0\{g < \alpha\}\\
                        &= (E-E_0)\{f < \alpha\} \cup (E\{g < \alpha\} \cap E_0)\\
                        &= (E\{f < \alpha\} - E_0) \cup (E\{g < \alpha\} \cap E_0)\\
                    \end{aligned}
                \end{equation*}
                $(E\{g < \alpha\} \cap E_0)$ is a subset of $E_0$, thus it is measure zero, thus measurable; $E\{f < \alpha\} - E_0$
            is the difference of two measurable set. Since the collection of all measurable sets is a $\sigma$ algebra, so it is measurable.
            The union of two measurable set are still measurable, thus $E\{g < \alpha\}$ is measurable. \par
                2. Suppose $f$ is measurable. Then 
                \begin{equation*}
                    \begin{aligned}
                        D\{f|_D < \alpha\} &= D \cap E\{f < \alpha\}\\
                        (E-D)\{f|_{D-E} < \alpha\} &= (E-D) \cap E\{f < \alpha\}
                    \end{aligned}
                \end{equation*}
                So $f|_D$ and $f|_{E-D}$ are measurable.\par
                On the other hand, if $f|_D$ and $f|_{E-D}$ are measurable, 
                \begin{equation*}
                    \begin{aligned}
                        E\{f < \alpha \} &= (E-D)\{f < \alpha\} \cup D\{f < \alpha\}\\
                        &= (E-D)\{f|_{E-D} < \alpha\} \cup D\{f|_D < \alpha\}
                    \end{aligned}
                \end{equation*}
                Thus the proposition is proved.
            \end{proof}
            Just as what is done when studing continuous function, we want to know that whether the sum, product and composition of 
        measurable functions are still measurable. But in general, the sum of two extended real valued is not even well defined. For example,
        if $f(x) = +\infty$ while $g(x) = -\infty$, then $f(x) + g(x)$ is not well defined. If we ask $f$ and $g$ to be finite a.e. in there 
        domain, then we can know the set of points which are not well defined is measure zero. By the former proposition, the value of a 
        function in a measure zero set don't influence the fact whether it is measurable. So we are allowd to talk about whether $f + g$ is 
        measurable in a slightly different condition.
            \begin{theorem}[Linear combination and mutlpilication]
                Let $f:E \to \RR^*, \ g: E \to \RR^*$ be measurable and finite a.e. on $E$, $r \in \RR$, then 
                \begin{enumerate}
                    \item $r \cdot f$ is measurable.
                    \item $f+g$ is measurable.
                    \item $f \cdot g$ is measurable.
                \end{enumerate}  
            \end{theorem}
            \begin{proof}
                One can easily find out that $f + g$ is well defined and finite a.e. on $E$. Let $E_0$ be the set that $f + g$ is well defined and 
            finite a.e.. Then by the prec
                1. Since $f$ is measurable, 
            $$E{f>\frac{\alpha}{r}} = E{f>\alpha}$$ is measurable.
                2. The element in $E\{f(x)+g(x) < c\}$ has the following property:
                \begin{equation*}
                    \begin{aligned}
                        f(x) + g(x) &< c\\
                        f(x) &< c - g(x)\\
                    \end{aligned}
                \end{equation*}
                Since the rational number are dense in $\RR$, so for each $x \in E$, we can find a $q \in \QQ$ such that
                $$f(x) < q < c-g(x)$$
                which is equivalence to 
                \begin{equation*}
                    \begin{aligned}
                        f(x) &< q \\
                        g(x) &< c-q
                    \end{aligned}
                \end{equation*}
                So 
                $$ E\{f(x) + g(x) < c \} = \bigcup_{q \in Q} E\{f(x) < c\} \cap E\{g(x) < c-q\}$$
                The intersection and countable union of measurable sets are still measurable, thus $E\{f(x) + g(x) < c \}$ is measurable.
                3. Observing that $fg = \frac{1}{2} (f + g)^2$, by 1. and 2., we only need to show that if f is measurable, then $f^2$ is 
            measurable. $\forall x \in E\{f^2(x)< c\}$, by elementary algebra, 
            $$-c^{\frac{1}{2}}< f(x)<c^{\frac{1}{2}}$$
            Thus $E\{f^2(x)<c\} = E\{f(x) > -c^{\frac{1}{2}}\} \cap E\{f(x)<c^{\frac{1}{2}}\}$, which is measurable. 
            \end{proof}
            Many properties such as continuity or differentiability, are preserved under the composition of functions. Howerver, in general,the 
        composition of measurable functions are not measurable. The counter-example is provided in the appendix. The basic idea of the counter 
        example is to prove that there exsit a measurable function which maps a measurable set to a non-measurable set, then composite it with
        a characteristic functio. To get the composition measurable, one more condition is needed.
            \begin{theorem}[Composition]
                Let $g$ be a measurable real valued function on a measurable $E$ and $f$ a continuous real valued function defined on $\RR$ 
            Then $f \circ g$ is measurable. 
            \end{theorem}
            We first prove a lemma.
            \begin{lemma}
                Let $f:E \to \RR$ be a function defined on a measurable set $E$. $f$ is measurable if and only if for each open sets $O$,
            $f^{-1}(O)$ is measurable.
            \end{lemma}
            \begin{proof}
                Suppose the inverse image of each open sets are measurable, then since $(a, +\infty)$ is open,
                $$E\{f(x)>a\} = f^{-1}(a,+\infty)$$ 
            is measurable, which suggests that $f$ is measurable.\par
                Suppose $f$ is measurable. Given any open sets, we can express it by a countable union of open intervals $\{I_n\}$.
            Suppose $I_n = (a_n, b_n)$. Then
            \begin{equation*}
                \begin{aligned}
                    f^{-1}(O) &= f^{-1}(\bigcup_{n=1}^{\infty} I_n)\\
                    &= \bigcup_{n=1}^{\infty} (f^{-1}(I_n))\\
                    &= \bigcup_{n=1}^{\infty} (E\{f>a_n\} \cap E\{f< b_n\})
                \end{aligned}
            \end{equation*}
            so the inverse image of $O$ is measurable 
            \end{proof}
            \begin{proof} (Proof of the Composition Theorem)
                Let $O \subset \RR$ be open, then we only need to show the inverse image of $O$ under $f \circ g$ is measurable. 
                $$(f \circ g )^{-1}(O) = g^{-1} (f^{-1}(O))$$
                Since $f$ is continuous, $f^{-1}(O)$ is open. By the measurability of $g$, $g^{-1}(f^{-1}(O))$ is measurable. By the 
            upper lemma, the proposition is proved.
            \end{proof}
            At the last of this section, we show that for finite many functions, their maximum function is still measurable.
            \begin{proposition}
                For a set of finite functions $\{f_n\}_{n=1}^{N}$, $max(f_1, \ f_2, \dots f_n)$ is measurable. 
            \end{proposition} 
            \begin{proof}
                We prove the case of $n=2$, and use the mathematic induction. 
                $$E\{max(f, \ f')< c \} = E\{f< c \} \cap E\{f'< c \}$$
                So it is measurable. The part of induction is rather trivial.
            \end{proof}
                The proposition for minimum can be proved in a simialr way. From the upper proposition, one can know 
                $$|f| = max(f, 0) - min(f,0)$$
                is measurable.

        \section{Simple function approximation of measurable function}
                \begin{definition}[Characteristic functions]
                    Given a subset $A$ of $\RR$, the function $\chi_A: \RR \to \RR$ defined by 
                    \begin{equation*}
                        \chi_A(x) = \left\{ \begin{matrix}
                            1  \ \ \ \ x \in A\\
                            0  \ \ \ \ x \notin A
                        \end{matrix} \right.
                    \end{equation*}
                    is called the characteristic function of $A$.
                \end{definition}
                It is clear that $\chi_A$ is measurable if and only if $A$ is measurable.
                \begin{proposition}[Some basic properties of characteristic function]
                    Let $A, \ B$ be sets, then 
                    \begin{equation*}
                        \begin{aligned}
                            &1. \ \ \chi_{A \cap B} = \chi_A \cdot \chi_B \\
                            &2.  \ \ \chi_{A \cup B} = \chi_A + \chi_B - \chi_A \cdot \chi_B\\
                            &3. \ \ \chi_{A^c} = 1 - \chi_A 
                        \end{aligned}
                    \end{equation*}\par
                \end{proposition}
                \begin{proof}
                    Here we prove only the first equality, the rest can be proved in the same manner, leaving as an exercise to the readers.\par
                    If $\chi_{A \cap B}(x) = 1$, then $x \in A$ and $x\in B$, so 
                $$\chi_A \cdot \chi_B = 1 \cdot 1 = 1$$
                If $\chi_{A \cap B}(x) = 0$, then $x \notin A \cap B$. So one of $\chi_A$, $\chi_B$ would be zero, which suggest 
                $$\chi_A \cdot \chi_B = 0$$ \par                   
                \end{proof}
                \begin{definition}[Simple function]
                    A real valued measurable function is \textbf{simple} if and only if it takes only finite values.
                Or in another word, a simple function is measurable and it's image is a finite subset of $\RR$.
                \end{definition}
                One can see immediately that $\chi_A$ is a simple function if it is measurable. We can express any simple function by sum 
            of characteristic functions.
                \begin{proposition}[Canonical representation of simple function]
                    If $\phi:E \to \RR$ is a simple function, then there exist a finite distinct sequence of numbers $\{a_n\}$ and a finite sequence of disjoint measurable sets $\{A_n\}$,
                such that 
                    \begin{equation}
                        \phi(x) = \sum_{n=1}^{N} a_n \chi_{A_n}(x) \label{sa}
                    \end{equation} 
                \end{proposition}
                \begin{proof}
                    Since $\phi$ takes only finite real values, we arrange these number into a sequence $\{a_n\}$. 
                Let $A_n$ be the set contains all the $x \in E$ such that $\phi(x) = a_n$. Because $\phi$ is measurable, so is $A_n$.
                \eqref{sa} is true for those sequence we have just found.
                \end{proof}
                It isn't hard to see that finite sum of simple functions is still simple function, scaler multiple of simple function is still simple 
            function. The proof of next proposition is left to the readers.
                \begin{proposition}[The linearity of simple functions]
                    Let $\varphi:E \to \RR, \ \psi:E \to \RR $ be simple functions, and $r$ be a real number. Then 
                    \begin{equation}
                            \varphi + \psi \ , \  \ c \cdot \varphi \mbox{ are simple function}
                    \end{equation}
                \end{proposition}
                \begin{lemma}[Simple approximation lemma]
                    Let $f:E \to \RR$ be a bounded measurable function. For all $\epsilon >0$, there exist
                simple functions $\varphi:E \to \RR, \ \psi:E \to \RR$, such that 
                    \begin{equation}
                        \begin{aligned}
                            \varphi \leq f \leq \psi\\
                            \psi - \varphi \leq \epsilon 
                        \end{aligned} \label{spl}
                    \end{equation} \par
                \end{lemma}
                \begin{proof}
                    Let $[c,d]$ be a close bounded interval, whose interior contains $f(E)$ and let $P = \{c=y_0, y_1, y_2, \dots y_n=d\}$ 
                be a partition of $[c,d]$ such that $y_i-y_{i-1} < \epsilon$. We define $E_i = E\{y_{i-1}\leq f<y_i\}$, and 
                \begin{equation}
                    \begin{aligned}
                        \varphi(x) &= \sum_{i=1}^{n} y_{i-1} \chi{E_i}(x)\\
                        \psi(x) &=  \sum_{i=1}^{n} y_{i} \chi_{E_i}(x)
                    \end{aligned}
                \end{equation} 
                Then it is easy to varify that these two functions satisfy \eqref{spl}
                \end{proof}
                \begin{theorem}[Simple approximation Theorem]
                    A extended real value function on a measuralbe set, say $f:E \to \RR^*$, is measurable if and only if there exists a 
                sequence of simple functions $\{\varphi_n\}$ which converge to $f$ on $E$ pointwisely, and has the property that
                $$|\varphi_n| \leq |f| \mbox{ on $E$ for all $n$}$$
                    More over, if the function is bounded, $\{\varphi_n\}$ can converge uniformly; if the function is nonegative, $\{\varphi_n\}$
                can converge increasingly.
                \end{theorem}
                \begin{proof}
                    Since each simple functions are measurable, and sequence of measurable functions converge to measurable function, so
                the "if" is proved. \par 
                    Assume f is measurable, and also assume $f\geq 0$ on $E$. Let $E_n = \{x \in E : f(x) \leq n\}$, then $f$ is nonegative bounded measurable function
                on $E_n$. So by simple approximation lemma, there exist $\varphi_n , \ \psi_n$, such that 
                $$\varphi_n \leq f \leq \psi_n$$
                $$\psi_n - \varphi_n \leq \frac{1}{n}$$ 
                on $E_n$. So 
                $$f - \varphi_n \leq \frac{1}{n}$$
                on $E_n$. We extend $\varphi_n$ on all $E$ by setting $\varphi_n(x) = n$ for $x$ such that $f(x) \geq n$.\par
                \textbf{Claim:} $\{\varphi_n\}$ converge to $f$. \par
                    If $f(x)$ is finite, then exists a $N$ such that $f(x) < N$. Then 
                    $$0< f(x) - \varphi_n(x) <\frac{1}{n} \mbox{ for $ n \geq N$}$$
                Thus $\varphi_n(x)$ converge to $f(x)$.\par
                    If $f(x) = \infty$, then $\varphi_n(x) = n$, thus we also have $\lim_{n\to \infty}\varphi_n(x) = \infty$.
                    Replacing each $\varphi_n$ by $max(\{\varphi_i\}_{i=1}^{n})$, then we have $\varphi_n$ converge increasingly.\par
                    If $f$ is not nonegative, let $f_+=max(f, 0), \ f_-= -min(f, 0)$, then $f = f_+ - f_-$. Since $f_+, \ f_-$ are nonegative 
                measurable functions, one can find two sequences of simple functions $\{\varphi_n\}, \ \{\psi_n\}$, whose limit are 
                $f_+$ and $f_-$ separately. Thus
                    \begin{equation}
                        \begin{aligned}
                            f &= f_+ - f_-\\
                            &= \lim_{n\to \infty} \varphi_n -lim_{n \to \infty} \psi_n\\
                            &= \lim_{n \to \infty} \varphi_n - \psi_n
                        \end{aligned}
                    \end{equation}
                    Since the sum of simple functions are still simple function, we find a sequence of simple function which converge to 
                $f$ pointwisely. \par
                    If $f$ is bounded, there exist a $M > |f|$. So $f(E) \subset [-M,M]$. Given $n>0$, We can choose a partition of $M, \ P=\{y_i\}_{i=1}^{n}$ 
                which is finer enough, such that $y_{i}-y_{i-1} <\frac{1}{n}$, and define $\varphi_n$ in the same manner. Then $\forall \epsilon > 0, \ \exists N$
                such that $\forall n \geq N, \ x \in E, \ |\varphi_n - f| \leq \epsilon$. 
                \end{proof}
        
        \section{Egoroff's Theorem \& Lusin's Theorem}
            Now we are able to prove the last two principles which have been mentioned in the beginning of this chapter. The result is 
        quite amazing. We will first deal with the Egoroff's theorem, which shows the thrid principle. In some sense, sequences of measurable 
        functions which converge pointwisely are nearly converge uniformly. To prove this theorem, we first need to know more about the
        convergence of measurable functions and measurable sets.
            \begin{proposition}[Continuity of measure]
                Let $\{A_n\}$ be a ascending sequence of sets, which means $A_1 \subset A_2 \subset \dots A_n \subset \dots$.
            Let $A = \bigcup_{n=1}^{\infty} A_n$. Then $$m(A) = \lim_{n \to \infty} m(A_n)$$
                If $\{B_n\}$ is descending, then let $B = \bigcap_{n=1}^{\infty} B_n$. We have 
                $$m(B) = \lim_{n \to \infty} m(B_n)$$

            \end{proposition}
            \begin{proof}
                We first prove the part of ascending sequence. If there is a index $k$ such that $m(A_{k}) = \infty$, then by the monotonicity of 
            measure $m(A) = \infty$, thus the inequality holds. If all $A_k$ are finite, we define $A_0 = \varnothing$, and $C_n = A_n -A_{n-1}$.
            Then$$A = \bigcup_{n=1}^{\infty} A_n =\bigcup_{n=1}^{\infty} C_n$$ 
            and $\{C_n\}$ are disjoint. For disjoint union of sets of finite measure, we can apply the countable additivity of measure, 
            and the excirsion property of the measurable set, 
            \begin{equation*}
                \begin{aligned}
                    m(A) = m(\bigcup_{n=1}^{\infty} C_n) &= \sum_{n=1}^{\infty} m(A_n - A_{n-1})\\
                    &= \sum_{n=1}^{\infty} m(A_n) - m(A_{n-1})
                    &= \lim_{n \to \infty} m(A_n) - m(A_0)
                    &=\lim_{n \to \infty} m(A_n)
                \end{aligned}
            \end{equation*}
                For descending sequence $\{B_n\}$, if for all $n$, $m(B_n) = \infty$, then?????????????? 
            
            \end{proof}
            \begin{proposition}[Pointwise limit of measurable functions]
                If a sequence of measurable functions $\{f_n\}$ converge pointwise a.e. on $E$, where $E$ is their domain. Then the limit 
            function $f$ to which $\{f_n\}$ converge, is measurable.     
            \end{proposition}
            \begin{proof}
                Suppose the $E_0$ is the subset which $f$ doesn't converge. By proposition \ref{MZS}, $f$ is measurable if and only if it's 
            ristriction on $E - E_0$ is measurable. So it is reasonable to assume that $\{f_n\}$ converge on $E$. 
                For all $x \in E\{f< c\}$, since $f$ is the pointwise limit, there exist a N and m such that 
            $$\forall n \geq N, \ f_n(x) < c - \frac{1}{m}$$
                Since $E\{f_n< c-\frac{1}{m}\}$ is measurable for all $m, \ n$, thus 
            $$\bigcap_{n=N}^{\infty} E\{f_n < c -\frac{1}{m}\}$$
            is measurable. Consequently, 
            $$E\{f(x)<c\} = \bigcup_{1 \leq N, m < \infty}(\bigcap_{n=N}^{\infty} E\{f_n < c -\frac{1}{m}\})$$
            is measurable. 
            \end{proof}
            \begin{theorem}[Egoroff's Theorem]
                If a sequence of measurable functions $\{f_n\}$ converge pointwise a.e. on $E$, where $E$ is their domain which has finite
            measure. Then $\forall \epsilon > 0$, there exist a closed $F \subset E$, such that $\{f_n\}$ converge uniformly on $F$, and
            $m(E-F) < \epsilon$. 
            \end{theorem}
            Since we want to prove the uniform convergence, the estimation $|f_n - f|< a$ will be important. Also, we need to find a closed
        set which is good enough. We will first find a measurable set which is good enough(which is easier), and then use closed set to approximate it. 
        To these purposes, it is conveneint to establish the following lemma.
            \begin{lemma}
                Under the assumptions of Egoroff's theorem, for each $a, b > 0 $, there is a measurable subset $A \subset E$ and a $N'$ such
            that 
            $$|f_n - f| < a \mbox{ on $A$ for all $n>N$ where } m(E -A) <b $$
            \end{lemma}
            \begin{proof}
                We can simplely assume $f_n \to f$ on $E$. By the previous discussion, one can know that $|f_n-f|$ is measurable function. 
            Let $$E_N = \bigcap_{n=N}^{\infty} E\{|f_n-f|<a\}$$
            which is a measurable subset of $E$. Since $\{f_n\}$ converge pointwisely to $f$, $\{E_N\}$ converge to $E$ as $n$ goes 
            to infinity. So by excirsion property, there exists a $N'$ such that $m(E - E_{N'}) = m(E) - m(E_{N'}) = b$. For the same $E_{N'}$, pick any $x$ in $E_{N'}$, we have 
            $$\forall n \geq N', \ |f_n - f| < a$$ 
            So $A = E_{N'}$ is the set we want. 
            \end{proof}
                \begin{proof}\textbf{(Proof of the Egoroff's Theorem)}
                Given $\epsilon > 0$, let $A = \bigcap_{m=1}^{\infty} A_m$, where $A_m$ is the set in the above lemma 
            such that there exists a $N$, for $n\geq N$, $|f_n -f|< \frac{1}{m}$ on $A_m$, and $m(E - A_m)< \frac{\epsilon}{2^{m+1}}$.
            It is easy to see that $A$ is measurable subset of $E$. And we also have 
            $$m(E-A) = m(E-(\bigcap_{m=1}^{\infty}A_m)) = m(\bigcap_{m=1}^{\infty} (E-A_m)) \leq \sum_{m=1}^{\infty} m(E - A_m)< 
            \sum_{m=1}^{\infty} \frac{\epsilon}{2^{m+1}} = \frac{\epsilon}{2}$$    
            $$m(E-A)< \frac{\epsilon}{2}, \ \ \ \forall \epsilon' >0, \ \forall x \in A, \ \exists N \ s.t. \
            \forall n >N , \ |f_n(x) - f(x)|< \epsilon' $$
            which means $f_n$ converge uniformly on $A$. Using the close inner approximation of measurable set, we can find a close 
        subset of $A$, say $F$, such that, $m(A-F) = \frac{\epsilon}{2}$
            Finally
            $$m(E-F) = m((E-A) \cup (A - F)) = m(E - A) +m(A - F) = \epsilon$$
            and $\{f_n\}$ converge uniformly on $F$.
            \end{proof}
            This amazing theorem is very powerful. For pointwise converge sequence of functions, you can use Egoroff's theorem to get a 
        strong condition on a very big subset. One thing need to be paid attention on is that we can only apply Egoroff's theorem 
        on the functions which are defined on finite measure sets. In mathematic analysis, uniform converge can preserve the integrability, continuity.
        The next theorem is the proof of the Littlewood second principle, which is related to continuity. We will see how Egoroff's theorem 
        work. \par
            The main idea of the prove of Lusin's theorem is to first prove the case of simple function, then use simple functions to 
        approximate measurable functions. Since for 
            \begin{theorem}[Lusin's Theorem]
                Let $f$ be a measurable function which is finite a.e. on $E$, a measurable set. Then given $\delta > 0$, 
            there exist a closed subset $F \subset E$, such that $m(E-F)< \delta$, $f$ is conintuous on $F$.
            \end{theorem}
            \begin{proof}
                Suppose $f$ is finite. First we consider the case of $f(x) = \sum_{n=1}^{N} a_n \chi_{E_n}(x)$ is a simple function,
            where the sum is the canonical representation. 
            Let $F_n$ be close subset of $E_n$, by inner approximation, we can require $m(E_n-F_n) = \frac{\delta}{n}$. On each $E_n$,
            $f$ is a constant function, so is it on each $F_n$. Let $F = \bigcup_{n=1}^{N} F_n$. Being a finite union of close sets, $f$
            is closed. Then
                \begin{equation*}
                    \begin{aligned}
                        m(E-F) &= m(\bigcup_{n=1}^{N} E_n - \bigcup_{n=1}^{N} F_n)\\
                        &\leq m(\bigcup_{n=1}^{N} E_n - F_n)\\
                        &= \sum_{n=1}^{N} m(E_n - F_n)\\
                        &\leq \sum_{n=1}^{N} \frac{\delta}{n} = \delta
                    \end{aligned}
                \end{equation*}
            where the first inequality is come from the fact that $\bigcup_{n=1}^{N} E_n - \bigcup_{n=1}^{N} F_n \subset \bigcup_{n=1}^{N} E_n - F_n $.
            Thus we proved the case of $f$ is simple functions. \par
                Suppose $m(E)<+\infty$. By simple approximation of measurable function, we can find a sequence of simple functions $\{\varphi_i\}$,
            which converge to $f$ pointwisely. By Egoroff's theorem, there exists a close subset $F \subset E$, such that $\{\varphi_i\}$ coneverge 
            uniformly on $F$ to $f$, while $m(E-F) < \frac{\epsilon}{2}$. \par
                From the previous part of this proof, for every $\{\varphi_i\}$, we can find a closed set of $E$, say $F_i$, 
            such that $\varphi_i$ is continuous on $F_i$, while $m(E-F_i) < \frac{\epsilon}{2^{i+1}}$. We can see that, all $\varphi_i$
            is continuous on $\bigcap_{i=1}^{\infty} F_i$, which is a close set, and we have 
            \begin{equation*}
                \begin{aligned}
                    m(E - (\bigcap_{i=1}^{\infty} F_i)) &= m(\bigcup_{i=1}^{\infty} E - F_i)\\
                    &\leq \sum_{i=1}^{\infty} m(E - F_i)\\
                    &\leq \sum_{i=1}^{\infty} \frac{\epsilon}{2^{i +1}}\\
                    &= \frac{\epsilon}{2}
                \end{aligned}
            \end{equation*}\par 
                On $F \cap (\bigcap_{i=1}^{\infty} F_i)$, which is a close set, $\{\varphi_i\}$ is continuous and uniformly converge to
            $f$. By knowledge from mathematic analysis, $f$ is continuous on $F \cap (\bigcap_{i=1}^{\infty} F_i)$. It only remains to show that 
            $m (E - (F \cap (\bigcap_{i=1}^{\infty} F_i)))$ is small. 
            \begin{equation*}
                \begin{aligned}
                    m (E - (F \cap (\bigcap_{i=1}^{\infty} F_i))) &= m ((E - F) \cup (E-\bigcap_{i=1}^{\infty} F_i))\\
                    &\leq m (E - F) + m(E-\bigcap_{i=1}^{\infty} F_i)\\
                    &\leq \frac{\epsilon}{2} +\frac{\epsilon}{2}\\
                    &=  \epsilon
                \end{aligned}
            \end{equation*}
            Then we proved the case that $E$ has finite measure.\par
                For the case that $m(E) = +\infty$, we consider the ristirction of $f$ on $E_n = E \cap [-n,n]$. Clearly, $f|_{E_n}$ are  
            measurable functions defined on finite measure sets, so by the Lusin's theorem for finite case, we can find $F_n$ for each 
            $E_n$ such that $f = f|_{E_n}$ is continuous on $F_n$, and $m(E_n-F_n) < \frac{\epsilon}{2^{n+1}}$.\par
                Let $F = \bigcup_{n=1}^{\infty} F_n$, which is measurable. By our construction, $f$ is continuous on $F$. With the fact that 
            $\bigcup_{n=1}^{\infty}E_n - \bigcup_{n=1}^{\infty}F_n \subset \bigcup_{n=1}^{\infty}E_n -F_n$ we can obtain the estimation below,
            \begin{equation*}
                \begin{aligned}
                    m(E-\bigcup_{n=1}^{\infty} F_n) &= m(\bigcup_{n=1}^{\infty}E_n - \bigcup_{n=1}^{\infty}F_n)\\
                    &\leq m( \bigcup_{n=1}^{\infty}E_n - F_n)\\
                    &\leq \sum_{n=1}^{\infty}m(E_n - F_n)\\
                    &\leq \sum_{n=1}^{\infty} \frac{\epsilon}{2^{n+1}}\\
                    &= \frac{\epsilon}{2} 
                \end{aligned}
            \end{equation*}
                But this is not the end, since $F$ is not a close set. According to the inner approximation of measurable function, we can
            find a close set of $F$, say $F'$, such that $m(F-F')< \frac{\epsilon}{2}$. Then 
            \begin{equation*}
                \begin{aligned}
                    m(E-F') &= m ((E-F) \cup (F-F'))\\
                    &= m(E-F) + m(F-F')\\
                    &=\frac{\epsilon}{2} +\frac{\epsilon}{2}
                    &=\epsilon 
                \end{aligned}
            \end{equation*}
            \end{proof}
            The continuous function on close set can be extended to $\RR$, so we can have the following form of Lusin' theorem.
            \begin{theorem}[Lusin's theorem: another form]
                Let $f$ be a measurable function which is finite a.e. on $E$, a measurable set. Then given $\delta > 0$, 
            there exist a closed subset $F \subset E$, and a continuous function $g: \RR \to \RR$, such that $m(E-F)< \delta$, 
            $f = g$ on $F$.
            \end{theorem}   
    
    \chapter{Lebesgue Integration}
        Now all the preparations have done, it is the time to define the integral. We will first define the the integral for simple functions
    on sets of finite measure; then for the bounded measurable functions on finite measure sets; then for nonegative measurable functions on measurable sets;
    finally the general measurable functions over measurable sets. Without other modifier, the word "integral" in this chapter means Lebesgue
    integral.
        \section{Integral of Simple Functions}
            Remember that in the previous section, we have show that a simple function $\varphi:E \to \RR$ can be written in this form
        $$\varphi(x)  = \sum_{n=1}^{N} a_n \chi{E_n}$$
        where $E_n = E\{f=a_n\}$. This way of writting simple functions suggest a natrual way to define the integral for them.
            \begin{definition}[The integral for simple functions]
                Let $\varphi:E \to \RR$ be a simple function, where $E$ is a measurable set. We define the integral of $\varphi$ over $E$
                \begin{equation}
                    \int_E \varphi = \sum_{n=0}^{N} a_n \cdot m(E_n)
                \end{equation}
            where $\sum a_n \chi_{E_n}(x)$ is the canonical representation of $\varphi$. 
            \end{definition}
            \begin{lemma}\label{thm:simpleintlemma}
                Let $\{E_i\}$ be a finite disjoint of subsets of a finite measure set $E$. Let $\{a_i\}$ be a sequence of real number, then
            $$\varphi = \sum_{i=1}^n a_i \chi_{E_i} \mbox{ on } E \ \Rightarrow \ \int_E \varphi = \sum_{i=1}^{n} a_i \cdot m(E_i)$$\par
                \textbf{Remark:} One may wonder why we should prove this(it just seems like have no difference to the definition of integral).
            Since $\{a_i\}$ is not distinct, $\sum_{i=1}^n a_i \chi_{E_i}$ may not be the canonical representation of $\varphi$.\par
            \end{lemma}
            \begin{proof}
                Let $\{\lambda_j\}$ be distinct values taken by $\varphi$ and $A_j=\{x:\varphi(x)=\lambda_j\}$. By the definition of integral,
            $$\int_E \varphi = \sum_{j=1}^{m} \lambda_j \cdot m(A_j)$$ 
            Let $I_j = \{i:a_i =\lambda_j\}$. It isn't hard to see that $\sum_{i \in I_j} m(E_i) = A_j$
            \begin{equation}
                \begin{aligned}
                    \sum_{i=1}^{n} a_i \cdot m(E_i) &= \sum_{j=1}^{n} (\lambda_j \cdot \sum_{i \in I_j} m(E_i))\\
                    &= \sum_{j=1}^{n} (\lambda_j \cdot m(A_i))\\
                    &= \int_E \varphi
                \end{aligned}
            \end{equation}
            \end{proof}
            Of course we want to know the properties of the general Lebesgue integral, here is our first step in this direction.
            \begin{proposition}[The linearity of the intergration: simple function]
                Let $\varphi, \psi$ be simple functions on $E$, which is a set of finite measure. Then for $\alpha, \ \beta \in \RR$,
                \begin{equation}
                    \int_E (\alpha \varphi + \beta \psi) = \alpha \int_E \varphi + \beta \int_E \psi \label{lsf}
                \end{equation}
            \end{proposition} 
            \begin{proof}
                For linearity, we first prove $\int_E \alpha \varphi = \alpha \int_E \varphi$(scaler can 'go though' the integral sign), then prove \eqref{lsf} for the case of
            $\alpha=\beta = 1$. We suppose $\varphi = \sum_{i=1}^n a_i \chi_{E_i}$ and $\psi = \sum_{i=1}^m b_i \chi_{F_i}$.
            \begin{equation}
                \begin{aligned}
                    \int_E \alpha \varphi &= \sum_{i=1}^{n} \alpha \cdot a_i m(E_i) \\
                    &= \alpha \sum_{i=1}^{n} a_i m(E_i)\\
                    &= \alpha \int_E \varphi
                \end{aligned}
            \end{equation}
                Let $A_{ij} = E_i \cap F_j$, where $1\leq i\leq n$, $1 \leq j \leq m$. For all $x \in E$, $x \in E_i$ and $x \in F_j$ for 
            some $i, \ j$, and since $\{E_i\}$ and $\{F_j\}$ are disjoint, so $\{A_{ij}\}$ is a disjoint finite collection of subset of $E$
            whose union is $E$. Let $a_{ij}, b_{ij}$ be the value of $\varphi, \ \psi$ on $A_{ij}$. By lemma \ref{thm:simpleintlemma},
            \begin{equation*}
                \begin{aligned}
                    \int_E \varphi &= \sum_{1\leq i \leq n, \ 1 \leq j \leq m} a_{ij} m(A_{ij}) \\
                    \int_E \psi &= \sum_{1\leq i \leq n, \ 1 \leq j \leq m} b_{ij} m(A_{ij}) \\
                    \Rightarrow \int_E \varphi + \int_E \psi &= \sum_{1\leq i \leq n, \ 1 \leq j \leq m} a_{ij} m(A_{ij}) +\sum_{1\leq i \leq n, \ 1 \leq j \leq m} b_{ij} m(A_{ij}) \\
                    &=\sum_{1\leq i \leq n, \ 1 \leq j \leq m} (a_{ij} + b_{ij}) m(A_{ij})\\
                    &= \int_E \varphi +\psi 
                \end{aligned}
            \end{equation*}
            \end{proof}
            \begin{proposition}[Monotonicity of the integral]
                Let $\varphi, \psi$ be simple functions on $E$, which is a set of finite measure. Then 
            $$\varphi \leq \psi \Rightarrow \int_E \varphi \leq \int_E \psi$$
            \end{proposition}
            \begin{proof}
                Just like what we did in the previous proposition, we define $A_{ij}, \ a_{ij}, \ b_{ij}$ such that 
            \begin{equation*}
                \begin{aligned}
                    \int_E \varphi &= \sum_{1\leq i \leq n, \ 1 \leq j \leq m} a_{ij} m(A_{ij}) \\
                    \int_E \psi &= \sum_{1\leq i \leq n, \ 1 \leq j \leq m} b_{ij} m(A_{ij}) 
                \end{aligned}
            \end{equation*}
                It is easy to see that $a_{ij} \leq b_{ij}$, so is their finite sum.
            \end{proof}

        \section{Integral for Bounded Measurable Functions over Bounded Sets}
            In this section, we define the integral over bounded sets for bounded measurable functions out of the integral for simple function.  
            \begin{definition}[Upper integral, lower integral and integral]
                Let $f:E \to \RR$ be a Bounded measurable function over a finite measure set. We define
                \begin{equation*}
                        \begin{aligned}
                            &\sup{\int_E \varphi: \varphi \mbox{ is simple and } \varphi \leq f}\\
                            &\sup{\int_E \psi: \psi \mbox{ is simple and } \psi \geq f}
                        \end{aligned}
                \end{equation*}
            to be the upper integral and the lower integral of $f$ respectively, and denote them as 
            $$\sup \int_E f, \ \ \inf \int_E f $$\par
                If upper integral and lower integral of $f$ on $E$ are equal, we define there common value to be the \textbf{Lebesgue integral}
            of $f$ on $E$. 
            \end{definition}
            \begin{theorem}[Lebesgue integral is a generalization of Riemann integral]
                If a bounded function $f$ defined on close bounded interval $[a,b]$ is Riemann integrable, then it is Lebesgue integrable,
            and two integrals are equal.
            \end{theorem}
            In this theorem, we use $(R)\int$ to represent Riemann integral. 
            \begin{proof}
                Since $f$ is Riemann integrable on $[a,b]$, it means 
            $$\sup \{(R)\int_{[a,b]} \varphi(x):\varphi(x)\leq f \mbox{ is a step function }\}
            = \inf \{(R)\int_{[a,b]} \psi(x):\psi(x)\geq f \mbox{ is a step function }\}$$
            Each step function is a simple function, and we can easily observe that the Riemann integral and Lebesgue integral are equal(
            since the lenth of the interval is the Lebesgue measure of the interval).
            We can find a sequence of step functions, which are measurable functions, converge to $f$ pointwisely. So $f$ is a measurable 
            function. The Riemann integralbility also indicate that $f$ is bounded. Then by the definition of Lebesgue integral for bounded
            measurable functions, the Lebesgue integral exist and $$\int_{[a,b]} f = (R) \int_{[a,b]} f$$                
            \end{proof}
            \begin{theorem}[Bounded measurable functions are integrable]
                Let $f$ be a bounded measurable function which is defined on a set with finite measure, say $E$.
            Then $f$ is measurable.
            \end{theorem}         
            \begin{proof}
                Since $f$ is bounded and measurable, given $\epsilon> 0$, we can find two sequence of simple functions $\{\varphi_n\}, \ \{\psi_n\}$, such that 
            \begin{equation*}
                \varphi \leq f \leq \psi 
            \end{equation*}
            and $\psi - \varphi < \frac{\epsilon}{m(E)}$. By the definition of upper and lower integral and the linearity of 
            the integral for simlpe functions  
            \begin{equation*}
                \begin{aligned}
                    \int_E \varphi \leq \inf \int_E f &\leq \sup \int_E f \leq \int_E \psi\\ 
                    \Rightarrow \sup \int_E f - \inf \int_E f &\leq \int_E \psi - \int_E \varphi\\
                    &= \int_E (\psi - \varphi)\\
                    &\leq \int_E \frac{\epsilon}{m(E)} \\
                    &=\epsilon 
                \end{aligned}
            \end{equation*} 
                Then by the arbritariness of $\epsilon$, $\sup \int_E f = \inf \int_E f$, thus the theorem is proved.
            \end{proof}
            
            In fact, a bounded function over finite measure set is integrable if and only if it is measurable, we will prove this later.
        Now we can establish the theorem for the linearity and monotonicity of the integral of bounded measurable functions.
            \begin{theorem}[Linearity]
                Let $E$ be a finite measure set, and $f, \ g$ be bounded measurable function define on $E$. Let $\alpha, \ \beta$ be
            real numbers. Then
                \begin{equation}
                    \alpha \int_E f + \beta \int_E g = \int_E \alpha f + \beta g
                \end{equation}
            \end{theorem}
            \begin{proof}
                We first prove that $$\int_E \alpha f = \alpha \int_E f$$, then prove $$\int_E f + \int_E g = \int_E f+g $$
            Suppose $\alpha>0$(the argument for $\alpha <0$ is similar), by simple approximate lemma, given $\epsilon$, there exist simple
            functions $\varphi, \ \psi$, such that $\psi - \varphi < \frac{\epsilon}{\alpha \cdot m(E)}$
                \begin{equation*}
                    \begin{aligned}
                        \varphi \leq &f \leq \psi \\
                        \alpha \varphi \leq \alpha&f \leq \alpha\psi \\
                    \end{aligned}
                \end{equation*}
            By the upper theorem, $\alpha f$ is integrable. And by the definition of the integral, the linearity of the integral of simple functions,
            \begin{equation*}
                \begin{aligned}
                    \int_E \alpha \varphi \leq &\int_E \alpha f \leq \int_E \alpha \psi\\
                    \Rightarrow \alpha \int_E \varphi \leq &\int_E \alpha f \leq \alpha \int_E \psi\\ 
                \end{aligned}
            \end{equation*}
            Notice that we also have 
            $$\alpha \int_E \varphi \leq \alpha \int_E f \leq \alpha \int_E \psi$$
            So 
            \begin{equation*}
                \begin{aligned}
                    |\alpha \int_E f - \int \alpha f| &\leq \alpha \int_E \psi - \alpha \int_E \varphi \\
                    &\leq \alpha \int_E \psi - \varphi \\
                    &\leq \alpha \int_E \frac{\epsilon}{\alpha \cdot m(E)}\\
                    &\leq \alpha \cdot m(E) \cdot \frac{\epsilon}{\alpha \cdot m(E)} = \epsilon 
                \end{aligned}
            \end{equation*}
            Thus the first part of the theorem is proved. \par 
                Since the sum of measurable function is still measurable, $f+g$ is integrable over $E$. Suppose 
            $\varphi_1, \ \varphi_2, \ \psi_1, \ \psi_2$ are simple functions such that 
            $$\varphi_1 \leq f \leq \psi_1, \ \varphi_2 \leq g \leq \psi_2$$
            We can see that 
            \begin{equation*}
                \begin{aligned}
                    \int_E f+ g &= \sup \int_E f+g \\
                    &\leq \int_E \psi_1 + \psi_2\\
                    &= \int_E \psi_1 + \int_E \psi_2\\
                \end{aligned}
            \end{equation*}
            Since the above inequality is true for all $\psi_1 > f$, $\psi_2 > g$, we have 
            \begin{equation*}
                \begin{aligned}
                    \int_E f+ g &\leq \inf_{\psi_1 \geq f}\int_E \psi_1 +  \inf_{\psi_2 \geq g}\int_E \psi_2\\
                    &= \int_E f +\int_E g 
                \end{aligned}
            \end{equation*}
            On the other hand, 
            \begin{equation*}
                \begin{aligned}
                    \int_E f+ g &= \inf \int_E f+g \\
                    &\geq \int_E \varphi_1 + \varphi_2\\
                    &= \int_E \varphi_1 + \int_E \varphi_2\\
                \end{aligned}
            \end{equation*}
            This inequality is also true for all $\varphi_1 > f$, $\varphi_2 > g$, thus 
            \begin{equation*}
                \begin{aligned}
                    \int_E f+ g &\geq \sup_{\varphi_1 \leq f}\int_E \varphi_1 +  \sup_{\varphi_2 \leq g}\int_E \varphi_2\\
                    &= \int_E f +\int_E g 
                \end{aligned}
            \end{equation*}
            Together the two inequality, we have 
            $$\int_E f+ g = \int_E f +\int_E g $$.
            Thus we prove the linearity of the integration for bounded measurable functions.
            \end{proof}
            \begin{theorem}[Monotonicity]
                Let $E$ be a finite measure set, and $f, \ g$ be bounded measurable function define on $E$. If $f \leq g$ on $E$, then 
                \begin{equation}
                    \int_E f \leq \int_E g
                \end{equation}
            \end{theorem}
            \begin{proof}
                Let $h = g-f$, which is a non-negative measurable function. By linearity, 
                $$\int_E g - \int_E f  = \int_E g-f = \int_E h$$
                Since $h$ is non-negative, $h\geq 0$. Let $\psi = 0$, we can see that 
                $$\int_E h \geq \int \psi = 0 $$
            which shows that $$\int_E g - \int_E f  \geq 0$$
            Thus the monotonicity of the integration of bounded measurable functions is proved.
            \end{proof}
            By the Monotonicity, and the fact that $-|f|\leq f\leq |f|$we can get the following useful conclusion.
            \begin{corollary}[Absolute value]
                Let $f$ be a bounded measurable function defined on a set of finite measure. Then
                $$\int_E |f| \geq | \int_E f |$$ 
            \end{corollary}
            \begin{proposition}[The integral over measure zero set is zero]\label{The integral over measure zero set is zero}
                Suppose $f$ is bounded measurable function on a measure zero set $ E$, then 
                $$\int_E f = 0$$
            \end{proposition}
            \begin{proof}
                Since $f$ is bounded, there exist a $M$, such that $|f|< M$ on $E$. So by monotonicity of the integral, we have 
                \begin{equation*}
                    \begin{aligned}
                        \int_E |f| &\leq \int_E M\\
                        &= M \cdot m(E) 
                        &= 0 
                    \end{aligned}
                \end{equation*}
                By previous corollary, the proposition is proved 
            \end{proof}
            Now we proved the last basic property of the integral: the additivity over domain. 
            \begin{lemma}
                Let $f$ be a bounded measurable function defined on $E$, and $E_0$ be a measurable subset. Then   
                \begin{equation*}
                    \int_{E_0} f = \int_E f \cdot \chi_{E_0}
                \end{equation*}
            \end{lemma}
            \begin{proof}
                For all simple functions $\varphi \leq f$, we exetend $\varphi$ to a new simple function on $E$ by letting 
            $\varphi'(x) = 0$ for $x \in E-E_0$. Notice that the integral of two functions are the same: 
            $$\int_{E_0} \varphi = \int_E \varphi'$$
            For $x \in E_0$, $\varphi \leq f|_{E_0} = f \cdot \chi_{E_0}$; for $x \in E-E_0$, $\varphi = 0 \leq f\cdot \chi_{E_0}$. Thus  
            \begin{equation*}
                \begin{aligned}
                    \int_{E_0} f &= \inf_{\varphi \leq f} \int_{E_0} \varphi \\
                    &= \inf_{\varphi' \leq f} \int_E \varphi' \\
                    &\leq \int_E f \cdot \chi_{E_0}
                \end{aligned}
            \end{equation*}
                For simple functions $\psi \geq f$, the extension $\psi'$ on $E$ with $\psi'(x) = 0$ for $x \in E-E_0$, is also a simple 
            funcion. For $x \in E_0$, $\psi' \geq f|_{E_0} = f \cdot \chi_{E_0}$; for $x \in E-E_0$, $\psi' = 0 \geq f\cdot \chi_{E_0}$.   
            \begin{equation*}
                \begin{aligned}
                    \int_{E_0} f &= \sup_{\psi \geq f} \int_{E_0} \psi \\
                    &= \inf_{\psi' \geq f} \int_E \psi' \\
                    &\geq \int_E f \cdot \chi_{E_0}
                \end{aligned}
            \end{equation*}
            Together the two inequality, we proved the theorem. 
            \end{proof}
            \begin{theorem}[The additivity over domain]
                Let $f$ be bounded measurable functions over $E$, which is a finte measure set. Suppose $A$ and $B$ are disjoint subsets
            of $E$, then 
            $$\int_{A \cup B} f = \int_{A} f + \int_B f$$
            \end{theorem}
            \begin{proof}
                By the linearity of the integral, and the lemma above,
                \begin{equation*}
                    \begin{aligned}
                        \int_A f +\int_B f &= \int_{A \cup B} f \cdot \chi_A +\int_{A \cup B} f\cdot \chi_B\\
                        &= \int_{A \cup B} f \cdot \chi_A + f\cdot \chi_B\\
                        &= \int_{A \cup B} f \cdot \chi_{A\cup B}\\
                        &= \int_{A \cup B} f 
                    \end{aligned}
                \end{equation*}
            \end{proof}
            From the additivity over domain, we can derive a interesting result, which can enhance many propositions which involving integral.
            \begin{corollary}[Measure zero set does not influence the integral]
                Let $f, \ g$ be bounded measurable function on $E$, which is a finite measure set. If $f = g$ a.e. on $E$, 
                $$\int_E f = \int_E g$$
            \end{corollary}
            \begin{proof}
                Let $E_0$ be the set that $f \neq g$. By additivity over domian,
                \begin{equation*}
                    \begin{aligned}
                        \int_E f &= \int_{E_0} f + \int_{E-E_0} f\\
                        \int_E g &= \int_{E_0} g + \int_{E-E_0} g\\
                        \Rightarrow \int_E g - \int_E f &= \int_{E_0} g - \int_{E_0} f
                    \end{aligned}
                \end{equation*}
                Since the integral over measure zero set is measure zero, the corollary is proved.
            \end{proof}
        
        \section{The Lebesgue Integral for Non-negative Functions}
            In this section, we push the definition of integral further, to non-negative measurable functions over general measurable 
        sets. We allowd the function to take large value(not being bounded anymore, but don't take values at infinity), and define it on a larger space(without requiring
        $E$ to have finite measure).\par  
            To this purpose, it is convenient to establish the concept of \textbf{finite support functions}.
            \begin{definition}[Support]
                For a function $f:E \to \RR^*$, where $E$ is a subset of $\RR$, we say it's \textbf{support} $E_0$ is the closure of 
            $$\{x\in E: f(x) \neq 0\}$$
                If the support is finite, then we say $f$ is a function has finite support, or compact support(notice that $E_0$ is closed 
            and bounded, thus compact).
            \end{definition}\par
            It is not hard to see that the linear combination of finite support functions are still finite support.\par 
            We haven't defined the integral on a set which has infinite measure. Suppose $f:E\to \RR$ is bounded, measurable
        and has a finite support $E_0$, then $$\int_E f := \int_{E_0} f$$
        is a very natrual definition. Any thing we proved in the last section can be applied on this new definition. For non-negative 
        measurable function, we can define it's integral through the integral of bounded measurable function of finte support.
            \begin{definition}[Integral of non-negative measurable functions]
                Let $f:E \to \RR^*$ be a non-negative measurable function defined on a measurable set. The integral of $f$ is defined as 
            \begin{equation}
                \int_E f := \sup\{\int_E \varphi : \varphi \mbox{ is bounded measurable and with finite support }; 0\leq \varphi \leq f\}
            \end{equation}
            \end{definition}
            To prove the properties of the integral of non-negative functions, \textbf{Chebychev's inequality} is needed.
            \begin{lemma}[Chebychev's inequality]
                Let $f$ be a non-negative function on $E$. Then for any $\lambda >0$,
                \begin{equation}
                    m(\{x \in E: f(x) \geq \lambda\}) \leq \frac{1}{\lambda} \cdot \int_E f
                \end{equation}
            \end{lemma}
            \begin{proof}
                We denoted $\{x \in E: f(x) \geq \lambda\}$ as $E_{\lambda}$. Suppose $g$ is defined by the following:
            $$g(x) := \left\{ \begin{matrix}
                \lambda & \mbox{ if } x \in E_{\lambda}\\
                0 & \mbox{ if } x \in E- E_{\lambda}
            \end{matrix} \right.$$
            Then $g \leq f$. If $m(E_{lambda})$, $g$ is measurable bounded with finite support. By definition, 
            \begin{equation*}
                \begin{aligned}
                    \int_E g &\leq \int_E f\\
                    \lambda \cdot m(E_{\lambda}) &\leq \int_E f\\
                    m(E_{\lambda}) &\leq \frac{1}{\lambda}\int_E f\\
                \end{aligned}
            \end{equation*} 
            If $m(E_{\lambda}) = +\infty$, let $E_{\lambda, n} := E_{\lambda} \cap [-n,n]$. Then $\psi_n = \chi_{E_{\lambda, n}}$
            is a bounded measurable function with finite support. And we have $\psi_n \leq f$. By continuity of the measure,
            \begin{equation*}
                \begin{aligned}
                    \lambda \cdot m(E_{\lambda}) &= \lambda \lim_{n \to \infty} E_{\lambda, n}\\ &= \int_E \psi_n\\ &\leq \int_E f\\
                \end{aligned}
            \end{equation*}
            Times $\frac{1}{\lambda}$ on the both side of the inequality, then we obtained the desirable conclusion.                 
            \end{proof}
            \begin{proposition}[Non-negative function and measure zero set]\label{Non-negative function and measure zero set}
                Let $f$ be a non-negative measurable function on $E$, then 
                $$\int_E f = 0$$
                if and only if $f= 0$ a.e. on $E$.
            \end{proposition}
            \begin{proof}
                If $f = 0$ a.e. on $E$, it is bounded, measurable, finite support. Then by the conclusion in the last section, 
            $$\int_E f = 0$$\par
                Suppose $\int_E f = 0$, and there exist a $E' \subset E$ such that $m(E') > 0 $ and $f > 0$ on $E'$. We can see that 
            $$E' = \bigcup_{n=1}^{\infty} \{x\in E: f(x) \geq \frac{1}{n} \}$$
            Since countable union of measure zero set is still measure zero, there exist $N$ such that 
            $m(\{x\in E: f(x) \geq \frac{1}{n} \}) > 0 $. By Chebychev inequality, 
            $$m(\{x\in E: f(x) \geq \frac{1}{n} \}) \leq n \int_E f = 0$$
            we arrive at a contradiction.
            \end{proof}
            \begin{theorem}[Linearity]
                Let $f$ and $g$ be two non-negative measurable functions on a measurable set, say $E$. Let $\alpha$, $\beta$ be two non-negative
            real numbers. Then
            $$\int_E \alpha f + \beta g = \alpha\int_E f + \beta \int_E g$$
            \end{theorem}
            \begin{proof}
                With ordinary method, we first prove the part of scalar multiple, then the part of addition.\par
                $\int_E \alpha f = \alpha \int_E f$: Notice that for $\alpha > 0 $, $0 \leq h \leq f$ if and only if $0<\alpha h < \alpha f$.
            Thus by the linearity of finite support, 
            \begin{equation*}
                \begin{aligned}
                    \int_E \alpha f &= \sup_{h' < \alpha} \int_E h'\\
                    &=\sup_{h < f} \int_E \alpha h \\
                    &= \alpha \sup_{h < f} \int_E h \\
                    &=\alpha \int_E f 
                \end{aligned}
            \end{equation*}
            where $h$ is bounded and measurable, with finite support. Thus the part of scalar multiple is done.\par
            Let $h , \ k$ be bounded and measurable with finite support, then their sum is also bounded and measurable, with finite support.
            By the definition of the integral,
            \begin{equation*}
                \begin{aligned}
                    \int_E h + \int_E k &=\int_E h + k\\
                    &\leq \int_E f+g 
                \end{aligned}
            \end{equation*}
            Thus for the supremum of the left side, the inequality still holds.
            $$\int_E f +\int_E g \leq \int_E f+g$$
            It remains to prove the inequality in the opposite side. Since $\int_E f+g$ is the supremum of $\int_E l$, where $l \leq f+g$
            is a bounded measurable function with finite support. So it is surffice to  prove the inequality for all such $l$ in place of 
            $f+g$.\par
                For a such $l$, let $h = min(f, l), \ k= l-h$. Then if $l(x) \leq f(x)$, $h(x) = l(x)$ and $k(x)= 0$; if $l(x) > f(x)$,
            $h(x)=f(x)$ and $k(x) = l(x)-f(x) < g(x)$. Then we can see that $h \leq f$, $k \leq g$. Thus by the additivity of the integral 
            for bounded measurable function of finite support,
            \begin{equation*}
                \begin{aligned}
                    \int_E l &= \int_E l + min(f, l) - min(f,l)\\
                    &=  \int_E min(f, l) + \int_E l - min(f,l)\\
                    &= \int_E h + \int_E k \\
                    &\leq \int_E f + \int_E g\\
                \end{aligned}
            \end{equation*}
            Thus proved the the part of additivity.
            \end{proof}
            \begin{theorem}[Monotonicity]
                Suppose $f$ and $g$ are non-negative measurable functions on measurable set, such that $ f \leq g$ on their domain $E$.
            Then
            $$\int_E f \leq \int_E g$$
            \end{theorem}
            \begin{proof}
                Notice the fact that for all bounded measurable functions $\varphi$ on $E$, which satisfy $\varphi \leq f$, we have
            $\varphi \leq g$. So we know that 
                \begin{equation*}
                    \begin{aligned}
                        \sup_{\varphi <f} \int_E \varphi &\leq \sup_{\psi <g} \int_E \psi\\ 
                        \int_E f &\leq \int_E g 
                    \end{aligned}
                \end{equation*}
                Thus the theorem is proved.
            \end{proof}
            \begin{theorem}[Additivity over domain]
                Let $f$ be a non-negative measurable function over a measurble set $E$. Suppose $E_1, E_2$ are disjoint measurable subsets,
            then we have
            $$\int_{E_1 \cup E_2 }f = \int_{E_1} f + \int_{E_2} f$$            
            \end{theorem}
            \begin{proof}
                Observing that $f = f\chi_{E_1} +f\chi_{E_2}$ on $E_1 \cup E_2$, by linearity of the integral, 
                \begin{equation*}
                    \begin{aligned}
                        \int_{E_1\cup E_2} f &= \int_{E_1\cup E_2} f\chi_{E_1} +f\chi_{E_2}\\
                        &= \int_{E_1\cup E_2} f\chi_{E_1} + \int_{E_1\cup E_2} f\chi_{E_2}\\
                    \end{aligned}
                \end{equation*}
                One can check that for all measurable bounded finite supported $\varphi $ on $E_1 \cup E_2$, $\varphi \leq f$ on $E_1 
            \cup E_2$ if and only if $\varphi \chi_{E_1} \leq f \chi_{E_1}$ on $E_1$ and $\varphi \chi_{E_2} \leq f \chi_{E_2}$ on $E_2$. So
            \begin{equation*}
                \begin{aligned}
                    \int_{E_1\cup E_2} f\chi_{E_1} + \int_{E_1\cup E_2} f\chi_{E_2} &= \sup_{\varphi<f} \int_{E_1} \varphi \chi_{E_1} +\sup_{\varphi<f} \int_{E_2} \varphi \chi_{E_2}\\
                    &= \sup_{\varphi<f} \int_{E_1} \varphi +\sup_{\varphi<f} \int_{E_2} \varphi \\
                    &= \int_{E_1} f + \int_{E_2} f \\
                \end{aligned}
            \end{equation*}
            \end{proof}
            
        \section{The Lebesgue Integral in General}
            For measurable function $f$ in general, our mean to construct it's integral is to express $f$ as the difference of
        two non-negative functions. In the chapter of Lebesgue measure functions, we have a conclusion that maximum and minimumu of two 
        measurable functions is still measurable, which allow us to split $f$ into two part.
            \begin{definition}[Positive part and negative part]
                Let $f$ be a measurable function on $E$. We define it positive part $f^+$ and negative part $f^-$ as the following:
                \begin{equation*}
                    \begin{aligned}
                        f^+ = max(f, 0)\\
                        f^- = -min(f, 0) 
                    \end{aligned}
                \end{equation*} 
            \end{definition}
            It is not hard to find that $f^+$, $f^-$ are non-negative functions. And the relation 
            $$f = f^+ - f^-$$
        is also clear. But $\int f^+,  \ \int f^-$ might be infinity at the same time, then $\infty - \infty$ is not a well defined value. To fix this
        problem, we introduce the concept of integrable for non-negative measurable functions.
            \begin{definition}[Integrable]
                Let $f$ be a non-negative function on $E$, $f$ is integrable on $E$ if and only if $$\int_E f$$
            is a finite value.
            \end{definition}
            \begin{proposition}\label{IntegrableFiniteAE}
                A non-negative function $f$ is integrable over $E$, then it is finite a.e. on $E$.
            \end{proposition}
            \begin{proof}
                If $f$ is integrable, by Chebychev inequality,
                $$m(\{x \in E: f(x) \geq \infty\}) \leq m(\{x \in E: f(x) \geq \lambda\}) \frac{1}{\lambda} \cdot \int_E f$$
                Let $\lambda \to \infty$, we see that the measure of the subset where $f=\infty$ is zero.
            \end{proof}
            \begin{proposition}
                 For a measurable function $f$ on $E$, $f^+$ and $f^-$ are integrable on $E$ if and only if $|f|$ is integrable on $E$. 
            \end{proposition}
            \begin{proof}
                Suppose $|f|$ is integrable on $E$. Notice that $|f|= f^+ + f^-$, suppose $f^+$ is not integrable 
                \begin{equation*}
                    \begin{aligned}
                        +\infty &= \int_E f^+ \\
                        &\leq \int_E f^+ \int_E f^-\\
                        &= \int_E |f|
                    \end{aligned}
                \end{equation*} 
            Contradicting with the fact that $|f|$ is integrable, $f^+$ is integrable. One can show that $f^-$ is integrable in the exact
            same way. Then we proved one direction of the proposition.\par
                Suppose $f^+$ and $f^-$ are integrable, which means their integrals are finite. Then 
                \begin{equation*}
                    \begin{aligned}
                        \int_E |f| &= \int_E (f^+ +f^-)\\ 
                        &= \int_E f^+ + \int_E f^-
                    \end{aligned}
                \end{equation*}
            So $|f|$ is integrable.
            \end{proof}
        \begin{definition}[Lebesgue integral for measurable functions]
            Let $f$ be a measurable function on $E$. If $|f|$ is integrable on $E$, then we say $f$ is integrable, and define the integral of $f$ on $E$ as 
            \begin{equation}
                \int_E f = \int_E f^+ - \int_E f^-
            \end{equation}
        \end{definition}
        \begin{proposition}
            Let $f$ be integrable over $E$, then $f$ is finite a.e. on $E$, and 
            \begin{equation}\label{GLIMeasureZeroSet}
                \int_E f = \int_{E-E_0} f
            \end{equation} 
            where $E_0$ is a measure zero subset of $E$.
        \end{proposition}
        \begin{proof}
            If $f$ is not finite almost every where, then $|f| = \infty$ on $F \in E$ where $m(F) > 0$. 
        Then by the additivity over domain, and the monotonicity of integration,
        \begin{equation*}
            \begin{aligned}
                \int_E |f| &= \int_{E- F} f + \int_{F} f\\
                &\geq \int_{F} f\\
                &= \infty \cdot m(F)\\
                &= \infty
            \end{aligned}
        \end{equation*}
        which contradict with the fact that $f$ is integral.\par
        Now we prove \eqref{GLIMeasureZeroSet}. Notice that since the proposition \ref{The integral over measure zero set is zero}, being 
        the supremum of $\{0\}$, $\int_{E_0} f^+ = \int_{E_0} f^- = 0$.
        By the definition of the integral,
        \begin{equation*}
            \begin{aligned}
                \int_{E} f &= \int_{E-E_0} f^+ - \int_{E-E_0} f^-\\
                &= \int_{E-E_0} f^+ + \int_{E_0} f^+ - \int_{E-E_0} f^- - \int_{E_0} f^-\\
                &= \int_{E-E_0} f^+ - \int_{E-E_0} f^-\\ 
                &= \int_{E-E_0} f
            \end{aligned}
        \end{equation*}
        Thus the proposition is proved.
        \end{proof}
        Now we prove the familiar basic properties of integration in the most general extend. The function will be the general measurable functions, 
    which can take value at infinity. Although we can't even defined the sum of two measurable functions if they have too much value of 
    infinite, but if we ask them to be integrable, every thing turns out to be nice.
        \begin{theorem}[Linearity]
            Let $f , \ g$ be two measurable functions which are integrable over $E$, $\alpha$ and $\beta$ are real numbers. Then 
        $\alpha f+\beta g$ is integrable, and 
            $$\int_E \alpha f+\beta g = \int_E \alpha f + \int_E \beta g $$
        \end{theorem}
        \begin{proof}
            As usual, we first prove $\alpha f$ is integrable and $\int_E \alpha f = \alpha \int_E f$. Since $|\alpha f| = |\alpha| |f|$,
        by the linerarity of the inftegral of non-negative functions,
            \begin{equation*}
                \begin{aligned}
                    \int_E |\alpha f | = \int_E |\alpha| | f | = |\alpha| \int_E | f |\\ 
                \end{aligned}
            \end{equation*}
            Since both $ |\alpha|$ and $\int_E | f |$ are finite,  $\alpha f $ is integrable. If $\alpha \geq 0$,
            \begin{equation*}
                \begin{aligned}
                    \int_E \alpha f &= \int_E (\alpha f)^+ + \int_E (\alpha f)^- \\
                    &= \alpha \int_E f^+ +  \alpha \int_E f^-\\
                    &= \alpha \int_E f
                \end{aligned}
            \end{equation*}
            If $\alpha < 0$, 
            \begin{equation*}
                \begin{aligned}
                    \int_E \alpha f &= \int_E (\alpha f)^+ + \int_E (\alpha f)^- \\
                    &= \alpha \int_E f^+ +  \alpha \int_E f^-\\
                    &= \alpha \int_E f
                \end{aligned}
            \end{equation*}
            By the proposition above, $f$ and $g$ are finite a.e. on $E$, thus their sum is well defined. By triangle inequality of real 
        numbers, $|f+g| \leq |f|+|g|$. Then by the monotonicity of the integral of non-negative functions, one can see that $\int_E|f+g|$
        is finite, which indecated that $f+g$ is integrable. \par
            We have the following relationship:
            \begin{equation*}
                (f+g)^+ - (f+g)^- = f+g = (f^+ - f^-) + (g^+ -g^-)
            \end{equation*}
            Since the value on a measure zero set doesn't influence the integral, one can assuming that the above functions take finite 
        value on $E$, so 
            \begin{equation*}
                (f+g)^+ + f^- +g^-  = (f+g)^- + f^+  + g^+ \mbox{ on } E
            \end{equation*}
            From the linearity of non-negative functions(notice that both side of the equality above is non-negative), 
            \begin{equation*}
                \begin{aligned}
                    \int_E [(f+g)^+ + f^- +g^-]  &= \int_E [(f+g)^- + f^+  + g^+] \\
                    \int_E (f+g)^+ + \int_E f^- +\int_E g^-  &= \int_E (f+g)^- + \int_E f^+  + \int_E g^+\\
                \end{aligned}
            \end{equation*}
            Since $f$, $g$, $f+g$ are all integrable, the integral of non-negative function above are all finite. So it is allowd to 
        move some of them to the other side.
        \begin{equation*}
            \begin{aligned}
                \int_E (f+g)^+ - \int_E (f+g)^-  &= \int_E f^+ - \int_E f^- + \int_E g^+   - \int_E g^-\\
                \int_E f+g &= \int_E f + \int_E g 
            \end{aligned}
        \end{equation*} 
        \end{proof} 
        \begin{theorem}[Monotonicity]
            Let $f$ and $g$ be integrable over $E$. If $f\leq g$ on $E$, then 
            \begin{equation*}
                \int_E g \geq \int_E f
            \end{equation*}
        \end{theorem}
        \begin{proof}
            Since $g \geq f$, it is not hard to see $g^+ \geq f^+$, and $g^- \leq f^-$. By monotonicity of the integral of non-negative
        function, 
        $$\int_E g^+ \geq \int_E f^+, \ \ \ \int_E g^- \leq \int_E f^-$$
        which indicated 
        \begin{equation*}
            \begin{aligned}
                \int_E g^+ - \int_E g^- &\geq \int_E f^+ - \int_E f^-\\
                \int_E g &\geq \int_E f
            \end{aligned}
        \end{equation*} 
        \end{proof}
        \begin{theorem}[Additivity over domain]
            Let $f$ be a integrable function over $E$, and $E_1, \ E_2$ be measurable subset of $E$. Then $f$ is integrable over $E_1$ and 
        $E_2$, and we have 
        $$\int_{E_1 \cup E_2} f = \int_{E_1} f + \int_{E_2} f$$
        \end{theorem}
        \begin{proof}
            By the symmetry of the proposition, we only need to prove that $f$ is integrable $E_1$. By the additivity over domian of the 
        integral of non-negative functions, and the fact that the integral of any non-negative function is non-negative,
        \begin{equation*}
            \begin{aligned}
                \int_E |f| &= \int_{E-E_1} |f| +\int_{E_1} |f| \\
                &\geq \int_{E_1} f \\
            \end{aligned}
        \end{equation*}
        So it is clear that $f$ is integrable over $E_1$. \par 
            By the additivity over domain of non-negative functions,
        \begin{equation*}
            \begin{aligned}
                \int_{E_1\cup E_2} f &= \int_{E_1\cup E_2} f^+ - \int_{E_1\cup E_2} f^-\\
                &= \int_{E_1} f^+ + \int_{E_2} f^+ - \int_{E_1} f^- - \int_{E_2} f^-\\
                &= \int_{E_1} f + \int_{E_2} f
            \end{aligned}
        \end{equation*}
        \end{proof}
            Here we sum up the property of Lebesgue integral:
            \begin{equation*}
                \begin{aligned}
                    \mbox{Measure zero sets and integral: }& m(E_0) = 0 \Rightarrow \int_E f = \int_{E-E_0} f\\
                    \mbox{Linearity: }& \int_E f + \int_E g = \int_E (f+g)\\
                    \mbox{Monotonicity: }& f \leq g \Rightarrow \int_E f \leq \int_E g\\
                    \mbox{Additivity over domain: }& \int_{E_1} f + \int_{E_2} f = \int_{E_1 \cup E_2} f\\ 
                \end{aligned}
            \end{equation*}
            The method of constructing the Lebesgue integral is very inspiring. When dealing with a rather gengeral mathematic concept,
        one can first consider some special case, and extend the proposition or property to the more general case.  
    \section{Lebesgue vs Riemann}
        In this section we give some example, and discuss some difference between Lebesgue integral and Riemann integral. In the section of 
    the integral for bounded measurable function, we have show that any Riemann integrable function is Lebesgue integrable, but didn't show 
    the converse is false. Now we give an counter-example.
        \begin{example}
            Let $f = \chi_{\RR -\QQ} \cap [0,1]$(the Dirichlet function). No matter how fine a Riemann partition is, the small interval
        determined by the partition will contains a rational and a irrational, which means the upper Riemann integral will be $1$ and 
        the lower will be $0$. So $f$ is not Riemann integrable. \par
            But with Lebesgue integral, $f$ is just a simple function, so is integrable and it's integral is $1$.
        \end{example}

        \begin{theorem}[The Lebesgue condition for Riemann integral funciton]
            
        \end{theorem}
        By the upper theorem, a Riemann integrable function on a closed bounded interval is continuous a.e. on it's domain; hbounded 
    functions on intervals is Lebesgue integrable if and only if it is measurable. By Lusin's theorem, given any $\epsilon>0$, it is 
    continouos on it's domain but a set whose measure is less than $\epsilon$. 

\chapter{Limit and Itergral}
        In the first section, we promised that Lebesgue integral will has better properties compare to Riemann integral. Now we have 
    seen that Lebesgue integral have defined on more functions and sets. In this chapter, we will focus on the issue 
    that under what condition, the following equation will hold: 
    \begin{equation*}
        \lim_{n \to \infty} \int_E f_n(x) = \int_E f(x) 
    \end{equation*}
    where $E$ is a measurable set, and $\{f_n\} \subset $\lmf{E} which converge to $f$.  
    \section{From Bounded Convergence Theorem to Lebesgue Dominated Convergence Theorem}    
        \begin{theorem}[Bounded convergence theorem]
            Let $\{f_n\}$ be a sequence of measurable function on $E$, where $m(E)<+\infty$. Suppose $f_n$ is uniform bounded on $E$. 
        If $\{f_n\} \to f$ pointwisely a.e. on $E$, then 
        $$\lim_{n \to \infty} \int_E f_n(x) = \int_E f(x) $$
        \end{theorem}\par
            Although we need to show that $f$ is integrable, this part is rather trivial, so is it left to the reader to check.
            Our basic idea of the prove is by using Egoroff's theorem, to show that $\{f_n\}$ coverge uniformly on a very large subset.
        Then for the part on that subset, the integral and the limit sign could commute. Out of the subset, the difference of integral 
        will be small enough.
        \begin{proof}
            Since our result is the equation between integrals, we can assume $\{f_n\}$ converge pointwisely on $E$. Assuming $|f_n| < M$ 
        for all possible $n$. Since $m(E)< +\infty$, Egoroff's theorem is available. Thus given $\epsilon >0 $, 
        we can find a close subset of $E$, say $F$, such that $m(E - F) < \frac{\epsilon}{ 4M \cdot m(E) }$  and $f_n$ converge uniform to 
        $f$. Also because the uniform convergence, for the same $\epsilon$ we can find a $N$ such that for $n\geq N$, 
        $$|\int_F f_n -\int_F f| \leq \frac{\epsilon}{2}$$
        Together with the above estimation and properties of integral,
            \begin{equation*}
                \begin{aligned}
                    |\int_E f_n - f| &= |\int_{E-F} f_n - \int_{E-F} f + \int_{F} f_n -\int_{F} f|\\
                    &\leq |\int_{E-F} f_n - \int_{E-F} f| + |\int_{F} f_n -\int_{F} f|\\
                    &\leq \int_{E-F} 2M  + \frac{\epsilon}{2}\\
                    &\leq 2M \cdot m(E-F) +\frac{\epsilon}{2}\\
                    &\leq 2M \cdot \frac{\epsilon}{ 4M \cdot m(E) } +\frac{\epsilon}{2}\\
                    &= \epsilon
                \end{aligned}
            \end{equation*}
        Thus we show that the limit sign could go though the integral sign.
        \end{proof}
            Uniform bounded is weaker than uniform converge, but we can prove much more strong converge theorem under the context 
        of Lebesgue integral. To prove the stronger converge theorems, it is conveneint to establish the following lemma.
        \begin{lemma}[Fatou's lemma]
            Let $\{f_n\}$ be a sequence of non-negative measurable functions on $E$. If $\{f_n\}$ converge pointwisely a.e. over $E$.
        then 
        $$\int_E f \leq \lim \inf \int_E f_n$$
        \end{lemma}
        Remember that $\lim \inf $ of a sequence $\{a_n\}$ is defined as $\sup \{\inf \{a_n\}_{n=k}^{\infty} : 1\leq i < \infty \}$.
        \begin{proof}
            First, measure zero sets doesn't influence the integral, thus one can assume $\{f_n\}$ converge pointwise on $E$.
        Since the integral of non-negative functions is defined as the supremum of the integral of bounded measurable function of finite 
        support, we only need to show that for all such function $\varphi \leq f$, 
        $$\int_E \varphi \leq \lim \inf \int_E f_n$$\par
            Let $\varphi_n = min(\varphi, f_n)$, then $\{\varphi_n\}$ coverge to $\varphi$ pointwisely. There exist a set of finite measure 
        $E_0$ on which $\varphi$ and $\varphi_n$ support on. So by bounded converge theorem,
        \begin{equation*}
            \begin{aligned}
                \int_E \varphi = \int_{E_0} \lim_{n \to \infty} h_n = \lim_{n \to \infty} \int_{E_0} h_n
            \end{aligned}
        \end{equation*}
        By the basic properties of limit and integral, and the fact that $h_n \leq f_n$we have 
        \begin{equation*}
            \begin{aligned}
                \int_E \varphi &= \lim_{n \to \infty} \int_E h_n\\
                &\leq \lim \inf \int_E f_n
            \end{aligned}
        \end{equation*}
        \end{proof}
            The prove above illustrate that if you want to prove a proposition of non-egative function, you can first prove it for bounded 
        measurable function of finite support. Also if yoou want to prove something for measurable function, it is always helpful to prove it 
        for simple functions. 
        \begin{theorem}[Monotone convergence theorem]
            Let $\{f_n\}$ be a sequence of nonegative measurable functions on $E$. If $f_m \leq f_n$ for $m<n$, and $\{f_n\} \to f$ pointwisely a.e. on
        $E$, then 
        $$\lim_{n \to \infty} \int_E f_n(x) = \int_E f(x) $$
        \end{theorem}
        \begin{proof}
            We assume $\{f_n\}$ converge to $f$ pointwisely on $E$. Since the sequence is increasing, by monotonicity, if $m<n$, we have
            $$\int_E f_m \leq \int_E f_n$$
            Thus by Fatou's lemma 
            $$\int_E f \leq \lim \inf \int_E f_n = \lim_{n\to \infty}\int_E f_n$$
        The last equality comes from the fact that $\{\int_E f_n\}$ is a increasing sequence. It remains to show the inequality from the other
        side. Since $\{f_n\}$ converge to $f$ increasingly,$f> f_n$ for all $n$. Thus by the monotonicity of the integration,
            $$\int_E f_n \leq \int_E f$$
            $$\Rightarrow \lim_{n \to \infty} \int_E f_n \leq \int_E f$$
        \end{proof}
        Notice that the existence of the limit of the integration of the sequence of function is guaranteened because the general limit exist 
        for monotonic sequences.
        \begin{corollary}
            Let $\{f_n\}$ be non-negative measurable function on $E$. If $f = \sum_{n=1}^{\infty} f_n$ converge pointwisely a.e. on $E$,
            $$\int_E f = \sum_{n=1}^{\infty} \int_E f_n$$
        \end{corollary}
        \begin{proof}
            Notice that the partial sum is coverge increasingly(since $f_n$ are non-negative), so we can apply the monotone converge theorem.
        Let $F_k = \sum_{n=1}^{k} \int_E f_n$.
            \begin{equation*}
                \begin{aligned}
                    \int_E f &= \lim_{n \to \infty} \int_E F_k\\
                    &= \lim_{n \to \infty} \int_E \sum_{n=1}^{k} \int_E f_n\\
                    &= \sum_{n=1}^{\infty} \int_E f_n
                \end{aligned}
            \end{equation*}
        \end{proof}
        \begin{lemma}[Beppo Levi's Lemma]
            Let $\{f_n\}$ be a sequence of non-negative measurable function on $E$. If $\{\int_E f_n\}$ is bounded, then $\{f_n\}$ converge 
        pointwisely to a $f$ on $E$, which is finite a.e on $E$, and 
        $$\lim{n \to \infty} \int_E f_n = \int_E f < \infty$$ 
        \end{lemma}
        \begin{proof}
            Since every monotone sequence of real number converge to a extended real number, so for every $x\in E$, we can define 
            $$f(x) = \lim_{n\to \infty} f_n (x)$$
        Then it is obvious that $\{f_n\}$ converge to $f$ pointwisely, and increasingly. Thus we can use monotone converge theorem, which shows
            $$\lim_{n\to \infty} \int_E f_n = \int_E f$$
            Since $\{\int_E f_n\}$ is bounded, $\int_E f$ is a finite value. So $f$ is integrable, which indicated $f$ is finite a.e. on $E$
            (proposition \ref{IntegrableFiniteAE}).
        \end{proof}
        \begin{theorem}[Legesgue dominated convergence theorem]
            Let $\{f_n\}$ be a sequence of measurable functions on $E$, which converge pointwisely a.e. to $f$ on $E$. If $f_n$ is dominated 
        by a integrable function $g$, in the sense of $|f_n|\leq g$, then
            $$\lim_{n \to \infty} \int_E f_n = \int_E f$$
        \end{theorem}
        \begin{proof}
            Again, the \textsl{a.e.} can be ignored. Notice that $f$ is also dominated by $g$. By monotonicity of the integral,
            $$\int_E |f| \leq \int_E g < \infty$$
        $f$ is integrable.
            $g-f_n$ and $g-f$ are all non-negative functions, and $(g-f_n) \to (g-f)$ pointwisely as $n \to \infty$. Thus we can apply 
        Fatou's lemma.
            \begin{equation*}
                \begin{aligned}
                    \int_E g-f &\leq \lim \inf \int_E g -f_n \\            
                    &= \int_E g - \lim \sup \int_E f_n\\
                    \Rightarrow \int_E g- \int_E f &\leq \int_E g - \lim \sup \int_E f_n\\
                    \Rightarrow \int_E f &\geq \lim \sup \int_E f_n\\
                \end{aligned}
            \end{equation*}
        Observing that $g+ f_n$ and $g+f$ and also non-negative, and $(g+f_n) \to (g+f)$ pointwisely as $n \to \infty$,
        using the same trick,
            \begin{equation*}
                \begin{aligned}
                    \int_E g+f &\leq \lim \inf \int_E g +f_n \\            
                    &= \int_E g + \lim inf \int_E f_n\\
                    \Rightarrow \int_E g+ \int_E f &\leq \int_E g + \lim \inf \int_E f_n\\
                    \Rightarrow \int_E f &\leq \lim \inf \int_E f_n\\
                \end{aligned}
            \end{equation*}
        Together the two inequality, we have 
            $$\int_E f \leq \lim \inf \int_E f_n \leq \lim \sup \int_E f_n \leq \int_E f$$
        Thus $\lim_{n\to \infty} f_n$ exist and 
        $$\int_E f = \lim_{n\to \infty} f_n$$ 
        \end{proof}
        The dominated converge theorem has a more general version, the proof is simialr. 
        \begin{theorem}[General dominated convergence theorem]
            Let $\{f_n\}$ be a sequence of measurable functions on $E$, that converge pointwisely a.e. on $E$ to $f$. Suppose there is a 
        sequence $\{g_n\}$ of non-negative functions on $E$, that converge to $g$ pointwisely a.e. on $E$. If $|f_n|<g_n$ on $E$ for all $n$,
        and $$\lim_{n\to \infty} \int_E g_n = \int_E g < \infty$$
        then we have 
        $$\lim_{n \to \infty} \int_E f_n = \int_E f$$
        \end{theorem}
        \begin{proof}
            Assume $\{f_n\}$ and $\{g_n\}$ converge to $f$ and $g$ on $E$. By monotonicity of limit, $g \geq |f|$, which shows $f$ is integrable
        The sequence $\{g_n - f_n\}$ is non-negative and converge to  $g-f$. So by Fatou's lemma,
        \begin{equation*}
            \begin{aligned}
                \int_E g-f &\leq \lim \inf \int_E g_n -f_n\\
                \int_E g - \int_E f &\leq \lim \inf \int_E g_n - \lim \sup f_n \\ 
                \int_E f &\leq \lim \sup f_n
            \end{aligned}
        \end{equation*}
        On the other hand, $\{g_n + f_n\}$ is non-negative and converge to $g+f$ pointwisely on $E$. We have 
        \begin{equation*}
            \begin{aligned}
                \int_E g+f &\leq \lim \inf \int_E g_n +f_n\\
                \int_E g + \int_E f &\leq \lim \inf \int_E g_n + \lim \inf f_n \\ 
                \int_E f &\leq \lim \inf f_n
            \end{aligned}
        \end{equation*}
        So, together the two inequality, theorem is proved.
        \end{proof}
        Now we using the following theorem to show how to use the dominated converge theorem, while the result of the theorem is also 
    important.
        \begin{theorem}[Countable additivity over domian]
            Let $f$ be a integrable function on $E$ and $\{E_n\}$ be a disjoint countable collection of measurable subsets of $E$, whose union is 
        $E$. Then 
        $$\int_E f = \sum_{n=1}^{\infty} \int_{E_n}f$$
        \end{theorem}
        \begin{proof}
            Let $f_n = f \chi_{\cup_{i=1}^{n}E_i} = \sum_{i=1}^{n} f \chi_{E_i}$(the second equality comes from the property of 
        characteristic functions that we have already proved). It is not hard to see that $\{f_n\}$ converge to $f$ pointwisely on $E$.
        Notice that all the function here is dominated by the integrable function $f$, thus the Lebesgue dominated converge theorem is 
        available:
            \begin{equation*}
                \begin{aligned}
                    \int_E f &= \lim_{n \to \infty} \int_E f_n \\
                    &= \lim_{n \to \infty} \int_E f \chi_{\cup_{i=1}^{n}E_i}\\
                    &= \lim_{n \to \infty} \int_E \sum_{i=1}^{n} f \chi_{E_i}\\
                    &= \lim_{n \to \infty} \sum_{i=1}^{n} \int_E f \chi_{E_i}\\
                    &= \lim_{n \to \infty} \sum_{i=1}^{n} \int_{E_i} f\\
                    &= \sum_{i=1}^{\infty} \int_{E_i} f
                \end{aligned}
            \end{equation*}
        \end{proof}
    
    \section{Uniform Integrability: The Vitali Convergence theorem}
            In this chapter, we will get into the concept of uniform integrability, and a new passage under integral sign. Here we first 
        prove some useful lemma.
        \begin{lemma}\label{UniformIntegrableLemma}
            Let $E$ be a set of finite measure and $\delta > 0$, then we can write $E$ as a disjoint union of finite collection of measurable
        set, each with measure less that $\delta$.
        \end{lemma}
        \begin{proof}
            By continuity of the measure, we have 
            $$\lim_{n \to \infty} m(E - [-n, n]) = 0$$
        since as $n \to \infty$, $E-[-n,n] \to \varnothing$. So there exist a $N$ such that $m(E-[-N,N]) < {\delta}$. Since 
        $E-[-N,N]$ is a bounded set, we can cover it with finite many set whose measure is less than $\delta$.
        Thus we can cover $E$ with finite many disjoint sets, each of 
        measure less than $\delta$.
        \end{proof}
        \begin{proposition}\label{EstimationOfIntOverSmallSet}
            Given any integrable function $f$ on $E$. For an $\epsilon >0$,
        There exists a $\delta$ such that for all measurable $E' \subset E$,
        \begin{equation}\label{EstimationOfIntOverSmallSet1}
            \int_{E'} |f| < \epsilon
        \end{equation}
        Conversely, in the case $m(E)< \infty$, if for any $\epsilon>0$,
        there exist a $\delta$ such that \eqref{EstimationOfIntOverSmallSet1}
        holds, then $f$ is integrable.
        \end{proposition}
        \begin{proof}
            The proposition is proved by separate $f$ in to positive
        and negative part. We first suppose $f$ is nonegative over
        $E$. Assuming $f$ is integrable, indicating $\int_E f \leq
        \infty$. By the definition of the integral of non-negative 
        functions, we can find a sequence of bounded measurable Functions
        of finite support, say $\{\varphi_n\}$, such that
        $0 \leq \varphi_ \leq f$ and 
        \begin{equation*}
            \int_E f - \int \varphi_n < \frac{\epsilon}{2}
        \end{equation*}
            Since $f -\varphi_n>0$ on $E$, if $E'$ is measurable,
        using the properties of integral for non-negative functions,
        \begin{equation*}
            \begin{aligned}
                \int_{E'} f - \int \varphi_n &= \int_{E'} f -\varphi_n\\
            &\leq \int_{E} f -\varphi_n\\
            &\leq \frac{\epsilon}{2}\\
            \end{aligned}
        \end{equation*}
        for $n \geq N$, where $N$ is some integer. Notice that $\varphi_N$ is
        bounded by some $M$. Consequently, we have
            \begin{equation*}
                \int_{E'} f \leq \int_{E'} \varphi_N + \frac{\epsilon}{2} 
                \leq M\cdot m(E') +\frac{\epsilon}{2}
            \end{equation*}
            Let $m(E')\leq \delta = \frac{\epsilon}{2M}$, we have 
            $$\int_{E'} f \leq \epsilon$$
            Then we prove the case when $f$ is non-negative. The case 
        where $f$ is general integrable function follows the fact that $|f| = f^+ +f^-$
        while $f^+$ and $f^-$ are both non-negative integrable function.\par
            Suppose $E$ is finite, and for every $\epsilon >0$, there exists a 
            $\delta$ such that for $m(E')< \delta$, 
            \eqref{EstimationOfIntOverSmallSet1} is true.
            Then by the previous lemma, $E$ can be expressed as the union of finite
        collection of subsets $\{E_n\}$, where $m(E_n)<\delta$.
            So letting, $\epsilon =1$
            \begin{equation*}
                \begin{aligned}
                    \int_E |f| &= \int_{\cup_{n=1}^{N} E_n} |f|\\ 
                    &= \sum_{n=1}^{N} \int_{E_n} |f| \\
                    &< \sum_{n=1}^{N} 1 = N
                \end{aligned}
            \end{equation*}
            Therefore $f$ is integrable. 
        \end{proof}
        \begin{definition}[Uniform integrable]
            Let $\mathcal{F}$ be a collection of measurable functions on $E$,
        a measurable set.
        We say $\mathcal{F}$ is uniform integrable if and only if for all
        $\epsilon >0$, there exists a $\delta$, such that for all $E'\subset E$, 
        while $m(E') \leq \delta$, we have
        \begin{equation}
            \int_{E'} |f| < \epsilon 
        \end{equation}
        where $f$ is any function in $\mathcal{F}$.
        \end{definition}
        To get familiar with this concept, we establish the following
        proposition.
        \begin{proposition}
            Let $\mathcal{F}$ be a finite collection of integrable functions over $E$.
        Then $\mathcal{F}$ is uniformly intergrable.
        \end{proposition}
        \begin{proof}
            The proof is not hard. By lemma \ref{UniformIntegrableLemma},
        Given $\epsilon>0$, 
        we can find a $\delta_n$ for $f_n$ such that for any $E'\subset E$,
        which satisfy $m(E') < \delta_n$,
        $$\int_{E'} |f_n| \leq \epsilon $$
        Let $\delta = min(\delta_n)$, then the theorem is proved.
        \end{proof}
        \begin{proposition}\label{LimitFunctionOfUniformIntegrable}
            Assume $E$ has finite measure, and $\{f_n\}$ be a collection of 
        function which is uniformly integrable on $E$. If $f_n \to f$ 
        pointwisely a.e. on $E$, $f$ is integrable.
        \end{proposition}
        \begin{proof}
            Assuming $f_n \to f$ on $E$, by the condition of uniform integrable, 
        $\forall \epsilon > 0$, $\exists \delta$ such that $\forall E' \subset E$,
        whose measure is less than $\delta$, $\int_{E'} |f_n| < \epsilon$.
            Since $m(E)$ is finite, we can find a finte disjoint collection of 
        measurable sets $\{E_n\}$, with each $m(E)< \delta$whose 
        union is $E$. So by the finite additivity over domain,
        \begin{equation*}
            \begin{aligned}
                \int_E |f_n| &= \int_{\cup_{n=1}^{N}E_n} |f_n|\\
                &= \sum_{n=1}^N \int_{E_n } |f_n|\\
                &\leq N \epsilon 
            \end{aligned}
        \end{equation*}
        Letting $\epsilon = 1$, then for each $n$, $|\int_E |f_n|<N$ , thus
        by Fatou's Lemma, 
        \begin{equation*}
            \begin{aligned}
                \int_E |f| &\leq \lim \inf \int_E |f_n|\\
                &\leq N
            \end{aligned}
        \end{equation*}
        we prove the proposition. 
        \end{proof}
            Now using the uniform integrability, we establish a new theorem of passage under integral sign.
        \begin{theorem}[Vetali's convergence theorem]
            Let $E$ be of finite measure, $\{f_n\}$ be a sequence of function,
        which is uniformly integrable. If $f_n \to f$ pointwisely a.e. on 
        $E$, 
        $$\lim_{n \to \infty} \int_E f_n = \int_E f$$ 
        \end{theorem}
        \begin{proof}
            By proposition \ref{LimitFunctionOfUniformIntegrable}, 
        $f$ is integrable. One can easily figure out that 
        $\mathcal{F} = \{f_n\}\cup \{f\}$ is still uniformly integrable.
        Using the condition that $\mathcal{F}$ is uniformly integrable,
        given $\epsilon$, $\exists \delta$ such that for $F \subset E 
        , \ m(F)< \delta$, 
        \begin{equation}\label{VitaliConvergenceTheorem1}
            \int_F f_n < \frac{\epsilon}{4} , \ \ \ \int_F f < \frac{\epsilon}{4}
        \end{equation}
        Notice that $m(E)< \infty$ means we can use Egoroff's
        theorem. The same $\delta >0$, we can find a closed $F \subset E$,
        on which $|f_n|$ converge to $|f|$ uniformly, and $m(E-F) <\delta$.
        For the smae $\epsilon$, there exists $N$, such
        that for $n\geq N$, 
        \begin{equation}\label{VitaliConvergenceTheorem2}
            \int_F |f_n - f| \leq \int_F |f_n| - \int_F |f| < \frac{\epsilon}{2}
        \end{equation}
        Thus by \eqref{VitaliConvergenceTheorem1} and \eqref{VitaliConvergenceTheorem2},
        \begin{equation*}
            \begin{aligned}
                |\inf_E f_n -\int_E f| &\leq \inf_E |f_n - f|\\
                &= \inf_F | f_n -f| + \inf_{E-F} | f_n -f|\\
                &< \frac{\epsilon}{2} + \inf_{E-F}|f_n| +\inf_{E-F}|f|\\
                &< \frac{\epsilon}{2} +\frac{\epsilon}{4}+\frac{\epsilon}{4}\\
                &= \epsilon
            \end{aligned}
        \end{equation*}
        Let $\epsilon \to 0$, we prove the theorem.
        \end{proof}
        \begin{theorem}[General Vetali's converge theorem]
            
        \end{theorem}
        
\chapter{Differentiation and Integration}
    In Riemann integral , we have proved a very powerful theorem, \textbf{the fundamental theorem of calculus}. This theorem, in calculus
textbook, is only apply to differentiable functions, Which is a very small collection of functions, comparing with continuous functions
(let alone measurable functions). In this chapter, we will focus on the question that under what condition, or to say with what kinds of 
functions, does 
$$\int_a^b f = f(b) - f(a)$$
hold. It turns out that monotonic functions has very good properties(better than continuous function), including it is differentiable a.e. over
close bounded, and we have FTC with monotonic function. Further more, there is a large kinds of function can be express as the difference 
of monotonic functions(which we say they are \textbf{absolute continuous}). And FTC does hold for them.
    \section{Monotonic functions and Lebesgue's Theorem }
        Remember, a function is differentiable at $x_0$ indicated that it is continuous at $x_0$. We first show that monotonic functions
    are continuous a.e. on interior of their domain.
        \begin{theorem}[Monotonic functions are almost continuous function]\label{MonotonicFunctionIsContinuous}
            Let $f$ be a monotonic function on a open interval $(a,b)$. Then the discontinuity of $f$ is countable. 
        \end{theorem}
        \begin{proof}
            Suppose $f$ is increasing(decreasing functions can be proved in the exact same way), and $(a,b)$ is bounded.
        Let $D$ be the subset of points at which $f$ is discontinuous. $\forall x_0 \in D$, since the function is monotonic, 
        $f(a)\leq f(x_0) \leq f(b)$. There exist a $(c,d) \subset (a,b)$ such that $c\neq d$ 
        \begin{equation*}
            \begin{aligned}
                \lim_{x \to x_0^-} f(x) = c\\
                \lim_{x \to x_0^+} f(x) = d\\
            \end{aligned}
        \end{equation*}
        We can see for all $x> x_0$, $f(x) \geq d$; for all $x < x_0$, $f(x) \leq c$. So for any two $x_1, x_2 \in D$, 
        the corresponding $(c_1, d_1), (c_2, d_2)$ are disjoint.
        By the density of rationals in real number, for each $x_0 \in D$, we can find a $q \in (c,d)$. Thus there is bijection from
        $D$ to a subset of $\QQ$, which shows that $D$ is countable. 
            For a interval $I$ which is not bounded, we can divide it into countable union of bounded intervals $\bigcup_{n=1}^{infty} I_n$. 
        By the previous argument, on each $I_n$, $f$ has countable discontinuities $D_n$. The discontinuities of $f$ on $I$ will be a
        subset of $$(\bigcup_{n=i}^{\infty} U_n)\cup (\bigcup_{n=1}^{\infty} {a_n, b_n})$$
        where $I_n = (a_n, b_n)$. So the proposition still holds for the case of un-bounded open interval.
        \end{proof}
        \begin{proposition}
            Let $D$ be a countable subset on a open interval $(a,b)$. Then there is an increasing function $f$ which is continuous only
        at $(a,b) - D$.
        \end{proposition}
        \begin{proof}
            Let ${d_n}$ be a enumeration of $D$. We defined $f:(a,b)\to \RR$ by 
            $$f(x) = \sum_{\{n: q_n \leq x\}} \frac{1}{2^n}$$ 
        Since the positive series $\sum_{n=1}^{\infty}\frac{1}{2^n}$ is converge, so $f(x)$ is well define for all points in $(a,b)$ and 
        f is increasing. Let $x'=q_k$, then for all $x <x'$ 
        $$f(x') - f(x) = \frac{1}{2^k}$$
        Thus $f$ is discontinuous on $D$.\par
            Let $x' \in (a,b) - D$, For all natural number $n$, there eixst a interval $I_n$ such that $x' \in I_n \subset (a,b)$, but
        $q_i \notin I_n$ for $i \leq n$. So for $x \in I_n$, $f(x) - f(x') \leq \frac{1}{2^n}$. Let $n \to \infty$, we conclude that $f$ 
        is continuous at $x$.
        \end{proof}
        With the notion of continuity of generalized real valued functions, We can prove a stronger version of the theorem \ref{MonotonicFunctionIsContinuous}.
        \begin{theorem}
            Given any subset $E$ of $R$, if $f:E \to \RR$ is a monotonic function, then $f$ has only countable many discontinuities.
        \end{theorem}
        \begin{proof}
            Since any function is continuous on isolated point, so we can assume $E$ has no isolated point. We can always find a open 
        interval $(\inf E+1, \sup E+1) \supset E$(with $\infty +1 = \infty$, this interval is well defined). Consider the extension of 
        $f$ on $(a, b)$ determined by the following condition:
        \begin{enumerate}
            \item For $x \in (\inf E +1, \inf E]$, let $f(x) = f(\inf E)$; 
            \item For $x \in [\sup E, \sup E +1)$, let $f(x) = f(\sup E)$; 
            \item For $x \in(\inf E, \sup E) - E$, let $f(x') = \sup_{x \in E; x<x'} f(x)$. 
        \end{enumerate}
            We denote the extension as $f'$. By construction, the ristirction of $f'$ on $E$ is $f$.  \par
            Since the monotonic function on interval has only countable many discontinuities, and every discontinuity of a ristriction of 
        a function is a discontinuity of the original function, there exist a injection from the discontinuities of $f$ to that of $f'$.
        The above consideration indicated that, to prove the proposition, we only need to show $f'$ is monotonic.\par
            If
            \begin{enumerate} 
                \item  $x' \in (\inf E +1, \inf E]$, and $x<x'$, $f'(x)= \inf f(x) \leq \inf f(x) =f'(x')$; 
                \item  $x' \in [\sup E, \sup E +1)$, and $x<x'$, $f'(x') = \sup f(x) \geq f'(x)$; 
                \item  $x' \in(\inf E, \sup E) - E$, and $x<x'$, 
                \begin{enumerate}
                    \item if $x\in E$, $f'(x') = \sup_{x \in E; x<x'} f(x) \geq f(x) = f'(x)$;
                    \item if $x \notin E$, $f'(x') = \sup_{y \in E ;y<x'} f(y) \geq \sup_{y \in E ;y<x} f(y) = f'(x)$. 
                \end{enumerate}
                \item  $x' \in E$, and $x<x'$, 
                \begin{enumerate}
                    \item if $x \leq \inf E$, $f'(x) \leq f'(x')$ is trivial.
                    \item if $x \in E$, then $f'(x) = f(x) \leq f(x') = f'(x')$.
                    \item if $x \in (\inf E, \sup E) - E$, $f'(x') = f(x') \geq \sup_{y \in E; y<x} f = f'(x)$
                \end{enumerate}
            \end{enumerate}
            Then the proposition is proved.
        \end{proof}

        \begin{definition}[Vitali cover]
            A collection $\mathcal{F}$ of closed, bounded, non-degenerate intervals is said to be cover a set $E$ in the sense of Vitali
        if for each point $x \in E$, and $\epsilon > 0$, there is an interval $I$ in $\mathcal{F}$ that contains $x$ such that 
        $l(I) < \epsilon$.
        \end{definition}
        This concept and the following lemma is needed for the proof of Lebesgue theorem. 
        \begin{lemma}[Vitali cover lemma]
            Let $E$ be a set of finite outer measure and $\mathcal(F)$ a collection of closed bounded which is a Vitali cover of $E$.
        Then for each $\epsilon > 0$, there is a finite disjoint sub-collection $\{I_n\}_{n=1}^N$ such that 
        $$m*(E - \cup_{n=1}^{N} I_n) \leq \epsilon$$
        \end{lemma}
        \begin{proof}
            Since $E$ has a finite outer measure, so there exist a open set $O \supset E$ such that $m*(E) > m*(O) -\epsilon$. We can legally
        ask every element of the Vitali cover of $E$, namely $F \in \mathcal{F}$, to be subset of $O$. Then by the countable additivity and monotonicity 
        of the measure, we have
        $$\sum_{n=1}^{\infty} l(I_n) \leq m*(O) < +\infty$$
        where $\{I_n\}$ is any disjoint sub-collection. \par
            Since each $I_n$ is closed and for every $x \in E$, we can find a small enough interval cover $x$, take any finite 
        sub-collection $\{I_n\}_{i=1}^{N}$, we have  
        $$E - \bigcup_{i=1}^{N} I_n \subset \bigcup_{I \in \mathcal{F}_n} I$$
        where 
        \begin{equation}\label{FnInVitaliLemma}
            \mathcal{F}_n = {I \in \mathcal{F}: I \cap \bigcup_{i=1}^{N} I_n = \varnothing}.
        \end{equation}\par
            \begin{remark}
                The relation above holds is because the finite set which we have take out from $E$ is a close set, thus $O - \bigcup_{i=1}^{N}$ 
        is open in $\RR$, which means any point is an interior point. 
            \end{remark}
            Suppose there is a $\{I_n\}_{n=1}^{N} \subset \mathcal{F}$, such that $\bigcup_{i=1}^{N} I_n \supset E$,
        then we are done. Suppose for every ${I_n}_{n=1}^{N} \subset \mathcal{F}$, $ \bigcup_{i=1}^{N} I_n$ doesn't covers $E$. We inductively
        construct the collection of disjoint close interval $\{I_k\}_{k=1}^{\infty}$ which has the following property:
        \begin{equation}\label{SubcollectionInVitaliLemma}
            E- \cup_{k=1}^{n} I_k \subset \bigcup_{k=n+1}^{\infty} 5*I_k
        \end{equation}
        for $n$ is any natural number, where $5*I_k$ is the interval which has the same mid-point as $I_k$ while $l(5*I_k) = 5l(I_k)$.\par
            Now we begin the selection. Let $I_1$ be any interval in $\mathcal{F}$. Suppose for $k <n$, $I_k$ has already been choosen. 
        Since $E- \bigcup_{k=1}^{n-1} I_k$ is not empty, the collection $\mathcal{F}_{n-1}$ defined as \eqref{FnInVitaliLemma} is not empty.
        We define $s_n = \sup_{I \in \mathcal{F}_n} l(I)$. Choose $I_n$ to be a element in $\mathcal{F}_{n-1}$, such that $l(I)> \frac{s_{n-1}}{2}$.\par
            By the selection above, clearly $\{I_k\}_{k=1}^{\infty}$ is disjoint. It remains to show the property \eqref{SubcollectionInVitaliLemma}.
        We infer from the fact so that $\sum_{k=1}^{\infty} l(I_k)$ is finite that $\lim_{k \to \infty}l(I_k) = 0$.
        Given $n>0$, $\forall x \in E- \cup_{k=1}^{n} I_k$, $x \in \bigcup_{I \in \mathcal{F}_n} I$. Which means $x \in I \in \mathcal{F}_n$
        for some $I$. Then $I$ intersect with some $I_K$ because if not, $l(I_k) > \frac{I}{2}$ for all $I$, which contradict with the 
        fact that $l(I_k) \to 0$. Then it is clear that $x \in 5*I_K$,
        since $l(I) \leq  s_K \leq 2 l(I_K)$, and the fact that $I_K \cap I \neq \varnothing$. \par
            Thus 
            $$E- \cup_{k=1}^{n} \subset \bigcup_{k=n+1}^{\infty} 5*I_k$$
            is varified. Given any $\epsilon>0$, let $N$ be the number such that 
            $$\sum_{N+1}^{\infty} l(I_k) <\frac{\epsilon}{5}$$
            Together with \eqref{SubcollectionInVitaliLemma} and the monotonicity, additivity of the measure, we prove the lemma.
        \end{proof}
        \begin{definition}[Dini derivative]
            Let $f$ be a real valued function on $E$, and $x \in E$ is a interior point. We define the upper and lower derivative as 
        the following:
        \begin{equation*}
            \begin{aligned}
                \overline{D}f(x) &= \lim_{h \to 0} \left[ \sup_{0<|t|\leq h} \frac{f(x+t)-f(x)}{t}\right]\\
                \underline{D}f(x) &= \lim_{h \to 0} \left[ \inf_{0<|t|\leq h} \frac{f(x+t)-f(x)}{t}\right]\\
            \end{aligned}
        \end{equation*}
        We have $\overline{D}f(x) \geq \underline{D}f(x)$. If the upper derivative and the lower derivative are equal, we say $f$ is 
        \textbf{differentiable} at $x$, and denote the common value as $f'(x)$. 
        \end{definition}
        One can easily check that the definition of differentiable above is compatible with the one that is familiar to us. It provides a  
    method to distinguish the differentiable points from the non-differentiable points. You will see how does it work in the proof of the
    following lemma and Lebesgue theorem.\par 
        Recall The Mean Value Theorem in calculus, given a continuous function on close bounded interval $[c,d]$, which is differentiable
    on $(a,b)$, we can conclude if $f'> c$ on $(a,b)$, $$c(b-a) \leq f(b) - f(a)$$
        The following lemma is a generalization of the upper inequality for monotonic functions. It gives an estimation of the area 
    of the upper derivative is bigger than a certain number, which is needed in the proof of Lebesgue theorem. 
        \begin{lemma}
            Let $f$ be an increasing function on the closed bounded interval $[a,b]$. Then for each $c>0$, 
            $$m^*\{x \in (a,b) : \overline{D}f(x) \geq c\} \leq \frac{1}{c} [f(b) -f(a)]$$
            and 
            \begin{equation}\label{LemmaUpperDerivativeMeasureEstimationInfty}
                m^*\{x \in (a,b): \overline{D}f(x) = \infty \} = 0                
            \end{equation}
        \end{lemma}
        \begin{proof}
            Let $c>0$, denote $E_c = \{x \in (a,b) : \overline{D} f(x) \geq c\}$. Choose $c' \in (0,c)$. Let $\mathcal{F}$ be the collection of 
        closed bounded intervals $[d,e] \subset (a,b)$, for which 
        \begin{equation}\label{InequalityOfUpperDerivativeEstimation}
            c'(e-d) \leq f(e)-f(d)
        \end{equation}
            Now we prove that $\mathcal{F}$ is a Vitali cover of $E_c$. Since $$c' < c \leq \overline{D}f =\lim_{h \to 0} \left[ \sup_{0<|t|\leq h} \frac{f(x+t)-f(x)}{t}\right] $$ 
        on $E_c$, for all $x \in E_c$
            \begin{equation*}
                \begin{aligned}
                    \lim_{h \to 0} \left[ \sup_{0<|t|\leq h} \frac{f(x+t)-f(x)}{t}\right] &> c'\\
                    \exists h', \ s.t. \ \forall h \leq h', \ \ \sup_{0<|t|\leq h} \frac{f(x+t)-f(x)}{t} &> c'\\
                    \exists h', \ s.t. \ \forall h \leq h', \ \exists |t'| < h, \ \ \frac{f(x+t')-f(x)}{t'} &> c'\\
                    \exists h', \ s.t. \ \forall h \leq h', \ \exists |t'| < h, \ \  f(x+t')-f(x) &> c'(x+t' - x)\\
                \end{aligned}
            \end{equation*}
            So $[x, x+t]$ is in $\mathcal{F}$, which indicated $E_c \in \mathcal{F}$; since $h$ can be arbritarly small, so $\mathcal{F}$
        is a Vitali cover of $E$. By Vitali cover lemma, we can find a finite disjoint sub-collection of closed bounded intervals, say 
        $\{I_k\}_{k=1}^{n}=\{(d_k, e_k)\}_{k=1}^{n}$,
        such that, $$ m^*(E_c - \bigcup_{k=1}^{n}I_k) < \epsilon$$\par
            Since $E_c =( E_c - \bigcup_{k=1}^{n}I_k )\cup \bigcup_{k=1}^{n}I_k$, by the sub-additivity of the outer measure, 
        additivity of the measure, and \eqref{InequalityOfUpperDerivativeEstimation} we have 
            \begin{equation*}
                \begin{aligned}
                    m*(E_c) &\leq m^*(E_c-\bigcup_{k=1}^{n}I_k) + m(\bigcup_{k=1}^{n}I_k)\\
                    &\leq \epsilon + \sum_{k=1}^{n}l(I_k)\\
                    &= \sum_{k=1}^{n} (e_k - d_k) + \epsilon\\
                    &= \frac{1}{c'}\sum_{k=1}^{n} (f(e_k) - f(d_k)) + \epsilon
                \end{aligned}
            \end{equation*} 
            Notice that $[d_k, e_k] \subset (a,b)$ and they are disjoint, 
            \begin{equation*}
                \begin{aligned}
                    m^*(E_c) \leq \frac{1}{c'}(b-a) + \epsilon 
                \end{aligned}
            \end{equation*} 
            Let $\epsilon \to 0$ and $c' \to c$, we arrive at the desire conclusion. \par
            Notice that, for each natural number $n$, we have $E_n \leq \frac{1}{n} [f(b) -f(a)]$. Let $n \to \infty $,
        we have \eqref{LemmaUpperDerivativeMeasureEstimationInfty} proved.
        \end{proof}
        Now we are prepared to proof Lebesgue's theorem.
        \begin{theorem}[Lebesgue's theorem]
            If $f$ is monotone on $(a, b)$, then it is differentiable a.e. on $(a,b)$. 
        \end{theorem}
        \begin{proof}
            Assume $f$ is increasing, and $(a,b)$ is bounded. If $f$ is not differentiable at $x$, then $\overline{D}f(x) > \underline{D}f$.
        So $x$ will be in $$E_{\alpha, \beta} =\{x \in (a,b): \overline{D}f(x) > \alpha > \beta >\underline{D}f\}$$ 
        where $\alpha, \beta$ are rationals. We only need to show that for every such $\alpha, \beta$, $m^(E_{\alpha, \beta}) = 0$. 
        This is because countable union of measure zeor set is still measure zero.
            Given a fixed $E=E_{\alpha, \beta}$. Let $\epsilon> 0$, let $O$ be a open set for which
            \begin{equation}\label{LebesgueTheoremBoundedOpenSet}
                E \subset O \subset (a, b) \mbox{ and } m(O) \leq m^*(E) + \epsilon
            \end{equation}
            Let $\mathcal{F}$ be the collection of intervals $[c,d]$ such that
            $$ f(d) - f(c) < \beta(d-c)$$  
            Since $\forall x \in E$ we have $\underline{D}f(x)< \beta$, $\mathcal{F}$ is a Vitali cover of $E$. The Vitali Covering Lemma
        tells us $\forall \epsilon >0$, $\exists \{[c_k,d_k]\}_{k=1}^n$ a sub-collection, such that 
            \begin{equation}\label{LebesgueTheoremEstimation1}
                E- \bigcup_{k=1}^n [c_k, d_k] < \epsilon 
            \end{equation}  
            Futher more, we can ask $[c_k,d_k] \in O$, so by the monotonicity of outer measure,
            \begin{equation}
                \sum_{k=1}^{n} f(d_k) - f(c_k) < \beta \sum_{k=1}^{n} (d_k -c_k) \leq \beta m(O) \leq \beta [m^*(E) +\epsilon]
            \end{equation} 
            Consider the ristriction of $f$ on $[c_k,d_k]$, by the previous lemma, we have 
            $$m^*(E \cap [c_k, d_k]) \leq \frac{1}{\alpha}[f(d_k)-f(c_k)]$$
            Thus
            \begin{equation*}
                \begin{aligned}
                    m^*(E) &\leq \sum_{k=1}^{n}m^*(E\cap(c_k,d_k))+\epsilon \\
                    &\leq \sum_{k=1}^n \frac{1}{\alpha}[f(d_k)-f(c_k)] +\epsilon\\
                    &\leq \frac{1}{\alpha}[\beta m^*(E) +\epsilon] +\epsilon\\
                    &\leq \frac{\beta}{\alpha} m^*(E) +\frac{\epsilon}{\alpha} +\epsilon    
                \end{aligned}
            \end{equation*}
            Let $\epsilon \to 0$, then 
            $$m*(E) = \frac{\beta}{\alpha} m*(E)$$
            From the fact that $\frac{\beta}{\alpha}<1$, $m^*(E)=0$. Then we prove the case of $[a,b]$ is bounded. If it is not bounded, write 
        it as the countable union of bounded intervals. Since the countable union of measure zero set is still measure zero set, 
        the theorem still holds for the unbounded intervals.

        \end{proof}
        Remember our goal in this chapter is to figure out under what condition, the Fundamental Theorem of Calculus holds. Now we know 
    that a monotonic function $f$ is differentiable a.e. on closed intervals $[a,b]$, so we can find a function $g$, which equals the
    $f'$ on the point that $f'$ is well defined. Since measure zero set doesn't influence the value of integral, 
    $$\int_a^b g = \int_a^b f' = f(b) +f(a)$$
    the upper equation might be true.\par
        To illustrate this idea in more clear, and rigorous way, \textbf{divided difference function, average value function} are introduced:
        \begin{equation*}
            \begin{aligned}
                \mbox{Diff}_h f(x) = \frac{f(x+h) - f(x)}{h}\\
                \mbox{Av}_h f(x) = \frac{1}{h} \int_{x}^{x+h} f
            \end{aligned}
        \end{equation*}
        where $0 <h\leq 1$ is a fixed number, and $f$ is integrable function defined on a closed bounded interval $[a,b]$, which is extended
    to $[a,b+1]$ by letting $f(x) = f(b)$ for $x \in (b,b+1]$.\par
        \begin{remark}
            By a change of variables, we have the following relation: for $a<u<v<b$, 
            \begin{equation*}
                \begin{aligned}
                    \int_u^v \mbox{Diff}_h f(x) &= \frac{1}{h}\int_u^v f(x+h) - \frac{1}{h}\int_u^v f(x)\\
                    &= \frac{1}{h}\int_{u+h}^{v+h} f(x) - \frac{1}{h}\int_u^v f(x)\\
                    &= \frac{1}{h}\int_{v}^{v+h} f(x) - \frac{1}{h}\int_{u}^{u+h} f(x)+ \frac{1}{h}\int_{u}^{v} f(x) - \frac{1}{h}\int_u^v f(x)\\
                    &= \frac{1}{h}\int_{v}^{v+h} f(x) - \frac{1}{h}\int_{u}^{u+h} f(x)\\
                    &= \mbox{Av}_h f(v) - \mbox{Av}_h f(u)
                \end{aligned}
            \end{equation*}
        \end{remark}
        
        \begin{corollary}\label{IncreasingFunctionFTCLikeInequality}
            Let $f$ be an increasing function on the closed bounded interval $[a,b]$. Then $f'$ is integrable over $[a,b]$ and 
        \begin{equation}
            \int_{a}^{b} f' \leq f(b) - f(a)
        \end{equation}
        \end{corollary}
        \begin{proof}
            With extending $f$ to $[a,b+1]$ by letting $f=f(b)$ on $(b,b+1)$, $f$ is a increasing function on $[a,b+1]$. By Lebesgue's
        theorem, $f$ is differentiable a.e. on $[a, b+1]$. Since $f$ is measurable(monotonic functions are measurablel), $\mbox{Diff}_h f(x)$
        is measurable. The sequence of function $\mbox{Diff}_{\frac{1}{n}}f$ is non-negative and converges to $f'$ pointwisely a.e. on $[a,b]$
        According Fatou's Lemma, we have 
        \begin{equation}\label{LebesgueCorolaryEstimation1} 
            \int_a^b f' \leq \lim \inf \int_a^b \mbox{Diff}_{\frac{1}{n}}f    
        \end{equation}
        
        By the equality in the above remark,
        \begin{equation*}
            \begin{aligned}
                \int_a^b \mbox{Diff}_{\frac{1}{n}}f &= \left[n \int_b^{b+\frac{1}{n}} f - n \int_a^{a+\frac{1}{n}} f \right]\\
                &=\lim \left[n \int_b^{b+\frac{1}{n}} f - n \int_a^{a+\frac{1}{n}} f \right]\\
                &\leq  f(b) - f(a)  
            \end{aligned}
        \end{equation*}
        Thus 
        \begin{equation}\label{LebesgueCorolaryEstimation2}
            \lim \sup \int_a^b \mbox{Diff}_{\frac{1}{n}}f \leq f(b) -f(a)
        \end{equation}
        Together \eqref{LebesgueCorolaryEstimation1} and \eqref{LebesgueCorolaryEstimation2}, we have the corollary proved.
        \end{proof}
    \section{Bounded Variation Functions}
        Since the monotonic functions are differentiable a.e., the difference of two monotonic funcitons is still differentiable a.e..
    It turned out that the class of function which can be represent as the difference of two monotonic function is suprisingling large.
    In this section, we first define what is bounded variation function, then prove the collection of bounded variation functions is 
    the collection of the functions which can be express as the difference of monotonic functions.
        \begin{definition}[Variation, total variation and bounded variation function]
            Given $f:[a,b] \to \RR$, and a partition of $[a,b]$, $P= {a = x_1, x_2, \dots, x_n=b}$, we define the \textbf{variation} of $f$ on $[a,b]$ respect
        to $P$ as 
            \begin{equation}
                V(f,P) = \sum_{k=1}^{n-1} |f(x_{k+1}) - f(x_k)|
            \end{equation}
            And define the \textbf{total variation} of $f$ on $[a,b]$ as 
            \begin{equation}
                TV(f) = \sup {V(f,P) : P \mbox{ is a partition of } [a,b]}
            \end{equation}
            If $TV(f)$ is finite , we say that $f$ is a \textbf{bounded variation function}.
        \end{definition}
        \begin{example}
            Let $f$ be a monotonic function on $[a,b]$, than it is of bounded variation on $[a,b]$. One can easily varify that for any 
        partition $P$,
            $$V(f,P) = \sum_{k=1}^{n-1} |f(x_{k+1}) - f(x_k)| = |f(b)-f(a)|$$ 
        \end{example}
        \begin{example}
            Let $f$ be Lipschitz on $[a,b]$, which means exists a $\lambda$ such that $\lambda (d-c) \geq |f(d) - f(c)|$ for any 
        $a\leq c \leq d \leq d$. Then $f$ is of 
        bounded variation on $[a,b]$. This is because given any partition $P$,
            \begin{equation*}
                \begin{aligned}
                    T(f,P) &= \sum_{k=1}^{n-1} |f(x_{k+1}) - f(x_k)| \\  
                    &\leq \sum_{k=1}^{n-1} \lambda (x_{k+1}-x_{k})\\
                    &\leq \lambda (b-a)
                \end{aligned}
            \end{equation*}
            Thus $TV(f) \leq \lambda (b-a)$
        \end{example}
        
        \begin{theorem}[Jordan's Theorem]
            Let $f:[a,b] \to \RR$ be a function, where $[a,b]$ is bounded. $f$ is of the bounded variation on $[a,b]$ if and only if $$f = h - g$$ where $h$ and $g$
        are increasing function.
        \end{theorem} 
        \begin{lemma}
            Every function of bounded variation on a closed bounded interval $[a,b]$, can be express as the difference of two increasing
        funcitons. 
        \end{lemma}
            This lemma is simply 'the half' of Jordan's Theorem. We prove it by construct two increasing functions whose difference 
        is $f$ from the property of total variation.
        \begin{proof}
            Observing that given a partition $P$, and any number $c \in (a,b)- P$, 
            \begin{equation*}
                \begin{aligned}
                    T(f,P \cup {c})- T(f,P) = |c - x_i|+|x_{i+1}-c|-|x_i-X_{i+1}| \geq 0
                \end{aligned}
            \end{equation*}
            By the mathematic induction, we can conclude that for any refinement of $P$, say $P'$,
            $$V(f.P)\leq V(f,P')$$
            Thus 
            $$TV(f|_{[a,b]}) = TV(f|_{[a,c]})+TV(f|_{[c,b]})$$
            From the properties above, we can legally defined what we call the \textbf{total variation function }for $f$:
            \begin{equation}
                TV_f : x \mapsto TV(f|[a,x])
            \end{equation}
            We can see $TV_f$ is monotonic. We show that $f+TV_f$ is also monotonic. Given $x ,x' \in [a,b]$, and $x>x'$, then 
            \begin{equation*}
                \begin{aligned}
                    f(x) +TV_f(x) - [f(x')+TV_f(x')] &= TV_f(x) - TV_f(x')  + f(x) - f(x')\\
                    &= TV(f|_{[a,x]}) - TV(f|_{[a,x']}) + [f(x) - f(x')]\\
                    &= TV(f|_{[x',x]}) +[f(x) - f(x')]
                \end{aligned}
            \end{equation*}
            Notice $P={x,x'}$ is the roughest partition of the interval $[x,x']$, so by the fact that $TV$ is the supremum, $f(x) +TV_f(x) - [f(x')+TV_f(x')]$
            above is bigger than zero. Thus $f+TV_f$ is also monotonic. Since  
            $$f = (f +TV_f) - TV_f$$
            Thus we find two increasing function whose difference is $f$.
        \end{proof}
        \begin{proof}\textbf{(Proof of Jordan's Theorem)}
            By the above lemma, we prove the one direction of the theorem. \par
            Suppose $f = h - g $, where $h, g$ are increasing functions on $[a,b]$. We need to show that $f$ is of bounded variation.
        For any partition $P= \{a=x_0, x_1, \dots, x_n = b\}$ of $[a,b]$, 
        \begin{equation*}
            \begin{aligned}
                V(f, P) &=\sum_{k=1}^{n} |f(x_i) -f(x_{i-1})|\\
                &= \sum_{k=1}^{n} |h(x_i) - g(x_i) -h(x_{i-1})+g(x_{i-1})|\\
                &\leq \sum_{k=1}^{n} |h(x_i) -h(x_{i-1})|+|g(x_{i-1})- g(x_i)|\\ 
                &\leq \sum_{k=1}^{n} |h(x_i) -h(x_{i-1})|+\sum_{k=1}^{n}|g(x_{i-1})- g(x_i)|\\
                &= V(h,P) +V(g, P) = h(a)-h(b) +g(a) -g(b)
            \end{aligned}
        \end{equation*}
        By the property of supremum, 
        $$TV(f) \leq TV(h) + TV(g) < +\infty$$
        \end{proof}
    \section{Absolute Continuous Functions}
        \begin{definition}[Absolute continuous functions]
            Let $f:[a,b] \to \RR$ be a function defined on a close bounded interval. We say $f$ is of \textbf{absolutely converge} on $[a,b]$ if and only 
        if $\forall \epsilon > 0$, there exist a $\delta$, such that for any finite disjoint collection $\{[c_k,d_k]\}_{k=1}^n$(where 
        $[c_k, d_k] \subset (a,b)$), which satisfy 
        \begin{equation*}
            \sum_{k=1}^n d_k -c_k < \delta
        \end{equation*}
        we have 
        \begin{equation*}
            \sum_{k=1}^n |f(d_k) -f(c_k)| < \epsilon
        \end{equation*}
        \end{definition}
        \begin{proposition}[Absolute continuous is stronger than continuous]
            Suppose $f:E \to \RR$ is absolutely continuous on 
            $E$. Then $f$ is continuous.
        \end{proposition}
        \begin{proof}
            $\forall \epsilon > 0$, we can find a $\delta$ such that
        if $d -c < \delta$, $|f(d) -f(c)| < \epsilon$. Above statement
        is the case when we take only one small interval, which is 
        obviously equivalent to the definition of continuous. 
        \end{proof}
            Now we give some example of this type of functions.
            \begin{proposition}[Lipschitz functions are absolute continuous]
                Let $f:[a,b] \to \RR$ be a Lipschitz function. Then it is of absolute continuous functions.
            \end{proposition}
            \begin{proof}
                Since $f$ is Lipschitz, so for all $a \leq c \leq d \leq b$, there exists $\lambda$ such that
                $$f(d) - f(c) \leq \lambda (d-c)$$
                Let $\epsilon> 0$, and collection of finite disjoint closed sub-intervals $\{[c_k,d_k]\}_{k=1}^n$ such that 
            $$\sum_{k=1}^n d_k -c_k < \frac{\epsilon}{\lambda}$$ 
                Then
                \begin{equation*}
                    \begin{aligned}
                        \sum_{k=1}^n f(d_k) -f(c_k) &\leq \sum_{k=1}^n \lambda (d_k - c_k)\\
                        &= \lambda \sum_{k=1}^n  (d_k - c_k)\\
                        &< \lambda \cdot \frac{\epsilon}{\lambda} = \epsilon
                    \end{aligned}
                \end{equation*}
                So the proposition is proved. 
            \end{proof}
            \begin{lemma}[Linear combination]
                Let $f$ and $g$ be two absolute continuous functions on $[a,b]$, and $\alpha, \beta \in \RR$, then  
            $\alpha f + \beta g$ is still a absolute continuous function.
            \end{lemma}
            \begin{proof}
                We first prove the scaler multiple part. Given $\epsilon> 0$, we have $\delta,$ such that for all collection of sub-intervals $P$, 
            of $[a,b]$ which satisfy
            $$\sum_{k=1}^n d_k -c_k < \delta \\$$
            we have 
            $$\sum_{k=1}^n |f(d_k) -f(c_k)| <\frac{\epsilon}{|\alpha|} \\$$
            It is not hard to see 
            $$\sum_{k=1}^n |\alpha f(d_k) - \alpha f(c_k)| = |\alpha| \sum_{k=1}^n |f(d_k) -f(c_k)| \leq \epsilon \\$$
                Now we prove the additivity. Given $\epsilon> 0$, we have $\delta, \delta_1, \delta_2$ such that for all collection of sub-intervals $P$, $P'$ of $[a,b]$ which satisfy
            $$\sum_{k=1}^n d_k -c_k <\delta_1 \ \ \ \sum_{k=1}^n d'_k -c'_k <\delta_2$$
            we have 
            \begin{equation*}
                \begin{aligned} 
                    \sum_{k=1}^n |f(d_k) -f(c_k)| <\frac{\epsilon}{2} \\ 
                    \sum_{k=1}^n |g(d'_k) -(g'_k)| <\frac{\epsilon}{2}    \\
                \end{aligned}
            \end{equation*}
            Let $\delta = min{\delta_1, \delta_2}$, and $Q = \{[c''_k, d''_k]\}$ be the collection of sub-intervals such that 
            $$\sum_{k=1}^n d_k -c_k < \delta$$
            \begin{equation*}
                \begin{aligned}
                    \sum_{k=1}^n |f(d_k) + g(d_k) -f(c_k) -g(c_k)| &\leq  \sum_{k=1}^n |f(d_k) -f(c_k)| + \sum_{k=1}^n |g(d'_k) -(g'_k)|\\
                    \leq \epsilon
                \end{aligned}
            \end{equation*}
            Thus we finish this lemma.
            \end{proof}
            \begin{lemma}
                Suppose $f$ is of absolutely converge, then $TV_f$ is also a absolutely converge.
            \end{lemma}
            \begin{proof}
                Since $f$ is of absolutely converge, $\forall \epsilon>0, \ \exists \delta$ such that for any  
            $\{[c_k, d_k]\}$, a finite collection of closed sub-intervals with the following property:
                $$\sum_{k=1}^n d_k -c_k < \delta$$
                we have
                $$\sum_{k=1}^n |f(d_k) -f(c_k)| < \frac{\epsilon}{2}$$
                Let $P_k$ be a partition of $[c_k, d_k]$, then we know 
                $$\sum_{k=1}^n |V(f|_{[c_k,d_k]},P_k)| < \frac{\epsilon}{2}$$
                This is because of we are only taking more points in the small intervals which is already in the collection.
            The upper equation can be apply to any partition, so the following is true,
                \begin{equation*}
                    \begin{aligned}
                        \epsilon &> \frac{\epsilon}{2}\\
                        &\geq \sum_{k=1}^n |TV_{f|_{[c_k,d_k]}}|\\
                        &=\sum_{k=1}^n |TV_{f|_{[a,d_k]}} -TV_{f|_{[a,c_k]}}|\\
                        &=\sum_{k=1}^n |TV_f(d_k) -TV_f(c_k)|
                    \end{aligned}
                \end{equation*}
                Thus we prove the lemma.
            \end{proof}
        \begin{theorem}[Absolute continuous means bounded variation]
            If $f:[a,b] \to \RR$ is absolutely continuous on a closed bounded $[a,b]$, then $f$ is of bounded variation on $[a,b]$. More over,
        we can find $f= g-h$ where $g$ and $h$ are increasing bounded variation functions on $[a,b]$.
        \end{theorem}
        \begin{proof}
            We first prove that $f$ is of bounded variation on $[a,b]$. Since $f$ is absolutely continuous, given $\epsilon>0$, we can find a $\delta>0$,
        such that for all $\{c_k,d_k\}_[k=1]^n$ which satisfy $\sum_{k=1}^n (d_k -c_k) < \delta$  
        \begin{equation}
            \sum_{k=1}^n |f(d_k) - f(c_k)| \leq \epsilon
        \end{equation}
            Take a partition $P = \{x_1, \dots x_n\}$ of $[a,b]$, such that $x_i - x_{i-1} \leq \delta$. Since the variation has a sense 
        of monotonicity respect to the partition, it is surffice to consider such $P$ above. 
            Due to the additivity of the total variance, we have 
            \begin{equation}\label{AbusolutContinuousBoundedVariation1}
                TV_f = \sum_{i=1}^{n-1} TV_{f|_{[x_i,x_i+1]}}
            \end{equation}
            Notice that since $x_{i+1} - x_i \leq \delta$, so no matter what partition $P'$ is taken, by the condition of absolute continuous,
        we have 
            $$ T(f,P') \leq \epsilon$$
            Let $\epsilon =1$, together with \eqref{AbusolutContinuousBoundedVariation1}, we have 
            \begin{equation*}
                \begin{aligned}
                    TV_f &\= \sum_{i=1}^{n-1} TV_{f|_{[x_i,x_i+1]}} \\
                    &\leq N    
                \end{aligned}
            \end{equation*}
            So we proved that $f$ is of the bounded variation. Now we need find two increasing absolutely continuous on $[a,b]$. By the 
        above lemma $TV_f$ and $f-TV_f$ is absolute converge, so we prove the theorem.

        \end{proof}


        \begin{theorem}[Absolute continuous and uniform integrable]\label{AbsoluteContinuousUniformIntegrable}
            Let $f:[a,b] \to \RR$ be a continuous function defined on a closed bounded interval. $f$ is of absolutely continuous if and only if 
        the collection of function $$\{\mbox{Diff}_hf\}_{0<h\leq 1}$$ is uniform integrable.
        \end{theorem}
        \begin{proof}
            Suppose $f$ is continuous and  
        $\{\mbox{Diff}_hf\}_{0<h\leq 1}$ is uniformly integrable
        on $E$. Let $\epsilon >0$, we have a $\delta$ such that 
        for $E \subset [a,b]$, given any $0<h\leq 1$,
            $$\int_E |\mbox{Diff}_h f| < \frac{\epsilon}{2}$$
            Let $\{[c_k, d_k]\}$ be finite collection of sub-intervals
        which has the total lenth less than $\delta$.
            Remember the following equation holds
            $$\int_{c_k}^{d_k} |\mbox{Diff}_h f| \leq |\int_{c_k}^{d_k} \mbox{Diff}_h f| = |Av_h (d_k)-Av_h (c_k)|$$
            Since $f$ is continuous, 
            $$\lim_{h \to \infty} |Av_h (d_k)-Av_h (c_k)| =|f(d_k) -f(c_k)|$$
            Then 
            $$\sum_{k=1}^{\infty} f(d_k)- f(c_k) \leq \frac{\epsilon }{2} <\epsilon$$
            Now we prove the converse.
            Suppose $f$ is absolutely continuous on $E$. 
            By the previous theorem, one can always write a absolutely,
            continuous function as the difference of two increasing function.
            Since by triangle inequality, the difference of two set of uniformly
            integrable functions $\{f_1 -f_2\}$, is uniformly integrable. So,
            we only need to prove the case of $f$ an increasing. \par
            Let $\epsilon> 0$, $\exists \delta$ such that for finite 
        collection of sub-interval $\{[c_k,d_k]\}$,
        $$\sum_{k=1}^{n} d_k-c_k < \delta \Rightarrow \sum_{k=1}^{n} f(d_k)-f(c_k) < \epsilon$$ 
        Notice that to prove the uniform integrability, one need 
        deal with all measurable subset, but in the condition of 
        absolute continuity, only interval is provided. So we need to 
        use some measure theory with link the general measurable set to 
        finite many intevals.\par 
            Given any measurable set $E$, we can find a $G_{\delta}$ set 
        such that their difference is a measure zero. For this $G_{\delta}$ set ,
        $$G_{\delta} = \bigcap_{n=1}^{\infty} O_n = \bigcap_{n=1}^{\infty} \bigcup_{m=1}^{\infty} I_{nm} $$
        where $O_n$ is descending. By continuity
        of the measure 
        $$\lim_{n\to infty} O_n = G_{\delta} , \ 
        \lim_{M\to \infty} \bigcup_{m=1}^M I_{nm} = O_n \ \Rightarrow 
        \lim_{n\to \infty} \bigcup_{m=1}^n I_{nm} = G_{\delta} < \infty$$
        Thus the only thing need to show is $\forall \epsilon > 0$, $\exists \delta$
        such that for all finite collection of open intervals
        $\{(c_k, d_k)\}$,
        \begin{equation}
            \sum_{k=1}^n d_k - c_k < \delta \Rightarrow 
            \int_E f < \epsilon
        \end{equation}
        where $E = \cup_{k=1}^n (c_k,d_k)$.
        Choose $\delta >0$ as the response to the $\frac{\epsilon}{2}$
        challenge regarding the criterion for the absolute continuity
        of $f$ on $[a, b+1]$. Let $g(t) = f(v+t) -f(u+t)$ for 
        $t \in [0,1]$ $u<v, \ u,v\in [a,b]$
        \begin{equation*}
            \begin{aligned}
                \int_u^v \mbox{Diff}_h f &= \mbox{Av}_hf(v) - \mbox{Av}_hf(u)\\
            &= \frac{1}{h}[ \int_v^{v+h} f  -\int_u^{u+h} f]\\
            &= \frac{1}{h}[ \int_0^{h} f(v+t)  -\int_0^{h} f(u+t)]\\
            &=\frac{1}{h} \int_0^{h} g
            \end{aligned}
        \end{equation*}
        Since $$\sum_{k=1}^n d_k- c_k <\delta \Rightarrow 
        \sum_{k=1}^n d_k+t - c_k +t<\delta
        \Rightarrow g < frac{\epsilon}{2}$$
        Thus 
        $$\int_E \mbox{Diff}_h f = \frac{1}{h} \int_0^{h} g < \epsilon $$
        The theorem is proved.
        \end{proof}
    \section{Integrating derivative and Differentiating Indefinite Integral}
        Now we have enough tools to link the derivative and the integral 
    together in the context of Lebesgue integral. We have the 
    discrete formulation of the fundamental theorem of caculus:
        For a continuous $f$ on the closed bounded inteval $[a,b]$,
    $$\int_a^b \mbox{Diff}_hf = \mbox(Av)_hf(b) - \mbox(Av)_hf(a)$$
        Since $f$ is continuous the right hand side of the equation 
    converge to $f(b)-f(a)$ as $n \to \infty$.
        Now we prove that if $f$ is absolutely continuous, the
    left hand side of the integral converge to $\int_a^b f'$, therefore
    establish the fundamental theorem of integral calculus for Lebesgue
    integral.
        \begin{theorem}\label{LebesgueFNCIntegration}
            Let the function $f$ be a absolute continuous function
        on the closed bounded interval $[a,b]$. $f'$ is defined 
        a.e. on $[a,b]$. $f'$ is integrable over $[a,b]$, and\
        \begin{equation}\label{FTCIntegralLebesgue}
            \int_a^b f' = f(b)-f(a)
        \end{equation}
        \end{theorem}
        \begin{proof}
            From the discrete formulation of the FTC which
        mentioned above, 
        \begin{equation*}
            \lim_{n \to infty} \int_a^b \mbox{Diff}_h f = f(b)- f(a)
        \end{equation*} 
            Since absolute continuous functions are difference of 
        monotonic functions, $f'$ is well defined a.e. on $[a,b]$.
        By theorem \ref{AbsoluteContinuousUniformIntegrable}, 
        $\{\mbox{Diff}_h f\}$ is uniformly integrable. We take a 
        sequence of functions $\{\mbox{Diff}_{\frac{1}{n}} f\}_{n=1}^{\infty}$,
        which converge to $f'$ a.e. on $[a,b]$. By proposition \ref{LimitFunctionOfUniformIntegrable},
        $f'$ is integrable on $[a,b]$. By Vitali Convergence Theorem
        (we are integrating over a bounded $[a,b]$),
        \begin{equation*}
            \begin{aligned}
                \lim_{n \to \infty} \int_a^b \mbox{Diff}_{\frac{1}{n}} f
                &= \int_a^b f' \\
                \Rightarrow \int_a^b f' &= f(b) -f(a)\\ 
            \end{aligned}
        \end{equation*}
        Thus the theorem is proved.
        \end{proof}
            Another half of the fundamental theorem is 
        $$\frac{d }{d x} \int_a^x f = f(x)$$
            To show this, we first need to define what is indefinite 
        integral in the Lebesgue context, and prove some properties of it.
            \begin{definition}[Indefinite integral]
                Let $f$ be a function defined on a closed bounded intervals.
            We say $f$ is a indefinite integral if there is a 
            $g$ which is integrable over $[a,b]$ and 
            $$f(x) = f(a) + \int_a^x g$$
            for all $x \in [a,b]$
            \end{definition}
            \begin{theorem}\label{AbsoluteContinuousAndIndefiniteIntegral}
                A function $f$ on a closed bounded $[a,b]$ is absolutely
            continouos if and only if $f$ is a indefinite integral on $[a,b]$
            \end{theorem}
            \begin{proof}
                If $f$ is a asbsolute continuous function, then by theorem
            \ref{LebesgueFNCIntegration}, $f'$ is defined a.e on $[a,b]$,
            and is integrable, and 
            $$\int_a^b f' = f(b) - f(a)$$
                Since the restriction of $f$ and $f'$ on $[a,x]$
            still has the properties above,
            $$f = f(a) +\int_a^x f'$$
                On the other hand, suppose $f$ is an indefinite integral,
            which means $f(x) = f(a) + \int_a^x g$, for some integrable
            $g$.
                \begin{equation*}
                    \begin{aligned}
                        |f(d_k) - f(c_k)| &= |f(a) + \int_a^{d_k}g -f(a) - \int_a^{c_k} g|\\
                        &=| \int_{c_k}^{d_k} g|
                    \end{aligned}
                \end{equation*}
                By \ref{EstimationOfIntOverSmallSet}, we proved
            that $f$ is a function of absolute continuous.
            \end{proof}
            \begin{corollary}
                Let $f$ be a monotone function on a closed bounded $[a,b]$.
            Then $f$ is absolutely continuous if and only if 
            $$\int_a^b f' = f(b) -f(a)$$
            \end{corollary}
            \begin{proof}
                Suppose $f$ is absolute continuous, then by theorem 
            \ref{LebesgueFNCIntegration}, we are done.\par
                Suppose $f$ is monotone and 
                $$\int_a^b f' = f(b) -f(a)$$
                \begin{equation*}
                    \begin{aligned}
                        0 &= \int_a^b f' f(b) - f(a)\\
                        &= \int_a^x f' -[f(x) -f(a)] + \int_x^b f' -[f(b)-f(x)]\\
                    \end{aligned}
                \end{equation*}
                By corollary \ref{IncreasingFunctionFTCLikeInequality},
            $\int_a^x f' -[f(x) -f(a)] \leq 0$ and $\int_x^b f' -[f(b) -f(x)] \leq 0$.
            Two non-positive number's sum is zero, then they are all zero.
                So 
                $$\int_a^x f' = f(x) -f(a)$$
                is true for all $x \in [a,b]$. Thus we have 
                $$f(x) = f(a) +f(x) -f(a)= f(a) \int_a^x f' $$
                By the theorem \ref{AbsoluteContinuousAndIndefiniteIntegral},
            $f$ is absolutely contionuous on $[a,b]$.  
            \end{proof}
            
            \begin{lemma}
                Let $f$ be integrable over the closed, bounded interval $[a,b]$. 
            $$f(x) = 0 \mbox{ for almost all } x \in [a,b]$$
            if and only if 
            $$\int_{x_1}^{x_2} f = 0 \mbox{ for all }(x_1,x_2) \subset [a,b]$$
            \end{lemma}
            \begin{proof}
                $"\Rightarrow"$ part of the prove is trivial. \par 
                Suppose $\int_{x_1}^{x_2} f = 0 \mbox{ for all }(x_1,x_2) \subset [a,b]$.
            We claim that  
            \begin{equation}\label{FTCDifferentiationLemma1}
                \int_E f = 0 \mbox{ for all }E \subset [a,b]
            \end{equation}
            is true for every measurable subset. We are actually using the 
            same argument in the proof of the 
            theorem \ref{AbsoluteContinuousUniformIntegrable}. Using 
            the continuity of the measure, one can finally show that for every
            $G_{\delta}$ set, the above equation is true. Since for any 
            measurable set we can find a $G_{\delta}$ such that their 
            difference is a set of measure zero, which doesn't influence 
            the integral. So \eqref{FTCDifferentiationLemma1} is true for 
            any measurable subset of $[a,b]$.\par
                Let $E^+ = \{x\in [a,b]: f(x) \geq 0\}$ and 
            $E^- = \{x \in [a,b]:f(x)\leq 0\}$, which are both 
            measurable. Thus we have the following two integrals of 
            non-negative functions to be zero:
            $$\int_{E^+} f = 0 = \int_{E^-} -f$$
                The integral of non-negative function is zero 
            if and only if the function equals $0$ a.e.. Thus 
            $f= 0$ a.e. on $E^+$ and $-f = 0$ a.e. on $E^-$, which 
            proves the lemma.
            \end{proof}
            \begin{theorem}
                Let $f$ be integrable over the closed bounded 
            interval $[a,b]$. Then
            \begin{equation}\label{FTCDifferentiation}
                \frac{d}{dx} \left[\int_a^x f\right] = f(x)
            \end{equation}
            for almost all $x \in (a,b)$ .
            \end{theorem}
            \begin{proof}
                We defined $F(x) = \int_a^x f$ for $x\in [a,b]$.
            Since $F$ is a indefinite integral, it is absolutely
            continuous on $[a,b]$ and $f' = \frac{d}{dx} F $ exist a.e. on $[a,b]$.
            To prove \eqref{FTCDifferentiation}, we only need to show 
            $f = f'$ a.e. on $[a,b]$. Using the lemma above, 
            it is surffice to show that 
            $$\int_c^d f -f' = 0$$
            for all $[c,d] \subset [a,b]$. Now take any sub inteval $[c,d]$,
            by the properties of the integration, and the theorem \ref{LebesgueFNCIntegration}
            \begin{equation*}
                \begin{aligned}
                    \int_c^d f -f' &= \int_c^d f -\int_c^df'\\
                    &=\int_c^d f - [F(d) -F(c)]\\
                    &=\int_c^d f - [\int_a^d f -\int_a^c f ]\\
                    &= 0 
                \end{aligned}
            \end{equation*}
            The theorem is proved.
            \end{proof}
            \begin{definition}[Singular]
                Let $f$ be a function of bounded variation on a closed bounded
            inteval. We say $f$ is singular if and only if it's derivative vanishes
            a.e. on $[a,b]$.
            \end{definition}
            \begin{theorem}[Lebesgue decomposition]
                Let $f$ be a function of bounded variation on a 
            close bounded $[a,b]$. Then $f = g +h$ where $g$ is absolutely
            continuous and $h$ is singular.
            \end{theorem}
            \begin{proof}
                Since $f$ is of bounded variation, $f'$ is defined 
            a.e. on $[a,b]$ and integrable(by Jordan's theorem and corollary \ref{IncreasingFunctionFTCLikeInequality}). 
                We define $g = \int_a^x f'$, and $h = f -\int_a^x f'$.
                $f=g+h$ is obvious. $g$ is a indefinite integral, so is 
            absolutely continuous; from the theorem above, $h$ is singular.
            \end{proof}
\chapter{Lp Space: Completeness and Approximation}
    In this chapter, we will study some of the Lebesgue integrable function,
in a very different perspective. Instead of considering them as 
isolated functions, we study the space where they lie in.
You will see later how does the $L^p$ space defined, which
is a normed linear spcace 
Remember that 
we have promised that comparing to Riemann integral, Lebesgue integral has many virtue.
Our goal in this chapter is to prove a very pleasing virtue that 
$L^p$ spcaces are complete, or namely, they are what we called 
Banach space.
    \section{Normed Linear Space of Functions}
        A lineary space(or is called vector space)
    is a set of object that has a linear structure on it, which
    means we can do addition and scalar multiplication(usualy scalars are borrowed 
    from a field) on it. Just like the vector spaces we have 
    studied in the course of linear algebra, the set of measurable function, or continuous 
    function are all linear spaces. 
 
    Let's first have a quick review of what is a norm.
        \begin{definition}[Norm]
            Let $X$ be a linear space. We say the function $\lVert \cdot\rVert$, 
        which maps the elements $X \times X$ to a real number, is a norm if the 
        followings is satisfied:
            \begin{enumerate}
                \item (Non-negativity)
                $$\lVert f \rVert \geq 0 \mbox{ and } \lVert f \rVert = 0 \Leftrightarrow f = 0$$
                \item (Positive homogenety)
                $$\lVert af \rVert = |a|\lVert f\rVert$$
                \item (The triangle inequality)
                $$\lVert f+g\rVert \leq \lVert f\rVert+\lVert g\rVert$$
            \end{enumerate}
            Particularly, if $X$ is a space of functions, for $f \in X$ whose norm
        equals to 1, we call $f$ a \textbf{unite function}. $\frac{f}{\lVert f \rVert}$
        is always a unite function, we call this function the normalization of $f$.
        \end{definition}
        A norm could act as a distance function in a metric space.
    The norm we are going to defined on the $L^p$ space is the following:
    $$\left(\int_E |f|^p\right)^{\frac{1}{p}}$$
    To normalize the space, we need to show the three properties above
    is satisfied. We first consider the nomarlization of the $L^1$ space.
        \begin{definition}[$L^1$ space(not formal definition!)]
            Let $L^1(E) = \{f:E \to \RR^*: f \mbox{ is integrable over }E \}$.
        The norm we hope to define on $L^1$ is 
        $$\left(\int_E |f|\right)$$
        \end{definition} 
        The first problem is not only the norm(though now it isn't a norm) of the zero function\
    is zero. For every $f$, such that $f= 0$ a.e on $E$, 
    $$\left(\int_E |f|\right) = 0$$
        Our solution here is not to defined $L^1$ as the space on 
    integrable functions, but on the equivalence classes of 
    integrable functions. For which equivalence relation we should
    take here is rather obvious.
        \begin{definition}
            Let $\mathcal{L}(E)$ be the collection of integrable 
        functions defined on $E$. In this section, let $f\sim g$ 
        to denote the relation for $f, g \in \mathcal{L}(E)$ such that $f= g + h$
        where $h=0$ a.e. on $E$.
        \end{definition}
        \begin{proposition}[The $\sim$ is a equivalence relation]
            Using the notation in the upper definition, let 
        $f, f' ,f'' \in \mathcal{L}(E)$, such that $f\sim f'$,
        $f'\sim f''$. we have 
            \begin{enumerate}
                \item $f \sim f $
                \item $f' \sim f$
                \item $f \sim f''$
            \end{enumerate}
        \end{proposition}
        \begin{proof}
            1. $f = f +0 \Rightarrow f \sim f$\par
            2. $f \sim f' \Rightarrow f = f' +h \Rightarrow f' = f-h 
            \Rightarrow f' \sim f$\par
            3. $f sim f' \mbox{ and } f' sim f'' \Rightarrow 
            f = f' +h \mbox{ and } f'= f'' +h' \Rightarrow f = f'' +( h-h')
            \Rightarrow f \sim f''$\par
        \end{proof}
        We use $\mathcal{L}(E)/ \sim$ denote the collection of
    the equivalence classes of the integrable functions which 
    determined by the upper equivalence 
    relation. The fact that our original addition and scalat multiplication
    for integrable function is also well defined for this 
    equivalence classes. The verification(whether the operation is 
    regardless of the choice of representatives) is left to the reader.
        Now we given the formal definition of the $L^1$ space,
    and prove that $\left(\int_E |f|\right)$ is a norm.
        \begin{definition}[$L^1$ space]
            Let $L^1(E) = \mathcal{L}(E)/ \sim$ be the set 
        of equivalence classes. With the original addition and 
        scalar multiplication of integrable functions, $L^1$ is 
        a linear space.     
        \end{definition}
        We will use $f$ instead of the usual notation for 
    equivalence classes, $\bar{f}$, to denote the element in 
    $L^1$(also $L^p$). Since it is more concise and 
    won't cause any ambiguity in our discussion. In fact, 
    in the most of time, we can just treat them as functions.
        \begin{proposition}[$L^1$ is a normed linear space]
            $\left(\int_E |f|\right)$ is a norm of $L^1$.
        So $L^1$ is a normed linear space.
        \end{proposition}
        \begin{proof}
            First of all, since measure zero set doesn't influence 
        the integral, so $\left(\int_E |f|\right)$ is well defined 
        for the equivalence classes. Let $f, g \in L^1$ and $a \in \RR$.\par
            $\left(\int_E |f|\right)$
        is a integral for non-negative functions, so 
        $$\left(\int_E |f|\right) \geq 0$$.
        By proposition \ref{Non-negative function and measure zero set}, 
        $\left(\int_E |f|\right) = 0$ if and only if $f = 0$ a.e. on E.\par
            By the linearity of integration,  
            $$\left(\int_E |af|\right) = \left(\int_E |a||f|\right) = |a|\left(\int_E |f|\right)$$
            Thus we have homogenety.\par
            By the triangle inequality of real number,
            $$\left(\int_E |f+g|\right) \leq 
            \left(\int_E |f|+|g|\right) = \left(\int_E |f|\right)
            +\left(\int_E |g|\right)$$
            We have all the properties proved, so $L^1$ is a normed 
        linear space under $\left(\int_E |f|\right)$
        \end{proof}
        We use the notation $\lVert \cdot\rVert_p$ to denote the $L^p$ norm,
        $$\left[\int_E |f|^p\right]^{\frac{1}{p}}$$
    Before discussing $L^p$ space, let we first have a look at $L^{\infty}$.
        \begin{definition}[Essentially bounded]
            Given $f \in \mathcal{L}(E)$, we say $f$ is \textbf{essentially bounded}
        if there exist a $M\geq 0$ such that 
        $$ |f(x)| \leq M \mbox{ for almost all } x\in E$$
        an we say $M$ is a \textbf{essential upper bound} of $f$.
        \end{definition}
        \begin{definition}[$L^[\infty]$ space]
            Let $L^{\infty}(E)$ be the set of equivalence classes of all 
        essentially bounded functions. 
        \end{definition}
        \begin{proposition}[$L^{\infty}$ is a norm vector space]
            For functions in $L^{\infty}$, we define 
            \begin{equation}\label{NormOfLInftySpace}
                \lVert f \rVert_{\infty} := \inf \{M : M \mbox{ is a essential upper bound of f } \}
            \end{equation}
            With $\lVert f \rVert_{\infty}$, $L^{\infty}$ is a normed linear space.
        \end{proposition}
        The prove is left to the readers. \par While for $L^p$ spaces with $p \in \NN^+$,
    the triangle inequality below ,
    $$\lVert f + g \rVert_p \leq \lVert f \rVert_p +\lVert g \rVert_p
    \Leftrightarrow \left[\int_E |f +g|^p \right]^{\frac{1}{p}} \leq 
    \left[\int_E |g|^p \right]^{\frac{1}{p}}+\left[\int_E |g|^p \right]^{\frac{1}{p}}$$
    which is is call Mincowski inequality, is not that easy to prove 
    $$$$
    \section{Important Inequalities}
        To prove Mincowski inequality, we need Holder's inequality; to prove 
    Holder's inequality, we need Young's inequality. 
        \begin{definition}[Conjugate numbers]
            Given any $p> 1$, we define it's conjugate number $q = \frac{p}{p-1}$.
        So we have $$\frac{1}{p} + \frac{1}{q} = 1$$. $1$, and $\infty$ are defined 
        to be each other's conjugate.
        \end{definition}
        \begin{proposition}[Young's inequality]
            For $p \in (1, \infty)$, with $q$ begin $p$'s conjugate, given any non-negative 
        $a,b$,
        $$ab \leq \frac{a^p}{p} +\frac{b^q}{q}$$
        \end{proposition}
        \begin{proof}
            If one of $a ,  \ b$ is zero, then the inequality is trivial.
        Suppose $a, \ b$ is positive. We define functions
        $g := \frac{1}{p}x^p + \frac{1}{q}-x$ for $x > 0$. Since $g' = x-1$,
        the derivative is increasing on $(1 , +\infty)$,
        decreasing on $(0 1)$, and the fact that $g(1) = 0$,
        we know that $g$ is nonegative on $[0, \infty)$.
            Thus 
            \begin{equation*}
                \begin{aligned}
                    x \leq \frac{1}{p}x^p + \frac{1}{q} 
                \end{aligned}
            \end{equation*}
            Letting $x = \frac{a}{b^{q-1}}$
            \begin{equation*}
                \begin{aligned}
                    \frac{a}{b^{q-1}} &\leq \frac{a^p}{pb^{p(q-1)}} + \frac{1}{q}\\
                    \Rightarrow \frac{a}{b^{q-1}}b^q &\leq \frac{a^p}{pb^q}b^q + \frac{b^q}{q}\\
                    \Rightarrow ab &\leq \frac{a^p}{p} + \frac{b^q}{q}\\
                \end{aligned}
            \end{equation*}
        \end{proof}
        \begin{theorem}[Holder's inequality]
            Let $E$ be a measurable set, $1\leq p < \infty$, and let 
        $q$ be the conjugate of $p$. If $f \in L^p(E)$ and $g \in L^q(E)$,
        Then $f\cdot g$ is integrable, and we have the following inequality:
        \begin{equation}\label{HolderInequality}
            \int_E |f \cdot g| \leq \lVert f \rVert_p \lVert g \rVert_q 
        \end{equation}        
        \end{theorem}
        \begin{proof}
            First consider the case that $p=1$. Then is $g$ essentially
        bounded. By the monotonicity of the integral,
        $$\int_E |f \cdot g| \leq \int_E |f|\lVert g \rVert_{\infty}
        = \lVert f \rVert_1 \lVert g \rVert_{\infty}$$\par
            Suppose $1<p<\infty$. If one of $f$ and $g$ is zero function,
        then we have nothing to prove. If $f$ and $g$ is not zero,
        the Holder's inequality is equivalence to 
        $$\int_E \frac{f}{\lVert f \rVert_p } \frac{g}{\lVert g \rVert_q } = 1$$
        Namely, we only need to show the Holder's inequality for 
        every $f$ and $g$ in $L^p(E)$ and $L^q(E)$, whose norm equals 
        $1$.\par
            By Young's inequality, 
            \begin{equation*}
                \begin{aligned}
                \int_E |f \cdot g| &\leq \frac{1}{p} \int_E f^p  
                + \frac{1}{q} \int_E g^q\\
                &= \frac{1}{p} +\frac{1}{q} =1
                \end{aligned}
            \end{equation*}
            Thus we proved the Holder's inequality.
        \end{proof}
        Now we establish one more lemma to reach the Mincowski 
    inequality. 
        \begin{lemma}[Conjugate functions]
            Let $f \neq 0$ in $L^p(E)$, $f^* = \lVert f \rVert_p^{1-p} \cdot sgn(f) \cdot |f|^{p-1}$.
        Then $f^*$ is in $L^q(E)$, and 
        \begin{equation}\label{fStarMincowski}
            \int_E f \cdot f^* = \lVert f \rVert_p \mbox{ and } \lVert f^* \rVert_q =1
        \end{equation}
        We call $f^*$ the conjugate function of $f$.
        \end{lemma}
        \begin{proof}
            For $p = 1$, $f^*= sgn(f)$. Obviously $f^*$ is in $L^{\infty}$,
        and $\lVert f\rVert_{|infty} = 1$. In this case, \eqref{fStarMincowski}
        is  trivial. \par
            For $p > 1 $,
            $$\int_E |f^*|^q = \int_E \lVert f \rVert_p^{q(1-p)}
            |f|^{q(p-1)} = \lVert f \rVert_p^{-p} \int_E |f|^p$$
            Since $f$ is in $L^p$, $f^*$ is in $L^q$. From the 
        upper equality,
        \begin{equation*}
            \begin{aligned}
                \lVert f^* \rVert_q^q &= \lVert f \rVert_p^{-p} \int_E |f|^p\\\
            &= 1\\
            \Rightarrow \lVert f^* \rVert_q = 1\\    
            \end{aligned}
        \end{equation*}
            \par
            \begin{equation*}
                \begin{aligned}
                    \int_E f \cdot f^* &= \lVert f \rVert_p^{1-p} 
                    \int_E |f|^{p}\\
                    &=  \frac{\lVert f \rVert_p^{p}}{\lVert f \rVert_p^{p-1}}\\
                    &= \lVert f \rVert_p
                \end{aligned}
            \end{equation*}
        \end{proof}
        \begin{theorem}[Mincowski inequality]
            Let $1\leq p\leq \infty$, $f, \ g$ be functions in $L^p(E)$. Then $f+g$ is in $L^p(E)$ and
            \begin{equation}\label{MincowskiInequality}
                \lVert f + g\rVert_p \leq \lVert f \rVert_p +\lVert g \rVert_p
            \end{equation}
        \end{theorem}
        \begin{proof}
            Since we have already proved this inequality for $p= 1 ,\infty$,
        now we only consider the case of $p \in (1, \infty)$.
        Consider the conjugate function of $(f+g)$, $(f+g)^*$.
        By the Holder's inequality and \eqref{fStarMincowski},
        \begin{equation*}
            \begin{aligned}
                \lVert f+g \rVert_p &= \int_E (f+g)(f+g)^*\\
                &=\int_E f(f+g)^* +\int_E g(f+g)^*\\
                &\leq \lVert f \rVert_p \lVert( f +g )^*\rVert_q +
                \lVert g \rVert_p \lVert( f +g )^*\rVert_q\\
                &=  \lVert f \rVert_p  + \lVert g\rVert_p
            \end{aligned}
        \end{equation*}
        So the inequality is proved.
        \end{proof}
        Notice that the Cauchy-Schwarz inequality is the special case 
    of Mincowski inequality, when $p=2$. With the upper theorem, one can conclude that $L^p$ 
    spaces are normed vector space with the norm $\lVert f \rVert_p$.
        \begin{theorem}
            $L^p(E)$ space is a normed linear space with the norm
        $$\lVert f \rVert_p = \left| \int_E |f|^p \right|^{\frac{1}{p}}$$
        \end{theorem}
        \begin{proof}
            We have already showed that $L^p(E)$ is a linear space. So 
        it is suffcient to show that $\lVert f\rVert_p$ is a norm on $L^p(E)$.
        Non-negativity is obvious, and only zero function is mapped to $0$
        is provided by \ref{Non-negative function and measure zero set}.
            Positive homogenety is indicated by the linearity of the 
        integration. Finally, Mincowski inequality proves the triangle inequality.
        \end{proof}
        \begin{corollary}[$L^p$ norm and uniform integration]
            Let $E$ be a  measurable set and $1<p< \infty$. Suppose
        $\mathcal{F}$ is a family of function for which there exists a 
        $M$ such that 
        $$\lVert f\rVert_p \leq M$$
        for all $f \in \mathcal{F}$, then $mathcal{F}$ is uniformly integrable
        over $E$.
        \end{corollary}
        \begin{proof}
            Let $A$ be any finite measure subset of $E$, and $q$ be the 
        conjugate number of $p$, $g$ be the function that equals $1$
        a.e. on $A$. Then $|fg| = |f|$ a.e. on $A$. Since $A$ has finite 
        measure, $g$ is in $L^q(A)$. By Holder's inequality,
        \begin{equation*}
            \begin{aligned}
                \int_A |f| &= \int_A |fg|\\
                &\leq \lVert f \rVert_p \cdot \lVert g\rVert_q\\
                &\leq M \cdot m(A)
            \end{aligned}
        \end{equation*}
            From the estimation above, one can conclude that given 
        $\epsilon > 0$, let the desired $\delta = \frac{\epsilon}{M+1}$,
        then for every $A \subset E$ whose measure is less thant $delta$
        $\int_A |f| < \epsilon$ for all $f \in \mathcal{F}$. Thus 
        $mathcal{F}$ is unifromly integrable over $E$.  
        \end{proof}
        \begin{corollary}
            Let $E$ be a measurable set of finite measure and 
        $1 \leq p_1 <p_2 \leq \infty$, then $L^{p_2} \subset L^{p_1}$.
        Moreover, 
        $$\lVert f\rVert_{p_1} \leq c\lVert f\rVert_{p_2} \mbox{ 
            for all } f \in L^{p_2}$$
        where $c = m(E)^{\frac{p_2-p_1}{p_2p_1}}$ if $p_2 <\infty$
        and $c = m(E)^{\frac{1}{p_1}}$ if $p_2 = \infty$.  
        \end{corollary}
        \begin{proof}
            We first prove the case that $p_2 = \infty$. Suppose 
        $f \in L^{p_2}(E)$, since $f$ is essentially bounded,
        and $E$ has finite measure, it is not hard to see $|f|^{p_1}$
        is integrable. Thus $L^{\infty} \subset L^{p_1}$ for every $p_1 \geq 1$.
        For every $f \in \mathcal{F}$,
            \begin{equation*}
                \begin{aligned}
                    \int_E |f| &\leq m(E) \lVert f\rVert_{\infty}\\
                \Rightarrow \int_E |f|^{p_1} &\leq m(E) \lVert f\rVert_{\infty}^{p_1}\\
                \Rightarrow \left[\int_E |f|^{p_1} \right]^{\frac{1}{p_1}} 
                &\leq m(E)^{\frac{1}{p_1}} \lVert f\rVert_{\infty}\\
                \end{aligned}
            \end{equation*}
        Thus the case of $p_2 = \infty $ is proved. Now suppose 
        $p_2< \infty$. Define $p=\frac{p_2}{p_1}$. Given any
        $f \in L^{p_2}(E)$, $f^{p_1}$ is in $L^p(E)$. Let $g=\chi_E$, 
        which is in $L^q(E)$, where $q$ is the conjugate of 
        $p$. By Holder's inequality,
            \begin{equation*}
                \begin{aligned}
                    \int_E |f|^{p_1} &=\int_E |f|^{p_1} \cdot g \\
                    &\leq \lVert f^{p_1} \rVert_p \cdot \lVert g \rVert_q\\
                    &= \left[\int_E |f|^{p_1p}\right]^{\frac{1}{p}} \cdot m(E)^{\frac{1}{q}}\\
                    &= \left[\int_E |f|^{p_2}\right]^{\frac{p_1}{p_2}} \cdot m(E)^{\frac{p_1(p_2-p_1)}{p_2p_1}}\\
                    &= \lVert f\rVert_{p_2}^{p_1} \cdot m(E)^{\frac{p_1(p_2-p_1)}{p_2p_1}}\\
                \end{aligned}
            \end{equation*}
            Take the both sides of the equation to the $\frac{1}{p_1}$ power, we obtain the desire 
        inequality.
        \end{proof}
        \begin{remark}
            The inclusion of $L^p$ space above is 
        actualy strick. Let $-\frac{1}{p_1}\alpha <-\frac{1}{p_2}$,
        then $f \in L^{p_1}(0,1] - L^{p_2}(0,1]$.\par
            For set of infinite measure, we have no inclusion between 
        $L^p$ spaces.
        \end{remark}
    \section{Riesz Fischer Theorem: Lp is Complete}
        \begin{definition}[Convergence in normed vectore space]
            Let $X$ be a normed vector space normed by $\lVert \cdot \rVert$. 
        For a sequence of element $\{x_n\}$ and a $x$ in $X$, 
        we say $\{x_n\}$ converge to $x$ in $X$ if and only if 
        $$ \lim_{n \to \infty} \lVert x_n -x\rVert = 0$$
        We denote this relation by $x_n \to x$ in $X$, or 
        $\lim_{n\to \infty} x_n = x$ in $X$. When it is clear in 
        the context, we can just say $\lim_{n\to \infty} x_n = x$. 
        \end{definition}
        Consider $C[a,b]$ with the maximum norm(the reader need to 
    verify this is a normed linear space). The convergence in $C[a,b]$
    is equivalence to what we called uniform convergence. 
    $\lVert f_n -f \rVert_{\mbox{max}}$ is the maximum of the 
    difference of the two functions, which illustrate the above 
    idea. Of course we can defined what is \textbf{cauchy} in 
    a normed linear space. \par
        \begin{definition}[Cauchy sequence in normed linear space]
            Let $X$ be a normed vector space normed by $\lVert \cdot \rVert$,
        and $\{x_n\}$ be a sequence of element in $X$. We say that 
        $\{x_n\}$ is cauchy if and only if $\forall \epsilon> 0, \ 
        \exists N$ such that $\forall i,\ j \geq N$,
        $$\lVert x_i - x_j \rVert < \epsilon$$
        We say $X$ is \textbf{complete} if and only if 
        every cauchy sequence converges in $X$. One usualy
        call a complete normed linear space as \textbf{Banach space}.
        \end{definition}
        The final goal of this section is to prove that $L^p$ space is 
    complete.
        \begin{proposition}\label{SubcauchySequence}
            Let $X$ be a normed vector space. Then every converge
        sequence is cauchy. If a cauchy sequence $\{x_n\}$ has a converge 
        subsequence, then $\{x_n\}$ converge. 
        \end{proposition}
        \begin{proof}
            Let $\{x_n\}$ be a converge sequence in $X$. 
        So for all $\epsilon > 0$, $\exists N$ such that,
        $\lVert x_n - x\rVert <\frac{\epsilon}{2}$ is true for all $n\geq N$.
        Let $i , \ j \geq N$. 
        \begin{equation*}
            \begin{aligned}
                \lVert x_i - x_j \rVert &= \lVert x_i +x - x- x_j \rVert\\
                &\leq\lVert x_i +x  \rVert+\lVert x- x_j \rVert\\
                &< \frac{\epsilon}{2} + \frac{\epsilon}{2}\\
                &= \epsilon 
            \end{aligned}
        \end{equation*}
        So is $\{x_n\}$ cauchy.\par 
            Suppose $\{x_n\}$ is a cauchy sequence and $\{x_{n_k}\}$ converges.
        Given $\epsilon$, there exist $N_1, \ N_2$ such that 
        \begin{equation*}
            \begin{aligned}
                \forall i, \ j\geq N_1, \ \ \lVert x_i - x_j \rVert < \frac{\epsilon}{2} \\
                \forall n_k \geq N_1, \ \ \lVert x_{n_k} - x \rVert < \frac{\epsilon}{2} \\
            \end{aligned}
        \end{equation*}
        Let $N = max( N_1 , N_2)$. For $n \geq N$, there exist $n_k > n$. So we have 
        \begin{equation*}
            \begin{aligned}
                \lVert x_n - x \rVert &= \lVert x_n - x_{n_k} + x_{n_k}- x \rVert\\
                &\leq \lVert x_n - x_{n_k} \rVert+\lVert x_{n_k}- x \rVert\\
                &< \frac{\epsilon}{2} + \frac{\epsilon}{2}\\
                &= \epsilon
            \end{aligned}
        \end{equation*}
        \end{proof}
        The proposition above provide a strategy to prove that a space is a Banach 
    space: one can first show that a certain type of cauchy sequence converge, and 
    then show that every cauchy sequence has a subsequence of this type. 
        \begin{definition}[Rapidly cauchy sequences]
            Let $X$ be a linear space normed by $\lVert \cdot \rVert$. A sequence $\{f\}$
        is said to be \textbf{rapidly cauchy} provided there is a convergent series 
        of positive numbers $\sum_{k=1}^{\infty} \epsilon_k$ for which 
        $$\lVert f_{k+1} - f_{k} \rVert \leq \epsilon_k^2$$
        for all $k$
        \end{definition}
        \begin{proposition}\label{RapidlyCauchy}
            Let $X$ be a normed linear space as usual. Every rapidly cauchy sequence 
        is cauchy. Every cauchy sequence has a rapidly cauchy subsequence. 
        \end{proposition}
        \begin{proof}
            Let $\{f_k\}$ be a rapid cauchy sequence.
            By definition, 
            \begin{equation*}
                \begin{aligned}
                    \lVert f_k - f_{k+1} \rVert &\leq \epsilon_k^2\\
                    \Rightarrow \sum_{k=n}^{m}\lVert f_k - f_{k+1} \rVert &\leq \sum_{k=n}^{m} \epsilon_k^2\\
                \end{aligned}
            \end{equation*}
            Since the right hand side series converges, and 
            $$\lVert f_n - f_m \rVert \leq \sum_{k=n}^{m-1}\lVert f_k - f_{k+1} \rVert$$
            we have 
            \begin{equation*}
                \begin{aligned}
                    \lVert f_n - f_m \rVert &\leq \sum_{k=n}^{m-1} \epsilon_k^2
                    \leq \sum_{k=N}^{\infty} \epsilon_k^2    
                \end{aligned}
            \end{equation*}
            where $n , \ m \geq N$ and $\sum_{k=N}^{\infty} \epsilon_k^2 \leq \epsilon$.
            So rapidly cauchy indecates cauchy.\par 
            Let $\{f_n\}$ be a cauchy sequence. Let $\epsilon'_k = \frac{1}{2^k}$, then
        for each $\epsilon'_k$, we can find a $n(k) = N$, while $N$ is the number such that 
        $\forall n,m >N, \ \lVert f_n -f_m \rVert < \epsilon'_k$. It is clear that 
        $\lVert f_{n(k)} - f_{n(k+1)} \rVert \leq \epsilon'_k$. Since 
        $$\sum_{k=1}^{\infty}\epsilon_k = \sum_{k=1}^{\infty} \sqrt{\epsilon'_k} 
        = \sum_{k=1}^{\infty}\frac{1}{\sqrt{2}^k}$$
        is a converge series, $\{f_{n(k)}\}$ is rapidly cauchy.
        \end{proof}
        \begin{lemma}[Borel-Cantelli lemma]
            Let $\{E_k\}_{k=1^{\infty}}$ be a countable collection of measurable sets for 
        which $\sum_{k=1}^{\infty} m(E_k)<\infty$. Then almost all $x \in \RR$ be longs to finite 
        many of the $E_k$'s.
        \end{lemma}
        \begin{proof}
            For each $n$, by subadditivity of the measure,
            $$m(\bigcup_{k=n}^{\infty}E_K) \leq \sum_{k=n}^{\infty} m(E_K)$$
            Thus by continuity of the measure 
            $$m( \bigcap_{n=1}^{\infty}\bigcup_{k=n}^{\infty}E_K) = 
            \lim_{n \to \infty} m(\bigcup_{k=n}^{\infty}E_K) 
            \leq \lim_{n \to \infty} \sum_{k=n}^{\infty} m(E_K) 
            =0$$
            Therefore almost all $x \in \RR$ is not in 
            $\bigcap_{n=1}^{\infty}\bigcup_{k=n}^{\infty}E_K)$. So every $x$ at most belongs to fintie 
            many $E_k$s.
        \end{proof}
        \begin{theorem}[Riesz-Fischer theorem]
            Let $E$ be a measurable set and $1\leq p \leq \infty$. Then $L^p(E)$
        is a Banach space. Moreover, if $\{f_n\} \to f$ in $L^p(E)$, a subsequence 
        of $\{f_n\}$ converge to $f$ pointwisely a.e. on $E$ to $f$.
        \end{theorem}
        \begin{proof}
            To show that $L^p(E)$ is a Banach space, we need show that every cauchy
        sequences converge. Since proposition \ref{SubcauchySequence} and lemma
        \ref{RapidlyCauchy} have already been proved, it remains to show that every 
        rapidly cauchy sequence converges.\par 
            Let $\{f_k\} \subset L^p(E)$ be a rapidly cauchy sequence, and let 
        $\sum_{k=1}^{\infty} \epsilon_k$ be the converge positive series such that 
        $\lVert f_k-f_{k+1}\rVert \leq \epsilon^2_k$. \par 
            Suppose $p=\infty$. \par 
            Now suppose $p< \infty$. Notice that the condition of being rapidly cauchy indicates
            \begin{equation}\label{RieszFischer1}
                \int_E |f_{k+1} - f_k |^p \leq \epsilon_k^{2p}
            \end{equation}
            Given a fixed $k$. Since $|f_k(x) - f_{k+1}(x)| > \epsilon_k$ if and only if 
            $|f_k(x) - f_{k+1}(x)|^p > \epsilon_k^p$, so by Chebychev's inequality and \eqref{RieszFischer1},
        \begin{equation*}
            \begin{aligned}
                m(E\{|f_k(x) - f_{k+1}(x)|>\epsilon_k\}) &= m(E\{|f_k(x) - f_{k+1}(x)|^p>\epsilon_k^p\})\\
                &\leq \frac{1}{\epsilon_k^p} \int_E |f_k(x) - f_{k+1}(x)|^p\\
                &\leq \epsilon_k^{p}
            \end{aligned}
        \end{equation*}
            We can legally assume all $\epsilon_k< 1$, so $\sum_{k=1}^{\infty} \epsilon_{k}^p$
        is converge. Since $\sum_{k=1}^{\infty}m(E\{|f_k(x) - f_{k+1}(x)|>\epsilon_k\})
        \leq \sum_{k=1}^{\infty} \epsilon_{k}^p$, by the Borel-Cantelli lemma, for almost every 
        $x \in E$, or to say $ x \in E-E_0$ where $E_0$ is a measure zero subset, 
        $|f_k(x) - f^{k+1}(x)|< \epsilon_k$. \par Since the fact that 
        $\sum_{k=n}^{\infty}\epsilon_k$ converge and there exist $K(x)$ such that 
        $$|f_n(x) - f_{m}(x)| \leq \sum_{k=n}^{m-1}|f_k(x) - f_{k+1}(x)| \leq 
        \sum_{k=n}^{\infty}\epsilon_k$$
        forall $x \in E-E_0$ and $k> K(x)$
        $\{f_k(x)\}$ is cauchy sequence in $\RR$ for every $x \in E-E_0$, which also means 
        $\lim_{k \to \infty} f_k(x)$ exist for all $x \in E-E_0$. Let $f = \lim_{k \to \infty} f_k$.
        Then from \eqref{RieszFischer1},
            $$\int_E |f_n - f_{n+k}|^p \leq 
            \sum_{k=n}^{n+k-1}  |f_k -f_{k+1}|^p
            \leq \sum_{k=n}^{\infty} \epsilon^{2p}_k$$ 
        By Fatou's lemma,
        $$\int_E |f_n-f|^p leq \left[\sum_{k=n}^{\infty} \epsilon^{2}_k\right]^p$$
        Taking to the $\frac{1}{p}$ power on both side of the equation, we proved that the 
        rapidly cauchy sequence $\{f_k\}$ have a limit, Which also indicated that 
        $L^p(E)$ is a Banach space.\par
            Moreover, the upper construction of $f$ suggest every rapidly cauchy sequence 
        converge pointwisely a.e. to the function to which they converge in $L^p(E)$ in $\RR$. 
        Since a cauchy sequence $\{f_n\}$ has a rapidly cauchy subsequence, every cauchy sequence 
        has a subsequence whose pointwise limit exists and that limit function would be 
        the limit of $\{f_n\}$ in the sense of $L^p(E)$ converge.
        \end{proof}
        \begin{remark}
            One might be interested in what is the relationship between the $L^p$
        convergence and the pointwise convergence. For $E=[0,1]$, $1\leq p <\infty$, and each 
        $n \in \NN$, let $f_n= n^{\frac{1}{p}} \chi_{(0,1]}$. This function converge to 
        zero pointwisely on $\RR$, but converge to $1$ with respect to the $L^p$ norm.
        Neither $L^p$ or pointwise limit could directely determinded the other one, but 
        this could be true if some more condition is added.
        \end{remark}
        \begin{theorem}
            Let $E$ be a measurable set and $1 \leq p < \infty$. Suppose $\{f_n\} $ is a
        sequence of function in $L^p(E)$ and it converges to $f\in L^p(E)$ pointwisely a.e.
        on $E$. Then 
        $$\{f_n\} \to f \mbox{ in } L^p(E) \Leftrightarrow 
        \lim_{n \to \infty} \int_E |f_n|^p  = \int_E |f|^p $$
        \end{theorem}
        \begin{proof}
            We assuming that $\{f_n\} \to f$ pointwisely on all $E$. Suppose that 
        $\{f_n\} \to f$ in $L^p(E)$. Then Mincowski inequality implies 
        $$|\lVert f \rVert_p -\lVert f_n \rVert_p| \leq \lVert f - f_n \rVert_p$$
        So $\lim_{n \to \infty} \int_E f_n = \int_E f$.\par 
            Suppose the passage under the integral sign is available. Let $\psi(t) = |t|^p$. Since only $p \geq 1$
        is in our consideration, $psi$ is convex, thus 
        $$\psi(\frac{a+b}{2}) \leq \frac{\psi(a)+\psi(b)}{2}$$
        for all $a,\ b$, which means 
        $$0 \leq \frac{|a|^p + |b|^p}{2} - |\frac{a+b}{2}|^p$$
        for all $a , \ b$.
            Let $h_n = \frac{|f_n(x)|^p + |f(x)|^p}{2} - |\frac{f_n(x)+f(x)}{2}|^p$. Then $h_n$ is 
        non-negative and converge to $|f|^p$ pointwisely on $E$. By Fatou's lemma, and the passage under 
        integral sign, 
        \begin{equation*}
            \begin{aligned}
                \int_E |f|^p &\leq \liminf \int_E h_n\\
                &= \liminf [\frac{|f_n(x)|^p + |f(x)|^p}{2} - |\frac{f_n(x)+f(x)}{2}|^p] \\
                &\leq \int_E |f|^p - \limsup[\int_E |\frac{f_n(x)+f(x)}{2}|^p]\\
            \end{aligned}
        \end{equation*}
            From this we can infer that $\limsup[\int_E |\frac{f_n(x)+f(x)}{2}|^p] \leq 0$, which indicates
        that $\{f_n\} \to f$ in $L^p(E)$.  
        \end{proof}

    \section{Approximation and Separability}
        Recall the Littlewood second priciple, any measurable function can be approximated by 
    continuous function in some sense. The generalized idea is we can understand the structure of 
    a class of function by study whether it can be approximated by a class of better functions. 
    Now we have developed the convergence in $L^p$ spaces, so is natural to ask what kinds of functions
    can be used to approximate the functions in $L^p$ spaces with respect to the $L^P$ norm. 
        \begin{definition}[Dense]
            Let $X$ be a linear space normed by $\lVert \cdot \rVert$, and $U \subset V$ are subsets of $X$. 
        We say that $U$ is dense in $V$ if and only if given $f \in V$ $\epsilon > 0$, there exists a
        $g \in U$ such that $\lVert f -g \rVert < \epsilon$.
        \end{definition}
        It is not har to see $U$ is dense in $V$ if and only if for any $f \in V$, there 
    exist a sequence $\{f_n\} \in U$ such that $\lim_{n \to \infty} f_n = f$. The most familiar
    example is that $\QQ$ is dense in $\RR$, with the absolute function as norm.
        \begin{proposition}\label{DenseTransitivity}
            Let $X$ be a linear space normed by $\lVert \cdot \rVert$, and $U \subset V \subset W$ are subsets of $X$. 
        If $U$ is dense in $V$ and $V$ is dense in $W$, then $U$ is dense in $W$
        \end{proposition}
        The prove is quite naive so is left to the readers.
        \begin{proposition}
            Let $E$ be a measurable set and $1\leq p\leq \infty$. Then the subspace of simple functions
        in $L^p(E)$ is dense in $L^p(E)$.
        \end{proposition}
        \begin{proof}
            First we consider the case of $p= \infty$. Suppose $f$ is bounded on $E-E_0$ where 
        $E_0$ is a measure zero set. Notice that $L^p$ norm is an integral, so what value of $f$
        takes on a measure zero set can't change the norm. Assuming $f$ is bounded over $E$,
        simple approximate theorem says there is a sequence of simple function $\{\varphi_n\}$
        which converge to $f$ uniformly, which is equivalent to say that $\{\varphi_n\}$
        converge to $f$ respect to the $L^{\infty}$ norm. \par 
            Now consider the case of $1\leq p < \infty$. Again, by simple function approximation,
        there exists $\{\varphi_n\}$ which converge to $f$ pointwisely, and 
        $$|\varphi| \leq |f|$$ 
        which shows $\varphi \in L^p(E)$. In order to use the dominated convergence theorem,
        we first establish the inequality below.
        $$\int_E |\varphi_n - f|^p \leq \int_E |2f|^p \leq \int _E 2^p|f|^p$$ 
        this shows that $|\varphi_n - f|^p$ is bounded by $2^p|f|^p$, which is certain 
        integrable on $E$. Notice that $|\varphi_n - f|^p \to 0$ as $n \to \infty$,
        by applying the dominated convergence theorem,
        $$\lim_{n \to \infty}\lVert \varphi_n -f \rVert_p 
        = \lim_{n \to \infty} [\int_E |\varphi_n - f|^p]^{\frac{1}{p}} = 
        [\int_E \lim_{n \to \infty}|\varphi_n - f|^p]^{\frac{1}{p}} = 0 $$
        we proved that the simple function subspace of 
        $L^p(E)$ is dense in $L^p(E)$.
        \end{proof}
        \begin{proposition}
            Let $[a,b]$ be a colse bounded and $1\leq p\leq \infty$. Then the subspace of step functions
        in $L^p([a,b])$ is dense in $L^p(a,b)$
        \end{proposition}
        \begin{proof}
            By proposition \ref{DenseTransitivity}, we only need to show that the set of step functions
        is dense in the set of simple functions.
            Given any simple function, it can be represent as the linear combination of 
        characteristic functions. Notice that the linear combination of step functions 
        is still a step function. If for every characteristic function $\chi_{E_i}$, where 
        $E_i$ is measurable subset of $[a,b]$
        we can find a step function $\psi_i$ such that $\int_E|\psi_i -\chi_{E_i}|^p$ can be 
        arbritarly small, then 
        \begin{equation*}
            \begin{aligned}
            \int_{[a,b]} |\sum_{i=1}^n \psi_i - \sum_{i=1}^n \chi_{E_i}|^p &\leq 
            \int_{[a,b]} |\sum_{i=1}^n \psi_i -\chi_{E_i}|^p \\
            &\leq \int_{[a,b]} |n\max_{1\leq i \leq n}(\psi_i -\chi_{E_i})|^p \\
            &\leq n^p\int_{[a,b]} |\max_{1\leq i \leq n}(\psi_i -\chi_{E_i})|^p \\
            \end{aligned}
        \end{equation*}
        which shows simple functions can be approximate by step function. \par 
            The only thing need to show is for every $\epsilon >0$ 
        and characteristic function $\chi_{E}$, we can find a step function $\psi$ 
        such that $$\int_{[a,b]}|\psi -\chi_{E}|^p< \epsilon$$
        By the theorem which shows the Littlewood first priciple, there exist 
        a finite disjoint collection of intervals $\{I_n\}$, such that 
        $m(\bigcup_{n=1}^N I_n \Delta E) \leq \epsilon$. Then 
        $\chi_{\bigcup_{n=1}^N I_n}$ is a step function, and 
        $$\int_{[a,b]} |\chi_{\bigcup_{n=1}^N I_n} - \chi_E |^p= 
        \int_{[a,b]} |\chi_{\bigcup_{n=1}^N I_n \Delta E}| = m(\bigcup_{n=1}^N I_n \Delta E) \leq \epsilon $$
        Thus the proposition is proved.
        \end{proof}
        \begin{definition}[Separability]
            Let $X$ be a normed linear space. We say $X$ is separable if and only if 
        there is a countable subset dense in $X$.
        \end{definition}
        Again the easy example is $\RR$, since $\QQ$ is countable. 
        \begin{theorem}
            Let $E$ be a measurable set and $1\leq p < \infty$. Then the normed linear space is 
        separable.
        \end{theorem}
        \begin{proof}
            Let $[a,b]$ be closed bounded inteval. We use $S[a,b]$ to denote the collection
        of step functions on $[a,b]$. Let $S'[a,b]$ be the set of step functions $\psi$ which take only rational 
        value, and for $P=\{x_i\}_{i=1}^n$ is a partition of $[a,b]$ where $x_i$ are rationals, such 
        that $\psi$ is constant on $(x_i, x_{i+1})$. \par 
            \textbf{Claim:} $S'[a,b]$ is countable. $S'[a,b]$ is dense in $S[a,b]$.\\
            Notice that step functions only has finite different values, we suppose the finite dimension
        vector $V=\{v_i\} \subset Q$ where $f(x) = v_i$ if $x_i < x < x_{i-1}$. So each $\psi \in S'[a,b]$
        is determinded by $V$ and $P$. Thus it is not hard to see that there is a injective 
        map from the collection of all such $(V,P)$ to the collection of all sequence in $\QQ$. Since 
        the latter one is injective(it is the countable union of countable sets), we can conclude that 
        $S'[a,b]$ is countable. What value should $\psi$ talkes at the end points of each 
        subinterval which is determinded by $P$ is not important, since considering the convergence 
        in $L^p$, we only need to construct an sequence which convege a.e., while the set of end 
        points is definitely of measure zero.\par 
            Since $\QQ$ is dense in $\RR$, for every $r \in \RR$ we can find a sequence $\{q_n\}$ which    
        converge  to $r$. Given $f \in S[a,b]$, $f(x) = \sum_{i=1}^n r_i \chi_{[x_i, x_{i+1}]}$.
        Let $\{q_{ij}\}$ be the sequence converge to $r_i$ increasingly and $\{y_{ij}\} \to x_i$. 
        Then $\{\psi_j\}$ is a sequence of functionsin $S'[a,b]$ which converge pointwisely to $f$ a.e. on $E$, 
        where $\psi_j(x) = q_{ij}$ for $x_i< x <x_{i+1}$. Clearly $f$ is bounded by some $M$, and therefore
        all but finite $\psi_j$ are bounded by $M+1$. Ignoring the finite terms, $|\psi_j|^p \to |f|^p$ 
        and all theses functions are bounded by $(M+1)^p$, so $|f-\psi_j|^p$ is bounded by $2^p(M+1)^p$. 
        Using the bounde converge theorem, and by the face $|f-\psi_j|^p \to 0$ a.e.,
        \begin{equation*}
            \begin{aligned}
                \lim_{j \to \infty}\lVert f - \psi_j \rVert &= \lim_{n to \infty} 
                [\int_a^b |f-\psi_j|^p]^{\frac{1}{p}} \\
                &=  [\int_a^b \lim_{n to \infty} |f-\psi_j|^p]^{\frac{1}{p}} 
                &= 0
            \end{aligned}
        \end{equation*}
            Thus we proved that $S'[a,b]$ is dense in $S[a,b]$. So $S'[a,b]$ is also dense in 
        $L^p([a,b])$.\par 
            To prove that $S'(E)$ is dense in $L^p(E)$ where $E$ is any measurable set, let 
        $\mathcal{F}_n$ be the set $S'[-n, n]$ which is countable. Let 
        $$\mathcal{F} = \bigcup_{n=1}^{\infty} \mathcal{F}_n $$ 
        which is also countable. By monotone convergence theorem,
        $$\lim_{n \to \infty} \int_{-n}^n |f|^p = \int_{\RR} |f|^p$$
        Thus $\mathcal{F}$ is dense in $L^p(\RR)$. For any measurable subset $E$,
        consider the ristirction of the functions in $\mathcal{F}$, they form the countable 
        set which dense in $E$. Therefor $L^p(E)$ is separable.
        \end{proof}
        %Why l\infty space is not separable?
        Since we can use continuous function of compact support to approximate step 
    functions, thus we have $C_c(E)$ dense in $L^p(E)$.
        
\chapter{Riesz Representation}
    \section{Linear Functional and Dual Space}
        \begin{definition}[linear functional]
            
        \end{definition}
    
        \begin{definition}[Bounded linear functional]
            
        \end{definition}

        \begin{proposition}
            Let $X$ be a normed linear space, and $T$ be a 
        bounded linear function on $X$. Then
        $$\lVert T \rVert_* = \sup\{T(f): \lVert f \rVert leq 1 , \ f \in X \} $$
        \end{proposition}
        \begin{proof}
            Let $s:=\sup\{T(f): \lVert f \rVert leq 1 , \ f \in X \} $. Since for 
        any $f \in X$, $\lVert \frac{f}{\lVert f \rVert}\rVert \leq 1$,
        $$\frac{T(f)}{\lVert f \rVert} = T(\frac{f}{\lVert f \rVert}) \leq s $$
        Thus 
        $$T(f) \leq s \lVert f \rVert \Rightarrow s \geq \lVert T \rVert_*$$
        Suppose $\lVert T \rVert_* < s$. By the property of the supremum,
        there exist a $\lVert f\rVert \leq 1$ such that
        $$T(f) > \lVert T \rVert_* \Rightarrow T(f) > \lVert T \rVert_* \lVert f\rVert$$
        which is contradict to the definition of $\lVert T \rVert_*$.
        \end{proof}
\chapter{Appendix 1: Non-measurable Sets, Non-measurable Functions}
    For the purpose to illustrate the theory in a more clear way, we didn't talk about the existence of non-measurable sets, non-measurable functions and many 
other counter examples. Counter-examples are important. Without showing there existence, our theory is meaningless(mabey all the sets are 
measurable). In fact, many of them are quite hard to construct, which is the reason why we didn't mention them in the main body. 
\end{document} 