\documentclass[lang=en, 12pt]{elegantbook}


\title{{\Huge{\textbf{Real Analysis}}}}
\author{Alex Luo}
\date{\today}
\cover{cover}

\newcommand{\vt}[1]{\textit{\textbf{#1}}}
\newcommand{\RR}{\mathbb{R}}
\newcommand{\QQ}{\mathbb{Q}}
\newcommand{\ZZ}{\mathbb{Z}}
\newcommand{\NN}{\mathbb{N}}
\newcommand{\lmf}[1]{$\mathcal{M} \mathfrak{F} \{#1\}$}

\begin{document}
\frontmatter
\maketitle
\tableofcontents

\mainmatter
    \chapter{Why We Need Lebesgue Integral}
        In the previous sections, we have already developed the theory of Riemann integral. But Riemann integral dose not 
    close after taking limit. If ${f_n}$ is a sequence of Riemann integrable functions, and it's limit function may not be Riemann integrable.   
    \begin{example}
        Let ${f_n}$ be a sequence of integrable function over $[0,1]$ and $f$ be its limit function. 
    \begin{enumerate}
        \item We already know that if $f_n$ converge uniformly, then $f$ must be integrable and 
        \begin{equation}
        \lim_{n \to \infty} \int_a^b f_n = \int_a^b \lim_{n \to \infty}f_n = \int_a^b f \label{limint}
        \end{equation}
        by the theory of uniform Convergence.
        \item ${f_n}$ pointwise converge to $f$ but $f$ isn't integrable.
        Let ${r_k}$ be the sequence which consits of all the rationals( $\mathbb{Q}$ is countable, so this sequence exists).
        Let $f_n(x) = 1 \mbox{ if } x = r_k $ for some $k < n$, and for other cases, $f_n(x) = 0$. For any $n$, $f_n$ has only 
        finite discontinuity, thus it is integrable. ${f_n}$ converge to the dirichlet functions, which we know isn't integrable.
        \item ${f_n}$ pointwise converge to $f$ and $f$ is integrable. Let $f_n = x^n$, then ${f_n}$ converge to 
        $f: f(x)= 1 $ if $ x = 1 ; f(x) = 0 $ if $ x \in [0,1)$. The integral exist and satisfies (\ref*{limint}).
    \end{enumerate} 
    \end{example} \par
        Riemann integral functions is in the exactly situation of rational number, if you take limit, you probably got something
    else. An integral like real number, with the completeness,  is desired.
    \par
        One of the motivation of the developing the theory of integration is to calculate the area, and the method we use basicly is to cut
    a whole thing into pieces. For Riemann integral, to calculate the area under a function, one take a partition the domain, and build up 
    a "step function". If the step function converge to the original function as partition gets finer, we say the integral exist.\par
        For Lebesgue integral, we also take partition, but instead of on it's domain, we take it on the range. Let $f:E\to \mathbb{R}$ be a real value function
    $P_{M_n} = \{\frac{2iM}{2^n} -M : i \in [0,2^n] \cap \mathbb{Z} \}$ be a partition of $f(E) \in \mathbb{R}$. The integral under this partition will be 
    look like 
    \begin{equation}
    \int_E{f(x)} = \lim_{n \to \infty} \sum_{i=1}^{2^n} [(\frac{2iM}{2^n} - M )m(E_i)] = \lim_{n \to \infty} \sum_{i=1}^{2^n} [(\frac{2(i+1)M}{2^n} - M )m(E_i)] \label{li1} 
    \end{equation}
    where $E_i=\{x:f(x) \in [\frac{2iM}{2^n} -M,\frac{2(i+1)M}{2^n}-M]\}$, $m(E_i)$ is it's lenth.\par
        One may find as $n \to \infty$, the last equal sign in \eqref{li1} is true no matter what the intergrand is, but this doesn't means all the functions
    are Lebesgue integrable. Because without defining the lenth of a set(or volume, for higher dimension), the equation above is meaningless.
    So, in order to develop the theory of Lebesgue integral, we need to find a set function $m$, which is called Lebesgue measure, mapping 
    a general set into $[0,+\infty]$. We also hope $m$, has the following properties(in order to fit in our notion of lenth):
        \begin{enumerate}
            \item (Monotonicity) If $E \subset F$, $m(E) \leq m(F)$. This is inherited from the intuitive notion of volume.
            \item $m([a,b]) = b - a$. The lenth of integral is agree with its measure.
            \item (Countable additivity) Let ${E_i}$ be a disjoint collection of set, then $$m(\bigcap_{i=1}^{\infty} E_i) = \sum_{i=1}^{\infty}m(E_i)$$
            which enable us doing integral.
        \end{enumerate} \par
        We will first discuss Lebesgue's theory on $\RR$, and for most sets that occurs will be subsets of $\RR$. In next section, we will focus 
    measure theory, which solves the problems above.

    \chapter{Lebesgue Measure}
        To define a such measure function(the Lebesgue measure) we have discussed in the previous section, we need to first define a function
    which is called \emph{Lebesgue outer measure}. This is because countable additivity is really a rigorous property. Outer measure is defined
    for all set, but is doesn't have the countable additivity. The true Lebesgue measure only is defined on a certain collection of sets, which
    we called measurable set. Basicly, we ristrict the Lebesgue outer measure on the measurable sets to obtain the Lebesgue measure. In the 
    general measure theory, there are many measures, and Lebesgue measure is just one of them. Since we will only talk about Lebesgue measure,
    for convenience, we will just call the Lebesgue as measure(also the concepts measurable).
        \section{Lebesgue Outer Measure}
            \begin{definition}[Lebesgue outer measure]
                Given a set $A$, let $\{I_n\}$ be a countable open interval cover of $A$, namely $A \subset \bigcap_{n=1}^{\infty}I_n$ and
            ${I_n}$ is a countable collection of open intervals. Let $m^*(A) = \inf\{\sum_{n=1}^{\infty}I_n:\{I_n\} \mbox{is a open interval cover of }A\}$,  
            We call the function $m^*$ Lebesgue outer measure. 
            \end{definition} % is {{I_k}: union I_k contains A} a set ?
            Since for every set such collection of number(can be infinity) is nonempty, so the outer measure of every set on $\RR$ is well defined. 
            \begin{proposition}[Properties of Lebesgue outer measure]
                Let $A,B,\{A_n\}$ be any sets and countable collection of sets. The following properties is satisfied.
            \begin{enumerate}
                \item Positivity: $m^*(A) \geq 0$.
                \item Monotonicity: $B \subset A \Rightarrow m^{*}(A) \geq m^{*}(b) $
                \item Countable Subadditivity: $\sum_{n=1}^{\infty} m^*(A_n) \geq m^*(\bigcap_{n=1}^{\infty}A_n)$
                \item Translation Invariance: For any number $y$, $m^*(A+y) = m^*(A)$ where $A+y = {a+y:a\in A}$ 
            \end{enumerate} 

            \end{proposition}
            \begin{proof}
                Positivity: Since for every open interval cover $\{I_n\}$, $\sum_{n=1}^{\infty}|I_n|$ is a positive series, $m^*(A) \geq 0$. \par    
                Monotonicity: If $B \subset A$, every open interval cover of $A$ is also that of $B$. Thus by the property of $\inf$,
                $m^*(B) \leq m^*(A)$. \par
                Countable subadditivity: Suppose $\sum_{n=1}^{\infty}m^*(A_n) < +\infty$(the $+\infty$ case is trivial). For each $\{A_n\}$,
            let $\{I_{n_k}\}$ be its open interval cover, and we can choose a $\{I_{n_k}\}$ such that 
                $$\sum_{n=1}^{\infty}\{I_{n_k}\} - m^*(A_n) \leq \frac{\epsilon}{2^n}$$ \par
                Then we know 
                $$\bigcap_{n=1}^{\infty}A_n \subset \bigcap_{n=1}^{\infty}\{I_{n_k}\},\ \sum_{n,k=1}^{\infty}|I_{n_k}| \leq \sum_{n=1}^{\infty}m^*(A_n) + \epsilon$$
                Let $\epsilon \to 0$, the desired proposition is proved.\par 
                (This is something called coset. You may have run into this concept in algebra, but if you haven't, never mind it)\par
                
                Translation invariance: For every open interval cover $\{I_n\}$ of $A$, $\{I_n+y\}$ is a open interval cover of $A+y$, and at the same time, 
                for every open cover $\{I_n\}$ of $A+y$, $\{I_n+y\}$ is a open cover of $A$. Since invervals is translation invariant,
                $$\sum_{n=1}^{\infty}|I_n| =\sum_{n=1}^{\infty}I_n+y$$ 
                Given any countable open interval cover $\{I_n\}$ of $A$,
                \begin{equation}
                    m^*(A+y) \leq \sum_{n=1}^{\infty}I_n+y = \sum_{n=1}^{\infty}I_n
                \end{equation}
                Then we can see that $m^*(A+y) \leq  m^*(A+y)$, since the former one is a lower bound and the latter one is the greatest lower bound.
                The inequality in the other way can be proved in the came way, then we conclude that $m^*(A)=m^*(A+y)$.  
            \end{proof}
            \begin{example}[The outer measure of $\varnothing$]
                Since $\varnothing \subset E$ for any set $E$, let $I_{n_k} = (0,\frac{1}{n 2^k})$, then 
            the collection $\{I_{n_k}\}_{k=1}^{\infty}$ is certain a open interval cover of $\varnothing$ and 
            $\sum_{k=1}^{\infty}|I_{n_k}|= \frac{1}{n}$. Let $n\to \infty$, then $\sum_{k=1}^{\infty}|I_{n_k}| \to 0$. 
            Thus $m^*(\varnothing) \leq 0$(remember $m^*$ of a set is a lower bound). By the positivity of outer measure, $m^*(\varnothing) = 0$
            \end{example}
            \begin{example}[The outer measure of countable set is $0$]
                Give any countable set $ A = \{a_k\}$, let $I_{n_k} = (a_k-\frac{1}{n2^{k+1}},a_k+\frac{1}{n2^{k+1}})$. For each $n$, $\{I_{n_k}\}$
            is a open interval cover of $A$, and 
                $$\sum_{k=1}^{\infty}|I_{n_k}|=\sum_{k=1}^{\infty}\frac{1}{n2^k}=\frac{1}{n}$$ \par
                Let $n \to \infty$, we arrive at the conclusion: $m^*(A) = 0$.
            \end{example}
            \begin{proposition}[The outer measure of intervals]
                The outer measure of an interval is its lenth. Let $I = [a,b]$, then $$m^*(I) = b-a $$
                The similar statements for $[a,b), (a,b], (a,b)$ are also true.\par
                
            \end{proposition}
            \begin{proof}
                For every close interval $[a,b]$, one can find an open interval $(a-\epsilon,b+\epsilon)$ cover it, which has the lenth 
            $b-a+2\epsilon$. Since episilon can be chosen freely, $m^*([a,b]) \leq b-a$. \par
                It remains to show $m^*([a,b]) \geq b-a$, which is
            equivalence to show for all countable open interval covers $\{I_n\}$, 
            \begin{equation}
                \sum_{n=1}^{\infty}I_n \geq b-a \label{tomoi1}
            \end{equation}
                Because we are dealing a
            close interval on $\RR$, which is a compat set, Heine-Borel theorem tells us it has a finite subcover for each open cover. So 
            \begin{equation}
                \sum_{n=1}^{N}I_n \geq b-a \label{tomoi2}
            \end{equation}
            will indicate \eqref{tomoi1}. \par
                Since $a$ is in $\bigcap_{n=1}^{N} I_n$, there exist a $I_{n_1}$ contains $a$. If $I_{n_1}$ contains b, $|I_{n_1}|>b-a$;
            if not, $I_{n_1} = (c,d) \ , \ c<a<d<b$, then $d$ is in $[a,b]$, which means exist  $I_{n_2}$ such that $d \in I_{n_2}$; if
            $b \in I_{n_2}$, then $|I_{n_1}|+|I_{n_2}|>b-a$; if not, $I_{2} = (f,g) \ , \ f<d,g<b$, \dots There must be a $I_{n_i}$ such that 
            $b \in I_{n_i}$, and since we have only finite $I_n$, so this process will be eventually ended. After doing this for enough times,
            one will find out that 
            $$\sum_{i=1}{N}|I_{n_i}|>b-a$$ \par
                For other intervals, since they are subsets of closed interval, by monotonicity, we have $m^*(I)\leq b-a$. By the subadditivity
            and our previous conculsion on the outer measure of countable set, $$m^*([a,b])=m^*((a,b)\cap \{a,b\})\leq m^*((a,b))+m^*(\{a,b\})=b-a$$
            The proof for $[a,b)$ and $(a,b]$ is simialr.
            \end{proof}
        \section{Lebesgue Measure}
            \begin{definition}[Measurable sets and Lebesgue measure]
                Let $E$ be a set, if for any set $A$, 
                \begin{equation}
                    m^*(A) = m^*(A \cap E) + m^* (A \cap E^c) \label{ms}
                \end{equation}
                then we say $E$ is a measuralbe set. The collection of all the measurable sets is denoted $\mathcal{M}$.\par
                If $E$ is a measurable set, we define its Lebesgue measure $m(E) := m^*(E)$. 
            \end{definition}
            \begin{example}[The sets with outer measure zero are measurable]
                First, notice that, if one wants to check whether a set $E$ is measurable, the only thing is to show is 
            $$m^*(A) \geq m^*(A \cap E) + m^* (A \cap E^c)$$
            since the other inequality is guaranteened by the subadditivity.\par 
                For the set $E$ whose outer measure is zero, and any set $A$, by monotonicity of outer measure, we have 
                \begin{equation*}
                    \begin{aligned}
                        &m^*(A \cap E = 0) \ , \ \ m^*(A \cap E^c) \leq m^*(A) \\
                        \Rightarrow &m^*(A) \geq  m^*(A \cap E) + m^* (A \cap E^c)  
                    \end{aligned}
                \end{equation*} 
                And together the subadditivity of outer measure, we have 
                $$m^*(A) = m^*(A \cap E) + m^* (A \cap E^c)$$
                which shows that $E$ is measurable.
            \end{example}
            In our expectation of the measure function, it should have the countable additivity. So we know that the set must satisfied 
        \eqref{ms}, because $(A \cap E) \cap (A \cap E^c) = \varnothing$ and $(A \cap E) \cup (A \cap E^c) = A$, which is a special case 
        of finite additivity. The Lebesgue measure define above is just the ristriction of $m^*$ on the collection of measurable sets. 
        At last, we will show that the measure defined above have all the good properties we want. Before that, we first would like to 
        prove some properties of the measurable sets.
            \begin{proposition}[Properties of measurable sets] 
                Let $E_1, E_2 \in \mathcal{M}$,and $\{E_i\} \subset \mathcal{M}$\par
                \begin{enumerate}
                    \item $\varnothing \in \mathcal{M}$
                    \item $E_1^c \in \mathcal{M}$.
                    \item $E_1 \cap E_2, \ E_1 \cup E_2, \ E_1/E_2 \in \mathcal{M}$. From this one can easily derive the closedness of finite 
                intersection and union.
                    \item If for any $i \neq j$, $E_i \cap E_j = \varnothing$, then 
                    $$m(A \cap \bigcup_{i=1}^{N}E_i) = \sum_{i=1}^{N}m(A \cap E_i)$$
                    The finite additivity among measurable sets is a special case of this proposition, taking $A=\mathbb{R}$. 
                    \item $\bigcap_{i=1}^{\infty} E_i, \ \bigcup_{i=1}^{\infty} E_i \in \mathcal{M}$
                \end{enumerate}\par
                
            \end{proposition}
            \begin{proof}
                1. This is true because we have show that measure zero sets are measurable.\par
                2. Since $(E^c)^c = E$, so 
                \begin{equation*}
                    \begin{aligned}
                        m^*(A) &= m^*(A\cap E) + m^*(A \cap E^c) \\
                               &= m^*(A\cap(E^c)^c) + m^*(A \cap E^c)
                    \end{aligned}
                \end{equation*}
                which shows that $E^c$ is measurable.\par
                3. Here we first prove the union of measurable sets is measurable. Since $E_1, \ E_2 \in \mathcal{M}$, for any set $A$
                \begin{equation*}
                    \begin{aligned}
                        m^*(A) &= m^*(A \cap E_1) + m^*(A \cap E_1^c)\\
                               &= m^*(A \cap E_1) + m^*((A \cap E_1^c) \cap E_2) + m^*((A \cap E_1^c) \cap E_2^c)  
                    \end{aligned}
                \end{equation*}
                By basic set identities,
                \begin{equation*}
                    \begin{aligned}
                        (A \cap E_1^c) \cap E_2^c = A \cap (E_1 \cup E_2) ^c\\
                        [A \cap E_1] \cup [(A \cap E_1^c) \cap E_2] = A \cap (E_1 \cup E_2)
                    \end{aligned}
                \end{equation*}
                and by the subadditivity of outer measure,
                \begin{equation*}
                    \begin{aligned}
                        m^*(A) &\geq m^*([A \cap E_1] \cup [(A \cap E_1^c) \cap E_2]) + m^*((A \cap E_1^c) \cap E_2^c)\\
                               &= m^*(A \cap (E_1 \cup E_2)) + m^*(A \cap (E_1 \cup E_2)^c)
                    \end{aligned}
                \end{equation*}
                is true. Thus the union of measurable set is measurable.\par
                4. The proof proceeds by induction on $N$. It is clear true for $N = 1$. Suppose it is true for $N-1$. By the measurability
            of $E_N$, and the fact that $\{E_n\}$ is disjoint sequence of sets, 
            \begin{equation*}
                \begin{aligned}
                m^*(A \cap (\bigcup_{n=1}^{N}E_n)) &= m^*( [A \cap (\bigcup_{n=1}^{N}E_n)] \cap E_N) + m^*([A \cap (\bigcup_{n=1}^{N}E_n)] \cap E_N^c)\\
                &= m^*(A \cap E_N) + m^*(A \cap \bigcup_{n=1}^{N-1}E_n) \\
                &= \sum_{n=1}^{N}E_n
                \end{aligned}
            \end{equation*}\par
                5. Let $E$ be the union of countable collection of measurable set, say $\{A_k\}$. By defining a new sequence of sets,
            $$ E_1 = A_1, \ \ \ E_n = A_n/E_{n-1} $$ 
            one can express $E$ as a union of disjoint sequence of measurable sets. Let $A$ be any set. Observing that $E^c \subset (\bigcup_{k=1}^N E_k)^c$
                \begin{equation*}
                    \begin{aligned}
                        m^*(A) &\geq m^*(A \cap \bigcup_{k=1}^N E_k) + m^*(A\cap (\bigcup_{k=1}^N E_k)^c\\
                               &\geq m^*(A\cap \bigcup_{k=1}^N E_k) + m^*(A\cap E^c)\\
                               &= \sum_{k=1}^N m^*(A\cap E_k) +m^*(A\cap E^c)
                    \end{aligned}
                \end{equation*}
                Since the relation is regardless of $N$
                \begin{equation*}
                    \begin{aligned}
                        m^*(A) &\geq \sum_{k=1}^{\infty} m^*(A\cap E_k) +m^*(A\cap E^c)\\
                            &\geq m^*(A\cap E) + m^*(A\cap E^c) 
                    \end{aligned}
                \end{equation*}
                Then the desired conclusion is proved.
            \end{proof}
            \begin{proposition}[Intervals are measurable]
                For every $a, \ b$ in $\mathbb{R}^*$, $[a,b], \ (a,b), \ [a,b), \ (a,b]$ are measurable.\par
                
            \end{proposition}
            \begin{proof}
                First, we show that $(a,\infty)$ is measurable. Give any set $A$, we need to show that
                $$m^*(A) \geq m^*(A\cap (a,+\infty)) + m^*(A \cap (-\infty,a))$$
            since $m^*(A)$ is a infimum, it is surffice to show
                \begin{equation}
                    m^*(A\cap (a,+\infty)) + m^*(A \cap (-\infty,a)) \leq \sum_{n=1}^{\infty} |I_{n}| \label{iim}
                \end{equation}
            where $I_n$ is a open interval cover of $A$.
                For each $n$, we define $J_n = I_n \cap (a, +\infty)$, $k_n = I_n \cap (-\infty, a)$. Then 
                \begin{equation*}
                    \begin{aligned}
                        |I_n| &= |J_n| + |K_n|\\
                        \sum_{n=1}^{\infty}|I_n| &= \sum_{n=1}^{\infty} |J_n| + |K_n| = \sum_{n=1}^{\infty} |J_n| + \sum_{n=1}^{\infty}|K_n|
                    \end{aligned}
                \end{equation*}
            Observing that $\sum_{n=1}^{\infty} |J_n|$ and $\sum_{n=1}^{\infty}|K_n|$ are open interval covers of $A \cap (a, +\infty)$ and $A \cap (-\infty, a)$,
        So \eqref{iim} is varified, then we proved that $(a,+\infty)$ is measurable. \par
            Since $\mathcal{M}$ is closed under complement, $(-\infty, a]$ is measurable; Since $\mathcal{M}$ is closed under countable,
        union, 
        $$(-\infty, a) = \bigcup_{n=1}^{\infty} (-\infty, a-\frac{1}{n}]$$ 
        is measurable, and also its complement $[a, \infty)$. Let $a \leq b$. 
        $$[a, b] = [a, \infty) \cap (-\infty, b)$$
        is measurable. Other intervals are measurable can be shown in the same manner.
        \end{proof}
            \begin{definition}[Algebra and $\sigma$ algebra]
                Let $A$ be a collection of sets, if $\forall F, \ E \in A$ 
                \begin{enumerate}
                    \item $\varnothing \in \mathcal{M}$
                    \item $F^c \in \mathcal{M}$.
                    \item $F \cap E \in \mathcal{M}$
                \end{enumerate}
                then, we say $A$ is an algebra.\par
                If a algebra $A$ is closed under countable intersection, namely if a countable collection of sets ${E_n} \in A$, then $\bigcap_{n=1}^{\infty}E_n \in A$,
            we call $A$ a $\sigma$ algebra.\par
                Given a collection of set $A$, we call the smallest algebra contains $A$(if you remove any part of it, it won't be an algebra or it won't 
            contains $A$ anymore) the algebra generate by $A$.                  
            \end{definition}
            By Demorgan identity, we can see that a algebra is also close under intersection and difference, and a $\sigma$ algebra is close under 
        countable intersection. The collection of all measurable set is a algebra. The collection of set with this particularstructure is 
        quite useful in analysis. One need to be careful not to mix up this concept with algebra in Algebra. 
            \begin{example}[Borel sets]
                We call the $\sigma$ algebra generated by the collection of open sets in $\RR$ Borel sets. Before Lebesgue, Borel had already built
            up a measure theory, but not in the purpose of defining an integral. We use $\mathcal{B}$ to denote the collection of Borel sets. 
            \end{example}
            \begin{definition}[$F_{\sigma}$ set and $G_{\delta}$ set]
                For a countable collection of close sets $\{F_n\}$, we say $F = \bigcup_{n=1}^{\infty}F_n$ is a $F_{\sigma}$ set; for any countable collection of 
            open sets $\{G_n\}$, we call the set $G = \bigcap_{n=1}^{\infty}$ a $G_{\delta}$ set. Also you can have an $F_{\sigma \delta}$ set
            and $G_{\delta \sigma}$ set, which are intersection of countable $F_{\sigma}$ sets and union of countable $G_{\delta}$ sets.   
            \end{definition}
            \begin{proposition}[The properties of Borel sets]
                \begin{enumerate}
                    \item Open sets, close sets, $G_{\delta}$ sets, $F_{\sigma}$ sets \dots, in $\mathbb{R}$ are all Borel sets.
                    \item $\mathcal{B} \subset \mathcal{M}$, the collection of Borel sets is contained in the collection of measuralbe set,
                    namely, every Borel sets are measuralbe.
                    %\item Any strickly monotonic increasing function on $\mathbb{R}$ maps Borel sets to Borel sets.
                \end{enumerate}
                
            \end{proposition}
            \begin{proof}\par
                1. By definition, all the open sets should be Borel set, and so are close sets, which are complement of open sets.
            $G_{\delta}, \ F_{\sigma}$ are countable intersection, countable union of Borel sets, so they are also Borel sets. \par
                2. By open set construction theorem, every open set $O = \bigcup_{n=1}^{\infty} I_n$, where $\{I_n\}$ is a countable collection
            of open intervals. Since $I_n$ is measurable, so is open sets. Because Borel set is the smallest $\sigma$ algebra contains
            all the open sets, thus $\mathcal{B} \subset \mathcal{M}$.  
            \end{proof}
            We end this section with the most important property of Lebesgue measure: countable additivity.
            \begin{proposition}
                Given a countable collection of disjoint measurable sets $\{E_n\}$, 
                \begin{equation}
                    m^*(\bigcup_{n=1}^{\infty} E_n) = \sum_{n=1}^{\infty} E_n \label{ca}
                \end{equation}
                
            \end{proposition}
            \begin{proof}
                We know that $\bigcup_{n=1}^{\infty} E_n$ is measurable. Since outer measure has subadditivity, it remains to show 
                \begin{equation}
                    m^*(\bigcup_{n=1}^{\infty} E_n) \geq \sum_{n=1}^{\infty} E_n \label{caa}
                \end{equation}
                For each number $N$, by finite additivity of measurable sets, 
                $$ m^*(\bigcup_{n=1}^{N} E_n) = \sum_{n=1}{N} E_n$$
                And we have $\bigcup_{n=1}^{N} E_n \subset \bigcup_{n=1}^{\infty} E_n$. By monotonicity of outer measure,
                \begin{equation*}
                    \begin{aligned}
                        m^*(\bigcup_{n=1}^{\infty} E_n) &\geq m^*(\bigcup_{n=1}^{N} E_n)\\
                        &= \sum_{n=1}{N} E_n
                    \end{aligned}
                \end{equation*}
                Let $N \to \infty$, then the proposition is proved.
            \end{proof}
        \section{Outer and Inner Approximation of Measurable Sets}
            One can find out a easy fact from the definition of measurable set, that is for a finite measuralbe set $E \subset B$ 
        we have the excision property
            \begin{equation*}
                m^*(B/E) = m^*(B) - m^*(E)
            \end{equation*}
            which is a quite useful obsevation. 

            \begin{theorem}[Use open sets, close sets, $G_{\delta}$, $F_{\sigma}$ to approximate measurable set]
                Let $E$ be a subset of $\RR$. The following assertions are equivalent to measurability of $E$.
                \begin{enumerate}
                    \item $\forall \epsilon > 0$, $\exists O \supset E$ is a open set, such that $m^*(O/E) \leq \epsilon$.
                    \item $\exists G \supset E$ is a $G_{\delta}$ set, such that $m^*(G/E) = 0$
                    \item $\forall \epsilon > 0$, $\exists C \subset E$ is a close set, such that $m^*(E/C) \leq \epsilon$.
                    \item $\exists F \subset E$ is a $F_{\sigma}$ set, such that $m^*(E/F) = 0$
                \end{enumerate} 
            \end{theorem}
            \begin{proof}\par
                1. Suppose $E$ is measurable. Since the measure of $E$ is a supremum, so given $\epsilon > 0$, there exists a open interval 
            cover, such that 
                \begin{equation}
                    m^*(E) + \epsilon \geq \sum_{n=1}^{\infty} |I_n| \geq m^*(E) 
                \end{equation}
                Let $O = \bigcup_{n=1}^{\infty} I_n$, then $O$ is a open set and $O \supset E$, it remains to show that 
                $$m^*(O/E) \leq \epsilon$$
                By definition of outer measure 
                $$m^*(O) \leq \sum_{n=1}^{\infty} |I_n| \leq m^*(E) + \epsilon$$
                which suggest that 
                $$m^*(O)-m^*(E) \leq \epsilon$$\par
                From the observation we made in the beginning of this section, we know the proposition will be true if $m^*(E) < +\infty$.
            For the case that $m^*(E) = \infty$, oen can express $E$ as countable disjoint collection of measurable sets of finite measure, say $\{E_K\}$.
            For instance, one construction of $\{E_k\}$ could be 
            $$E_{k} = [\frac{k}{2}-1, \frac{k}{2}) \cap E \mbox{ while } k=2z $$
            $$E_{k} = [-\frac{k+1}{2}, -\frac{k+1}{2}+1) \cap E \mbox{ while } k=2z+1$$
            Then we can find a sequence of open sets $\{O_k\}$ such that $O_k \supset E_k$ and $m^*(O_k/E_k) \leq \frac{\epsilon}{2^k}$.
            Since $O/E$ is a subset of $\bigcup_{k=1}^{\infty} O_k/E_k$, by monotonicity and subadditivity,
            $$m^*(O/E) \leq m^*(\bigcup_{k=1}^{\infty} O_k/E_k) \leq \sum_{k=1}^{\infty} m^*(O_k/E_k) \leq \epsilon$$
            Thus part 1 is proved.\par
                2. By 1., for every $n$, we can find a open set $O_n \supset E$ such that $m^*(O/E) \leq \frac{1}{n}$. Let 
            $G = \bigcap_{n=1}^{\infty} O_n$. It is clear that $G \supset E$, and $G$ is a $G_{\delta}$ set. Since $G/E \subset O_n/E$
            for every $n$,
            \begin{equation*}
                \begin{aligned}
                    m*(G/E) &\leq m^*(O_n /E)\\
                            &\leq \frac{1}{n}
                \end{aligned}
            \end{equation*}
            Let $n \to \infty$, we have $m^*(G/E) = 0$.\par
                Suppose 2. holds. We know $G$ and $G/E$ are measurable (one is $G_{\delta}$, one has out measure zero), so $E$ must be 
            measurable.\par
            3. $E^c$ is clearly measurable, so for every $\epsilon > 0$, we can find a $O \supset E^c$ which is a open set such that 
        $m^*(O/E^c) \leq \epsilon$. It is not hard to see that $O^c \subset E$ is a close set, and it remains to show that $m^*(E/O^c) \leq \epsilon$.
        $$E/O^c = E \cap O = O/E^c$$ Thus part 3 is proved.\par
            4. For every $n$, we can find a $C_n \subset E$ which is a close set, such that $m^*(E/C_n) \leq \frac{1}{n}$. Let 
        $F=\bigcup_{n=1}^{\infty} C_n$, then F is a $F_{\sigma}$ set in $E$. Since $E/\bigcup_{n=1}^{\infty}C_n$ is a subset of $E/C_n$,
        $m^*(E/F)=0$.\par
            Suppose 4. holds. Then $F$ and $E/F$ are measurable, so is $E$.
            \end{proof}
            This theorem indicated that every measurable set is nearly a simple set. Proving the next theorem, we push this idea futher.
            \begin{theorem}[Measurable sets are nearly finite disjoint union of open intervals]\label{LittlewoodFirstPrinciple}
                Given a measuralbe set $E$ of finite measure, for each $\epsilon > 0$, there is a finite collection of disjoint open intervals
            $\{I_n\}_{n=1}^N$, for which if $O = \bigcup_{n=1}^{N} I_n$,
                \begin{equation}
                    m^*(E/O)+m^*(O/E) < \epsilon
                \end{equation} \par
            \end{theorem}
            \begin{proof}
                $\forall \epsilon > 0$, $\exists O'$ is a open set such that $O' \supset E$ and 
                \begin{equation*}
                    m^*(O'/E) < \frac{\epsilon}{2} 
                \end{equation*}
                Every open set can be expressed as a countable collection of disjoint open intervals, so is $O'$. Let $O' = \bigcup_{n=1}^{\infty} I_n$.
            From $m^*(E)$ is finite, we can know that $m^*(O')$ is finite. So
            $$\sum_{n=1}^{\infty}|I_n|=m^*(O')$$
            is a converge positive series. Thus there exists a $N$ such that 
            $$\sum_{n=N}^{\infty}|I_n| < \frac{\epsilon}{2}$$
                We define $O =\bigcup_{n=1}^{N}I_n$. Since $O/E \subset O'/E$, and $E/O \subset O'/O$
                \begin{equation*}
                    \begin{aligned}
                        m^*(E/O)+m^*(O/E) &\leq  m^*(O'/O)+m^*(O'/E)\\
                        &< \sum_{n=N}^{\infty}|I_n| + \frac{\epsilon}{2}\\
                        &< \epsilon
                    \end{aligned}
                \end{equation*}
            \end{proof}
            Notice that we used the excision property of measurable set($m^*(B/E) = m^*(B) - m^*(E)$) in the previous theorems, thus they only
        apply to finite measure sets. One should pay attention to the difference between $m^*(O/E) < \epsilon$ and $m^*(O)-m^*(E)<\epsilon$, 
        as the later one is true for any set $E$. In the next proposition, we show that none measurable sets doesn't possess the excersion 
        property.
            \begin{proposition}
                Suppose $E$ is not measurable, and has finite out measure. Then there exist a open set $O$, such that 
                $$m^*(O/E) > m^*(O) - m^*(E)$$
            \end{proposition}
            \begin{proof}
                Since $E$ is not measurable, there exists a set $A$, such that 
                \begin{equation}\label{excision}
                    m^*(A) < m^*(E \cap A) + m^*(A / E)
                \end{equation}
                Given $\epsilon > 0$, let $I_n$ be an open interval cover of $A \cup E$, and $O = \bigcup_{n=1}^{\infty} I_n$, where
            $$m^*(A \cup E) \leq m^*(O) \leq m^*(A \cup E) + \epsilon$$ 
                We only need to show $m^*(O)<m^*(O/E) + m^*(E)$. Since \eqref{excision}, $\exists \epsilon$ such that  
                \begin{equation*}
                    \begin{aligned}
                        m^*(O) &< m^*(A \cup E) + \epsilon \\
                        &< m^*(A) + m^*(E)
                        &< m^*(O/E) + m^*(E)
                    \end{aligned}
                \end{equation*}
                Thus the proposition is proved.
            \end{proof}

    \chapter{Lebesgue Measuralbe Functions}
            Remembering that our purpose is to define Lebesgue integral, integral is an operator on functions, so the concept of measurable function
        is needed. In the first chapter, the range of the function $f$ which we want to integrate is cut into small intervals, and we want to 
        measure the size of their inverse image(of course using the Lebesgue measure). Generaly, the inverse image of these small intervals are not
        measurable, so we make compromise, by working with a specific type of function. 
            In this chapter, we would see three interesting and importent properties of measurable function, called Littlewood's three principle. It
        says \textbf{measurable sets are nearly finite union of open intervals}; \textbf{measurable functions are nearly continuous functions};
        \textbf{a sequence of measurable functions which converge pointwisely is almost converge uniformly}. The first Priciple has already 
        shown by \eqref{LittlewoodFirstPrinciple}. We will show the latter two by Egoroff's theorem and Lusin's theorem. Although the 
        utterance of the principles are not in the rigourous mathematic laguagues, it is still important since it provide a intuitive 
        understanding of those theorems. In my opinion, this kind of understanding is more in need in the mathematic learning, rather than
        the rigourous proof.

        \section{Definition and basic properties}
            Although we want to have $f^{-1}[a,b]$ to be measurable, it turned out that the following condition is equivalent to that.
            \begin{definition}[Lebesgue measuralbe functions]
                Given an extended real value function $f: E \to \mathbb{R}^*$ while $E\in \mathcal{M}$, We define $$E\{f>\alpha\}:=\{x\in E:f(x) > \alpha\}$$. \par
                If $\forall \alpha, \ E\{f>\alpha\} \in \mathcal{M}$ 
                then we say $f$ is measuralbe. We denote the set of all Lebesgue measuralbe functionson a measurable set E by \lmf{E}
            \end{definition}
            \begin{proposition}
                Let the function $f$ have measurable domain. Then the following condition are equivalent to the measurability of $f$.
                \begin{enumerate}
                    \item For each $\alpha$, $E\{f \geq \alpha\}$ is measurable.
                    \item For each $\alpha$, $E\{f < \alpha\}$ is measurable.
                    \item For each $\alpha$, $E\{f \leq \alpha\}$ is measurable.
                \end{enumerate}
            \end{proposition}
            \begin{proof}
                Suppose $f:E \to \RR$ is measurable, which means $\forall \alpha \in \RR, \ E\{f>\alpha\}$ is measurable. Observing that
            $$E\{f \geq \alpha\} = \bigcap_{n=1}^{\infty} E\{f > \alpha + \frac{1}{n}\}$$
            which indicate $\forall \alpha \in \RR, \ E\{f \geq \alpha\}$ is measuralbe.\par
                Suppose $\forall \alpha \in \RR, \ E\{f \geq \alpha\}$ are measuralbe. 
                $$E\{f < \alpha\} = E\{f \geq \alpha\}^c \cap E$$
            Thus $\forall \alpha \in \RR, \ E\{f < \alpha\}$ is measurable.\par
                Suppose $\forall \alpha \in \RR, \ E\{f < \alpha\}$ are measuralbe.
                $$E\{f \leq \alpha\} = \bigcap_{n=1}^{\infty} E\{f < \alpha - \frac{1}{n}\}$$
            Thus $\forall \alpha \in \RR, \ E\{f \leq \alpha\}$ is measurable.\par 
                And by the similar trick, one can show if $\forall \alpha \in \RR, \ E\{f\leq \alpha\}$ is measurable, then 
                $$E\{f > \alpha\} = E\{f \leq \alpha\}^c \cap E$$
            is measurable. 
            \end{proof}\par
            Let's see some examples of measurable functions.
            \begin{proposition}
                $f$ is define on $E \in \mathcal{M}$. 
                \begin{enumerate}
                    \item If $f$ is continued on $E$, $f \in$ \lmf{E}.
                    \item If f is monotonic on $E$, $f \in$ \lmf{E}.
                    \item Define $\chi_E:\mathbb{R} \to \mathbb{R}$ by following ristriction: if $x \in E$, $\chi_E(x)=1$; 
                otherwise, $\chi_E(x)=0$. Then $E \in \mathcal{M} \Longleftrightarrow\ \chi_E \in$ \lmf{E}. 
                \end{enumerate}
            \end{proposition}
            \begin{proof}
                1. Since the inverse image of open set is open(relative open), given $O \in \RR$, we have $f^{-1}(O) = E \cap U$, where $U$ is open.
            Being the intersection of measurable sets, $f^{-1}(O)$ is measurable. Since each $\{x \in E:x > \alpha\}$ is open, thus $E\{f > \alpha\}$
            is measurable.\par
                2. Let $x_0 = \inf( E\{f > \alpha\})$, then for any $x>x_0, \ x \in E$, 
                $$f(x) \geq f(x_0) > \alpha$$
                Thus $E\{f> \alpha \}$ can only be $[\alpha,\infty) \cap E$ or $(\alpha,\infty) \cap E$. Both of them are measurable.\par
                3. Notice that the range of $f$ only contains two points, $1$ and $0$. If $\alpha < 0$, $E\{ f > \alpha \} = \RR$;
                if $\alpha \geq 1$, $E\{ f > \alpha \} = \varnothing$; if $ 0 \leq \alpha < 1$,$E\{ f > \alpha \} = E$, thus 
                $\chi_E$ is measurable if and only if $E$ is measurable.
            \end{proof}
            In some sense, a set with measure zero is very small. Suppose $P$ is a property, $E$ is a set and $E_0$ is a measure zero subset. 
        It turned out that the condition \textsl{for all $x \in E-E_0$, $P(x)$ is true} is strong enough to derive many desirable conclusion.  
        For conveniency, we say \textsl{$P$ is true almost everywhere on $E$} to mean that for all $x \in E$ but a set of measure zero, $P(x)$ 
        is true. And it's abbriviation is \textsl{$P$ is true on E a.e.} 
            \begin{proposition}\label{MZS}
                Let $f:E \to \RR^*$ be a extended real value function, while $E$ is a measurable set. 
                \begin{enumerate}
                    \item If $f$ is measurable and $f = g$ a.e. on $E$, then $g$ is measurable.
                    \item For a measurable subset $D \subset E$, $f$ is measurable if and only if it's restriction on $D$ and $E-D$ are 
                    measurable.
                \end{enumerate}
            \end{proposition}
            \begin{proof}
                1. Suppose $f \neq g$ on $E_0$, which is measure zero. 
                \begin{equation*}
                    \begin{aligned}
                        E\{g < \alpha\} &= (E-E_0)\{g < \alpha\} \cup E_0\{g < \alpha\}\\
                        &= (E-E_0)\{f < \alpha\} \cup (E\{g < \alpha\} \cap E_0)\\
                        &= (E\{f < \alpha\} - E_0) \cup (E\{g < \alpha\} \cap E_0)\\
                    \end{aligned}
                \end{equation*}
                $(E\{g < \alpha\} \cap E_0)$ is a subset of $E_0$, thus it is measure zero, thus measurable; $E\{f < \alpha\} - E_0$
            is the difference of two measurable set. Since the collection of all measurable sets is a $\sigma$ algebra, so it is measurable.
            The union of two measurable set are still measurable, thus $E\{g < \alpha\}$ is measurable. \par
                2. Suppose $f$ is measurable. Then 
                \begin{equation*}
                    \begin{aligned}
                        D\{f|_D < \alpha\} &= D \cap E\{f < \alpha\}\\
                        (E-D)\{f|_{D-E} < \alpha\} &= (E-D) \cap E\{f < \alpha\}
                    \end{aligned}
                \end{equation*}
                So $f|_D$ and $f|_{E-D}$ are measurable.\par
                On the other hand, if $f|_D$ and $f|_{E-D}$ are measurable, 
                \begin{equation*}
                    \begin{aligned}
                        E\{f < \alpha \} &= (E-D)\{f < \alpha\} \cup D\{f < \alpha\}\\
                        &= (E-D)\{f|_{E-D} < \alpha\} \cup D\{f|_D < \alpha\}
                    \end{aligned}
                \end{equation*}
                Thus the proposition is proved.
            \end{proof}
            Just as what is done when studing continuous function, we want to know that whether the sum, product and composition of 
        measurable functions are still measurable. But in general, the sum of two extended real valued is not even well defined. For example,
        if $f(x) = +\infty$ while $g(x) = -\infty$, then $f(x) + g(x)$ is not well defined. If we ask $f$ and $g$ to be finite a.e. in there 
        domain, then we can know the set of points which are not well defined is measure zero. By the former proposition, the value of a 
        function in a measure zero set don't influence the fact whether it is measurable. So we are allowed to talk about whether $f + g$ is 
        measurable in a slightly different condition.
            \begin{theorem}[Linear combinenation and mutlpilication]
                Let $f:E \to \RR^*, \ g: E \to \RR^*$ be measurable and finite a.e. on $E$, $r \in \RR$, then 
                \begin{enumerate}
                    \item $r \cdot f$ is measurable.
                    \item $f+g$ is measurable.
                    \item $f \cdot g$ is measurable.
                \end{enumerate}  
            \end{theorem}
            \begin{proof}
                One can easily find out that $f + g$ is well defined and finite a.e. on $E$. Let $E_0$ be the set that $f + g$ is well defined and 
            finite a.e.. Then by the prec
                1. Since $f$ is measurable, 
            $$E{f>\frac{\alpha}{r}} = E{f>\alpha}$$ is measurable.
                2. The element in $E\{f(x)+g(x) < c\}$ has the following property:
                \begin{equation*}
                    \begin{aligned}
                        f(x) + g(x) &< c\\
                        f(x) &< c - g(x)\\
                    \end{aligned}
                \end{equation*}
                Since the rational number are dense in $\RR$, so for each $x \in E$, we can find a $q \in \QQ$ such that
                $$f(x) < q < c-g(x)$$
                which is equivalence to 
                \begin{equation*}
                    \begin{aligned}
                        f(x) &< q \\
                        g(x) &< c-q
                    \end{aligned}
                \end{equation*}
                So 
                $$ E\{f(x) + g(x) < c \} = \bigcup_{q \in Q} E\{f(x) < c\} \cap E\{g(x) < c-q\}$$
                The intersection and countable union of measurable sets are still measurable, thus $E\{f(x) + g(x) < c \}$ is measurable.
                3. Observing that $fg = \frac{1}{2} (f + g)^2$, by 1. and 2., we only need to show that if f is measurable, then $f^2$ is 
            measurable. $\forall x \in E\{f^2(x)< c\}$, by elementary algebra, 
            $$-c^{\frac{1}{2}}< f(x)<c^{\frac{1}{2}}$$
            Thus $E\{f^2(x)<c\} = E\{f(x) > -c^{\frac{1}{2}}\} \cap E\{f(x)<c^{\frac{1}{2}}\}$, which is measurable. 
            \end{proof}
            Many properties such as continuity or differentiability, are preserved under the composition of functions. Howerver, in general,the 
        composition of measurable functions are not measurable. The counter-example is provided in the appendix. The basic idea of the counter 
        example is to prove that there exsit a measurable function which maps a measurable set to a non-measurable set, then composite it with
        a characteristic functio. To get the composition measurable, one more condition is needed.
            \begin{theorem}[Composition]
                Let $g$ be a measurable real valued function on a measurable $E$ and $f$ a continuous real valued function defined on $\RR$ 
            Then $f \circ g$ is measurable. 
            \end{theorem}
            We first prove a lemma.
            \begin{lemma}
                Let $f:E \to \RR$ be a function defined on a measurable set $E$. $f$ is measurable if and only if for each open sets $O$,
            $f^{-1}(O)$ is measurable.
            \end{lemma}
            \begin{proof}
                Suppose the inverse image of each open sets are measurable, then since $(a, +\infty)$ is open,
                $$E\{f(x)>a\} = f^{-1}(a,+\infty)$$ 
            is measurable, which suggests that $f$ is measurable.\par
                Suppose $f$ is measurable. Given any open sets, we can express it by a countable union of open intervals $\{I_n\}$.
            Suppose $I_n = (a_n, b_n)$. Then
            \begin{equation*}
                \begin{aligned}
                    f^{-1}(O) &= f^{-1}(\bigcup_{n=1}^{\infty} I_n)\\
                    &= \bigcup_{n=1}^{\infty} (f^{-1}(I_n))\\
                    &= \bigcup_{n=1}^{\infty} (E\{f>a_n\} \cap E\{f< b_n\})
                \end{aligned}
            \end{equation*}
            so the inverse image of $O$ is measurable 
            \end{proof}
            \begin{proof} (Proof of the Composition Theorem)
                Let $O \subset \RR$ be open, then we only need to show the inverse image of $O$ under $f \circ g$ is measurable. 
                $$(f \circ g )^{-1}(O) = g^{-1} (f^{-1}(O))$$
                Since $f$ is continuous, $f^{-1}(O)$ is open. By the measurability of $g$, $g^{-1}(f^{-1}(O))$ is measurable. By the 
            upper lemma, the proposition is proved.
            \end{proof}
            At the last of this section, we show that for finite many functions, their maximum function is still measurable.
            \begin{proposition}
                For a set of finite functions $\{f_n\}_{n=1}^{N}$, $max(f_1, \ f_2, \dots f_n)$ is measurable. 
            \end{proposition} 
            \begin{proof}
                We prove the case of $n=2$, and use the mathematic induction. 
                $$E\{max(f, \ f')< c \} = E\{f< c \} \cap E\{f'< c \}$$
                So it is measurable. The part of induction is rather trivial.
            \end{proof}
                The proposition for minimum can be proved in a simialr way. From the upper proposition, one can know 
                $$|f| = max(f, 0) - min(f,0)$$
                is measurable.

        \section{Simple function approximation of measurable function}
                \begin{definition}[Characteristic functions]
                    Given a subset $A$ of $\RR$, the function $\chi_A: \RR \to \RR$ defined by 
                    \begin{equation*}
                        \chi_A(x) = \left\{ \begin{matrix}
                            1  \ \ \ \ x \in A\\
                            0  \ \ \ \ x \notin A
                        \end{matrix} \right.
                    \end{equation*}
                    is called the characteristic function of $A$.
                \end{definition}
                It is clear that $\chi_A$ is measurable if and only if $A$ is measurable.
                \begin{proposition}[Some basic properties of characteristic function]
                    Let $A, \ B$ be sets, then 
                    \begin{equation*}
                        \begin{aligned}
                            &1. \ \ \chi_{A \cap B} = \chi_A \cdot \chi_B \\
                            &2.  \ \ \chi_{A \cup B} = \chi_A + \chi_B - \chi_A \cdot \chi_B\\
                            &3. \ \ \chi_{A^c} = 1 - \chi_A 
                        \end{aligned}
                    \end{equation*}\par
                \end{proposition}
                \begin{proof}
                    Here we prove only the first equality, the rest can be proved in the same manner, leaving as an exercise to the readers.\par
                    If $\chi_{A \cap B}(x) = 1$, then $x \in A$ and $x\in B$, so 
                $$\chi_A \cdot \chi_B = 1 \cdot 1 = 1$$
                If $\chi_{A \cap B}(x) = 0$, then $x \notin A \cap B$. So one of $\chi_A$, $\chi_B$ would be zero, which suggest 
                $$\chi_A \cdot \chi_B = 0$$ \par                   
                \end{proof}
                \begin{definition}[Simple function]
                    A real valued measurable function is \textbf{simple} if and only if it takes only finite values.
                Or in another word, a simple function is measurable and it's image is a finite subset of $\RR$.
                \end{definition}
                One can see immediately that $\chi_A$ is a simple function if it is measurable. We can express any simple function by sum 
            of characteristic functions.
                \begin{proposition}[Canonical representation of simple function]
                    If $\phi:E \to \RR$ is a simple function, then there exist a finite distinct sequence of numbers $\{a_n\}$ and a finite sequence of disjiont measurable sets $\{A_n\}$,
                such that 
                    \begin{equation}
                        \phi(x) = \sum_{n=1}^{N} a_n \chi_{A_n}(x) \label{sa}
                    \end{equation} 
                \end{proposition}
                \begin{proof}
                    Since $\phi$ takes only finite real values, we arrange these number into a sequence $\{a_n\}$. 
                Let $A_n$ be the set contains all the $x \in E$ such that $\phi(x) = a_n$. Because $\phi$ is measurable, so is $A_n$.
                \eqref{sa} is true for those sequence we have just found.
                \end{proof}
                It isn't hard to see that finite sum of simple functions is still simple function, scaler multiple of simple function is still simple 
            function. The proof of next proposition is left to the readers.
                \begin{proposition}[The linearity of simple functions]
                    Let $\varphi:E \to \RR, \ \psi:E \to \RR $ be simple functions, and $r$ be a real number. Then 
                    \begin{equation}
                            \varphi + \psi \ , \  \ c \cdot \varphi \mbox{ are simple function}
                    \end{equation}
                \end{proposition}
                \begin{lemma}[Simple approximation lemma]
                    Let $f:E \to \RR$ be a bounded measurable function. For all $\epsilon >0$, there exist
                simple functions $\varphi:E \to \RR, \ \psi:E \to \RR$, such that 
                    \begin{equation}
                        \begin{aligned}
                            \varphi \leq f \leq \psi\\
                            \psi - \varphi \leq \epsilon 
                        \end{aligned} \label{spl}
                    \end{equation} \par
                \end{lemma}
                \begin{proof}
                    Let $[c,d]$ be a close bounded interval, whose interior contains $f(E)$ and let $P = \{c=y_0, y_1, y_2, \dots y_n=d\}$ 
                be a partition of $[c,d]$ such that $y_i-y_{i-1} < \epsilon$. We define $E_i = E\{y_{i-1}\leq f<y_i\}$, and 
                \begin{equation}
                    \begin{aligned}
                        \varphi(x) &= \sum_{i=1}^{n} y_{i-1} \chi{E_i}(x)\\
                        \psi(x) &=  \sum_{i=1}^{n} y_{i} \chi_{E_i}(x)
                    \end{aligned}
                \end{equation} 
                Then it is easy to varify that these two functions satisfy \eqref{spl}
                \end{proof}
                \begin{theorem}[Simple approximation Theorem]
                    A extended real value function on a measuralbe set, say $f:E \to \RR^*$, is measurable if and only if there exists a 
                sequence of simple functions $\{\varphi_n\}$ which converge to $f$ on $E$ pointwisely, and has the property that
                $$|\varphi_n| \leq |f| \mbox{ on $E$ for all $n$}$$
                    More over, if the function is bounded, $\{\varphi_n\}$ can converge uniformly; if the function is nonegative, $\{\varphi_n\}$
                can converge increasingly.
                \end{theorem}
                \begin{proof}
                    Since each simple functions are measurable, and sequence of measurable functions converge to measurable function, so
                the "if" is proved. \par 
                    Assume f is measurable, and also assume $f\geq 0$ on $E$. Let $E_n = \{x \in E : f(x) \leq n\}$, then $f$ is nonegative bounded measurable function
                on $E_n$. So by simple approximation lemma, there exist $\varphi_n , \ \psi_n$, such that 
                $$\varphi_n \leq f \leq \psi_n$$
                $$\psi_n - \varphi_n \leq \frac{1}{n}$$ 
                on $E_n$. So 
                $$f - \varphi_n \leq \frac{1}{n}$$
                on $E_n$. We extend $\varphi_n$ on all $E$ by setting $\varphi_n(x) = n$ for $x$ such that $f(x) \geq n$.\par
                \textbf{Claim:} $\{\varphi_n\}$ converge to $f$. \par
                    If $f(x)$ is finite, then exists a $N$ such that $f(x) < N$. Then 
                    $$0< f(x) - \varphi_n(x) <\frac{1}{n} \mbox{ for $ n \geq N$}$$
                Thus $\varphi_n(x)$ converge to $f(x)$.\par
                    If $f(x) = \infty$, then $\varphi_n(x) = n$, thus we also have $\lim_{n\to \infty}\varphi_n(x) = \infty$.
                    Replacing each $\varphi_n$ by $max(\{\varphi_i\}_{i=1}^{n})$, then we have $\varphi_n$ converge increasingly.\par
                    If $f$ is not nonegative, let $f_+=max(f, 0), \ f_-= -min(f, 0)$, then $f = f_+ - f_-$. Since $f_+, \ f_-$ are nonegative 
                measurable functions, one can find two sequences of simple functions $\{\varphi_n\}, \ \{\psi_n\}$, whose limit are 
                $f_+$ and $f_-$ separately. Thus
                    \begin{equation}
                        \begin{aligned}
                            f &= f_+ - f_-\\
                            &= \lim_{n\to \infty} \varphi_n -lim_{n \to \infty} \psi_n\\
                            &= \lim_{n \to \infty} \varphi_n - \psi_n
                        \end{aligned}
                    \end{equation}
                    Since the sum of simple functions are still simple function, we find a sequence of simple function which converge to 
                $f$ pointwisely. \par
                    If $f$ is bounded, there exist a $M > |f|$. So $f(E) \subset [-M,M]$. Given $n>0$, We can choose a partition of $M, \ P=\{y_i\}_{i=1}^{n}$ 
                which is finer enough, such that $y_{i}-y_{i-1} <\frac{1}{n}$, and define $\varphi_n$ in the same manner. Then $\forall \epsilon > 0, \ \exists N$
                such that $\forall n \geq N, \ x \in E, \ |\varphi_n - f| \leq \epsilon$. 
                \end{proof}
        
        \section{Egoroff's Theorem \& Lusin's Theorem}
            Now we are able to prove the last two principles which have been mentioned in the beginning of this chapter. The result is 
        quite amazing. We will first deal with the Egoroff's theorem, which shows the thrid principle. In some sense, sequences of measurable 
        functions which converge pointwisely are nearly converge uniformly. To prove this theorem, we first need to know more about the
        convergence of measurable functions and measurable sets.
            \begin{proposition}[Continuity of measure]
                Let $\{A_n\}$ be a ascending sequence of sets, which means $A_1 \subset A_2 \subset \dots A_n \subset \dots$.
            Let $A = \bigcup_{n=1}^{\infty} A_n$. Then $$m(A) = \lim_{n \to \infty} m(A_n)$$
                If $\{B_n\}$ is descending, then let $B = \bigcap_{n=1}^{\infty} B_n$. We have 
                $$m(B) = \lim_{n \to \infty} m(B_n)$$

            \end{proposition}
            \begin{proof}
                We first prove the part of ascending sequence. If there is a index $k$ such that $m(A_{k}) = \infty$, then by the monotonicity of 
            measure $m(A) = \infty$, thus the inequality holds. If all $A_k$ are finite, we define $A_0 = \varnothing$, and $C_n = A_n -A_{n-1}$.
            Then$$A = \bigcup_{n=1}^{\infty} A_n =\bigcup_{n=1}^{\infty} C_n$$ 
            and $\{C_n\}$ are disjoint. For disjoint union of sets of finite measure, we can apply the countable additivity of measure, 
            and the excirsion property of the measurable set, 
            \begin{equation*}
                \begin{aligned}
                    m(A) = m(\bigcup_{n=1}^{\infty} C_n) &= \sum_{n=1}^{\infty} m(A_n - A_{n-1})\\
                    &= \sum_{n=1}^{\infty} m(A_n) - m(A_{n-1})
                    &= \lim_{n \to \infty} m(A_n) - m(A_0)
                    &=\lim_{n \to \infty} m(A_n)
                \end{aligned}
            \end{equation*}
                For descending sequence $\{B_n\}$, if for all $n$, $m(B_n) = \infty$, then?????????????? 
            
            \end{proof}
            \begin{proposition}[Pointwise limit of measurable functions]
                If a sequence of measurable functions $\{f_n\}$ converge pointwise a.e. on $E$, where $E$ is their domain. Then the limit 
            function $f$ to which $\{f_n\}$ converge, is measurable.     
            \end{proposition}
            \begin{proof}
                Suppose the $E_0$ is the subset which $f$ doesn't converge. By proposition \ref{MZS}, $f$ is measurable if and only if it's 
            ristriction on $E - E_0$ is measurable. So it is reasonable to assume that $\{f_n\}$ converge on $E$. 
                For all $x \in E\{f< c\}$, since $f$ is the pointwise limit, there exist a N and m such that 
            $$\forall n \geq N, \ f_n(x) < c - \frac{1}{m}$$
                Since $E\{f_n< c-\frac{1}{m}\}$ is measurable for all $m, \ n$, thus 
            $$\bigcap_{n=N}^{\infty} E\{f_n < c -\frac{1}{m}\}$$
            is measurable. Consequently, 
            $$E\{f(x)<c\} = \bigcup_{1 \leq N, m < \infty}(\bigcap_{n=N}^{\infty} E\{f_n < c -\frac{1}{m}\})$$
            is measurable. 
            \end{proof}
            \begin{theorem}[Egoroff's Theorem]
                If a sequence of measurable functions $\{f_n\}$ converge pointwise a.e. on $E$, where $E$ is their domain which has finite
            measure. Then $\forall \epsilon > 0$, there exist a closed $F \subset E$, such that $\{f_n\}$ converge uniformly on $F$, and
            $m(E-F) < \epsilon$. 
            \end{theorem}
            Since we want to prove the uniform convergence, the estimation $|f_n - f|< a$ will be important. Also, we need to find a closed
        set which is good enough. We will first find a measurable set which is good enough(which is easier), and then use closed set to approximate it. 
        To these purposes, it is conveneint to establish the following lemma.
            \begin{lemma}
                Under the assumptions of Egoroff's theorem, for each $a, b > 0 $, there is a measurable subset $A \subset E$ and a $N'$ such
            that 
            $$|f_n - f| < a \mbox{ on $A$ for all $n>N$ where } m(E -A) <b $$
            \end{lemma}
            \begin{proof}
                We can simplely assume $f_n \to f$ on $E$. By the previous discussion, one can know that $|f_n-f|$ is measurable function. 
            Let $$E_N = \bigcap_{n=N}^{\infty} E\{|f_n-f|<a\}$$
            which is a measurable subset of $E$. Since $\{f_n\}$ converge pointwisely to $f$, $\{E_N\}$ converge to $E$ as $n$ goes 
            to infinity. So by excirsion property, there exists a $N'$ such that $m(E - E_{N'}) = m(E) - m(E_{N'}) = b$. For the same $E_{N'}$, pick any $x$ in $E_{N'}$, we have 
            $$\forall n \geq N', \ |f_n - f| < a$$ 
            So $A = E_{N'}$ is the set we want. 
            \end{proof}
                \begin{proof}\textbf{(Proof of the Egoroff's Theorem)}
                Given $\epsilon > 0$, let $A = \bigcap_{m=1}^{\infty} A_m$, where $A_m$ is the set in the above lemma 
            such that there exists a $N$, for $n\geq N$, $|f_n -f|< \frac{1}{m}$ on $A_m$, and $m(E - A_m)< \frac{\epsilon}{2^{m+1}}$.
            It is easy to see that $A$ is measurable subset of $E$. And we also have 
            $$m(E-A) = m(E-(\bigcap_{m=1}^{\infty}A_m)) = m(\bigcap_{m=1}^{\infty} (E-A_m)) \leq \sum_{m=1}^{\infty} m(E - A_m)< 
            \sum_{m=1}^{\infty} \frac{\epsilon}{2^{m+1}} = \frac{\epsilon}{2}$$    
            $$m(E-A)< \frac{\epsilon}{2}, \ \ \ \forall \epsilon' >0, \ \forall x \in A, \ \exists N \ s.t. \
            \forall n >N , \ |f_n(x) - f(x)|< \epsilon' $$
            which means $f_n$ converge uniformly on $A$. Using the close inner approximation of measurable set, we can find a close 
        subset of $A$, say $F$, such that, $m(A-F) = \frac{\epsilon}{2}$
            Finally
            $$m(E-F) = m((E-A) \cup (A - F)) = m(E - A) +m(A - F) = \epsilon$$
            and $\{f_n\}$ converge uniformly on $F$.
            \end{proof}
            This amazing theorem is very powerful. For pointwise converge sequence of functions, you can use Egoroff's theorem to get a 
        strong condition on a very big subset. One thing need to be paid attention on is that we can only apply Egoroff's theorem 
        on the functions which are defined on finite measure sets. In mathematic analysis, uniform converge can preserve the integrability, continuity.
        The next theorem is the proof of the Littlewood second principle, which is related to continuity. We will see how Egoroff's theorem 
        work. \par
            The main idea of the prove of Lusin's theorem is to first prove the case of simple function, then use simple functions to 
        approximate measurable functions. Since for 
            \begin{theorem}[Lusin's Theorem]
                Let $f$ be a measurable function which is finite a.e. on $E$, a measurable set. Then given $\delta > 0$, 
            there exist a closed subset $F \subset E$, such that $m(E-F)< \delta$, $f$ is conintuous on $F$.
            \end{theorem}
            \begin{proof}
                Suppose $f$ is finite. First we consider the case of $f(x) = \sum_{n=1}^{N} a_n \chi_{E_n}(x)$ is a simple function,
            where the sum is the canonical representation. 
            Let $F_n$ be close subset of $E_n$, by inner approximation, we can require $m(E_n-F_n) = \frac{\delta}{n}$. On each $E_n$,
            $f$ is a constant function, so is it on each $F_n$. Let $F = \bigcup_{n=1}^{N} F_n$. Being a finite union of close sets, $f$
            is closed. Then
                \begin{equation*}
                    \begin{aligned}
                        m(E-F) &= m(\bigcup_{n=1}^{N} E_n - \bigcup_{n=1}^{N} F_n)\\
                        &\leq m(\bigcup_{n=1}^{N} E_n - F_n)\\
                        &= \sum_{n=1}^{N} m(E_n - F_n)\\
                        &\leq \sum_{n=1}^{N} \frac{\delta}{n} = \delta
                    \end{aligned}
                \end{equation*}
            where the first inequality is come from the fact that $\bigcup_{n=1}^{N} E_n - \bigcup_{n=1}^{N} F_n \subset \bigcup_{n=1}^{N} E_n - F_n $.
            Thus we proved the case of $f$ is simple functions. \par
                Suppose $m(E)<+\infty$. By simple approximation of measurable function, we can find a sequence of simple functions $\{\varphi_i\}$,
            which converge to $f$ pointwisely. By Egoroff's theorem, there exists a close subset $F \subset E$, such that $\{\varphi_i\}$ coneverge 
            uniformly on $F$ to $f$, while $m(E-F) < \frac{\epsilon}{2}$. \par
                From the previous part of this proof, for every $\{\varphi_i\}$, we can find a closed set of $E$, say $F_i$, 
            such that $\varphi_i$ is continuous on $F_i$, while $m(E-F_i) < \frac{\epsilon}{2^{i+1}}$. We can see that, all $\varphi_i$
            is continuous on $\bigcap_{i=1}^{\infty} F_i$, which is a close set, and we have 
            \begin{equation*}
                \begin{aligned}
                    m(E - (\bigcap_{i=1}^{\infty} F_i)) &= m(\bigcup_{i=1}^{\infty} E - F_i)\\
                    &\leq \sum_{i=1}^{\infty} m(E - F_i)\\
                    &\leq \sum_{i=1}^{\infty} \frac{\epsilon}{2^{i +1}}\\
                    &= \frac{\epsilon}{2}
                \end{aligned}
            \end{equation*}\par 
                On $F \cap (\bigcap_{i=1}^{\infty} F_i)$, which is a close set, $\{\varphi_i\}$ is continuous and uniformly converge to
            $f$. By knowledge from mathematic analysis, $f$ is continuous on $F \cap (\bigcap_{i=1}^{\infty} F_i)$. It only remains to show that 
            $m (E - (F \cap (\bigcap_{i=1}^{\infty} F_i)))$ is small. 
            \begin{equation*}
                \begin{aligned}
                    m (E - (F \cap (\bigcap_{i=1}^{\infty} F_i))) &= m ((E - F) \cup (E-\bigcap_{i=1}^{\infty} F_i))\\
                    &\leq m (E - F) + m(E-\bigcap_{i=1}^{\infty} F_i)\\
                    &\leq \frac{\epsilon}{2} +\frac{\epsilon}{2}\\
                    &=  \epsilon
                \end{aligned}
            \end{equation*}
            Then we proved the case that $E$ has finite measure.\par
                For the case that $m(E) = +\infty$, we consider the ristirction of $f$ on $E_n = E \cap [-n,n]$. Clearly, $f|_{E_n}$ are  
            measurable functions defined on finite measure sets, so by the Lusin's theorem for finite case, we can find $F_n$ for each 
            $E_n$ such that $f = f|_{E_n}$ is continuous on $F_n$, and $m(E_n-F_n) < \frac{\epsilon}{2^{n+1}}$.\par
                Let $F = \bigcup_{n=1}^{\infty} F_n$, which is measurable. By our construction, $f$ is continuous on $F$. With the fact that 
            $\bigcup_{n=1}^{\infty}E_n - \bigcup_{n=1}^{\infty}F_n \subset \bigcup_{n=1}^{\infty}E_n -F_n$ we can obtain the estimation below,
            \begin{equation*}
                \begin{aligned}
                    m(E-\bigcup_{n=1}^{\infty} F_n) &= m(\bigcup_{n=1}^{\infty}E_n - \bigcup_{n=1}^{\infty}F_n)\\
                    &\leq m( \bigcup_{n=1}^{\infty}E_n - F_n)\\
                    &\leq \sum_{n=1}^{\infty}m(E_n - F_n)\\
                    &\leq \sum_{n=1}^{\infty} \frac{\epsilon}{2^{n+1}}\\
                    &= \frac{\epsilon}{2} 
                \end{aligned}
            \end{equation*}
                But this is not the end, since $F$ is not a close set. According to the inner approximation of measurable function, we can
            find a close set of $F$, say $F'$, such that $m(F-F')< \frac{\epsilon}{2}$. Then 
            \begin{equation*}
                \begin{aligned}
                    m(E-F') &= m ((E-F) \cup (F-F'))\\
                    &= m(E-F) + m(F-F')\\
                    &=\frac{\epsilon}{2} +\frac{\epsilon}{2}
                    &=\epsilon 
                \end{aligned}
            \end{equation*}
            \end{proof}
            The continuous function on close set can be extended to $\RR$, so we can have the following form of Lusin' theorem.
            \begin{theorem}[Lusin's theorem: another form]
                Let $f$ be a measurable function which is finite a.e. on $E$, a measurable set. Then given $\delta > 0$, 
            there exist a closed subset $F \subset E$, and a continuous function $g: \RR \to \RR$, such that $m(E-F)< \delta$, 
            $f = g$ on $F$.
            \end{theorem}   
    
    \chapter{Lebesgue Integration}
        Now all the preparations have done, it is the time to define the integral. We will first define the the integral for simple functions
    on sets of finite measure; then for the bounded measurable functions on finite measure sets; then for nonegative measurable functions on measurable sets;
    finally the general measurable functions over measurable sets. Without other modifier, the word "integral" in this chapter means Lebesgue
    integral.
        \section{Integral of Simple Functions}
            Remember that in the previous section, we have show that a simple function $\varphi:E \to \RR$ can be written in this form
        $$\varphi(x)  = \sum_{n=1}^{N} a_n \chi{E_n}$$
        where $E_n = E\{f=a_n\}$. This way of writting simple functions suggest a natrual way to define the integral for them.
            \begin{definition}[The integral for simple functions]
                Let $\varphi:E \to \RR$ be a simple function, where $E$ is a measurable set. We define the integral of $\varphi$ over $E$
                \begin{equation}
                    \int_E \varphi = \sum_{n=0}^{N} a_n \cdot m(E_n)
                \end{equation}
            where $\sum a_n \chi_{E_n}(x)$ is the canonical representation of $\varphi$. 
            \end{definition}
            \begin{lemma}\label{thm:simpleintlemma}
                Let $\{E_i\}$ be a finite disjoint of subsets of a finite measure set $E$. Let $\{a_i\}$ be a sequence of real number, then
            $$\varphi = \sum_{i=1}^n a_i \chi_{E_i} \mbox{ on } E \ \Rightarrow \ \int_E \varphi = \sum_{i=1}^{n} a_i \cdot m(E_i)$$\par
                \textbf{Remark:} One may wonder why we should prove this(it just seems like have no difference to the definition of integral).
            Since $\{a_i\}$ is not distinct, $\sum_{i=1}^n a_i \chi_{E_i}$ may not be the canonical representation of $\varphi$.\par
            \end{lemma}
            \begin{proof}
                Let $\{\lambda_j\}$ be distinct values taken by $\varphi$ and $A_j=\{x:\varphi(x)=\lambda_j\}$. By the definition of integral,
            $$\int_E \varphi = \sum_{j=1}^{m} \lambda_j \cdot m(A_j)$$ 
            Let $I_j = \{i:a_i =\lambda_j\}$. It isn't hard to see that $\sum_{i \in I_j} m(E_i) = A_j$
            \begin{equation}
                \begin{aligned}
                    \sum_{i=1}^{n} a_i \cdot m(E_i) &= \sum_{j=1}^{n} (\lambda_j \cdot \sum_{i \in I_j} m(E_i))\\
                    &= \sum_{j=1}^{n} (\lambda_j \cdot m(A_i))\\
                    &= \int_E \varphi
                \end{aligned}
            \end{equation}
            \end{proof}
            Of course we want to know the properties of the general Lebesgue integral, here is our first step in this direction.
            \begin{proposition}[The linearity of the intergration: simple function]
                Let $\varphi, \psi$ be simple functions on $E$, which is a set of finite measure. Then for $\alpha, \ \beta \in \RR$,
                \begin{equation}
                    \int_E (\alpha \varphi + \beta \psi) = \alpha \int_E \varphi + \beta \int_E \psi \label{lsf}
                \end{equation}
            \end{proposition} 
            \begin{proof}
                For linearity, we first prove $\int_E \alpha \varphi = \alpha \int_E \varphi$(scaler can 'go though' the integral sign), then prove \eqref{lsf} for the case of
            $\alpha=\beta = 1$. We suppose $\varphi = \sum_{i=1}^n a_i \chi_{E_i}$ and $\psi = \sum_{i=1}^m b_i \chi_{F_i}$.
            \begin{equation}
                \begin{aligned}
                    \int_E \alpha \varphi &= \sum_{i=1}^{n} \alpha \cdot a_i m(E_i) \\
                    &= \alpha \sum_{i=1}^{n} a_i m(E_i)\\
                    &= \alpha \int_E \varphi
                \end{aligned}
            \end{equation}
                Let $A_{ij} = E_i \cap F_j$, where $1\leq i\leq n$, $1 \leq j \leq m$. For all $x \in E$, $x \in E_i$ and $x \in F_j$ for 
            some $i, \ j$, and since $\{E_i\}$ and $\{F_j\}$ are disjoint, so $\{A_{ij}\}$ is a disjoint finite collection of subset of $E$
            whose union is $E$. Let $a_{ij}, b_{ij}$ be the value of $\varphi, \ \psi$ on $A_{ij}$. By lemma \ref{thm:simpleintlemma},
            \begin{equation*}
                \begin{aligned}
                    \int_E \varphi &= \sum_{1\leq i \leq n, \ 1 \leq j \leq m} a_{ij} m(A_{ij}) \\
                    \int_E \psi &= \sum_{1\leq i \leq n, \ 1 \leq j \leq m} b_{ij} m(A_{ij}) \\
                    \Rightarrow \int_E \varphi + \int_E \psi &= \sum_{1\leq i \leq n, \ 1 \leq j \leq m} a_{ij} m(A_{ij}) +\sum_{1\leq i \leq n, \ 1 \leq j \leq m} b_{ij} m(A_{ij}) \\
                    &=\sum_{1\leq i \leq n, \ 1 \leq j \leq m} (a_{ij} + b_{ij}) m(A_{ij})\\
                    &= \int_E \varphi +\psi 
                \end{aligned}
            \end{equation*}
            \end{proof}
            \begin{proposition}[Monotonicity of the integral]
                Let $\varphi, \psi$ be simple functions on $E$, which is a set of finite measure. Then 
            $$\varphi \leq \psi \Rightarrow \int_E \varphi \leq \int_E \psi$$
            \end{proposition}
            \begin{proof}
                Just like what we did in the previous proposition, we define $A_{ij}, \ a_{ij}, \ b_{ij}$ such that 
            \begin{equation*}
                \begin{aligned}
                    \int_E \varphi &= \sum_{1\leq i \leq n, \ 1 \leq j \leq m} a_{ij} m(A_{ij}) \\
                    \int_E \psi &= \sum_{1\leq i \leq n, \ 1 \leq j \leq m} b_{ij} m(A_{ij}) 
                \end{aligned}
            \end{equation*}
                It is easy to see that $a_{ij} \leq b_{ij}$, so is their finite sum.
            \end{proof}

        \section{Integral for Bounded Measurable Functions over Bounded Sets}
            In this section, we define the integral over bounded sets for bounded measurable functions out of the integral for simple function.  
            \begin{definition}[Upper integral, lower integral and integral]
                Let $f:E \to \RR$ be a Bounded measurable function over a finite measure set. We define
                \begin{equation*}
                        \begin{aligned}
                            &\sup{\int_E \varphi: \varphi \mbox{ is simple and } \varphi \leq f}\\
                            &\sup{\int_E \psi: \psi \mbox{ is simple and } \psi \geq f}
                        \end{aligned}
                \end{equation*}
            to be the upper integral and the lower integral of $f$ respectively, and denote them as 
            $$\sup \int_E f, \ \ \inf \int_E f $$\par
                If upper integral and lower integral of $f$ on $E$ are equal, we define there common value to be the \textbf{Lebesgue integral}
            of $f$ on $E$. 
            \end{definition}
            \begin{theorem}[Lebesgue integral is a generalization of Riemann integral]
                If a bounded function $f$ defined on close bounded interval $[a,b]$ is Riemann integrable, then it is Lebesgue integrable,
            and two integrals are equal.
            \end{theorem}
            In this theorem, we use $(R)\int$ to represent Riemann integral. 
            \begin{proof}
                Since $f$ is Riemann integrable on $[a,b]$, it means 
            $$\sup \{(R)\int_{[a,b]} \varphi(x):\varphi(x)\leq f \mbox{ is a step function }\}
            = \inf \{(R)\int_{[a,b]} \psi(x):\psi(x)\geq f \mbox{ is a step function }\}$$
            Each step function is a simple function, and we can easily observe that the Riemann integral and Lebesgue integral are equal(
            since the lenth of the interval is the Lebesgue measure of the interval).
            We can find a sequence of step functions, which are measurable functions, converge to $f$ pointwisely. So $f$ is a measurable 
            function. The Riemann integralbility also indicate that $f$ is bounded. Then by the definition of Lebesgue integral for bounded
            measurable functions, the Lebesgue integral exist and $$\int_{[a,b]} f = (R) \int_{[a,b]} f$$                
            \end{proof}
            \begin{theorem}[Bounded measurable functions are integralbe]
                Let $f$ be a bounded measurable function which is defined on a set with finite measure, say $E$.
            Then $f$ is measurable.
            \end{theorem}         
            \begin{proof}
                Since $f$ is bounded and measurable, given $\epsilon> 0$, we can find two sequence of simple functions $\{\varphi_n\}, \ \{\psi_n\}$, such that 
            \begin{equation*}
                \varphi \leq f \leq \psi 
            \end{equation*}
            and $\psi - \varphi < \frac{\epsilon}{m(E)}$. By the definition of upper and lower integral and the linearity of 
            the integral for simlpe functions  
            \begin{equation*}
                \begin{aligned}
                    \int_E \varphi \leq \inf \int_E f &\leq \sup \int_E f \leq \int_E \psi\\ 
                    \Rightarrow \sup \int_E f - \inf \int_E f &\leq \int_E \psi - \int_E \varphi\\
                    &= \int_E (\psi - \varphi)\\
                    &\leq \int_E \frac{\epsilon}{m(E)} \\
                    &=\epsilon 
                \end{aligned}
            \end{equation*} 
                Then by the arbritariness of $\epsilon$, $\sup \int_E f = \inf \int_E f$, thus the theorem is proved.
            \end{proof}
            
            In fact, a bounded function over finite measure set is integrable if and only if it is measurable, we will prove this later.
        Now we can establish the theorem for the linearity and monotonicity of the integral of bounded measurable functions.
            \begin{theorem}[Linearity]
                Let $E$ be a finite measure set, and $f, \ g$ be bounded measurable function define on $E$. Let $\alpha, \ \beta$ be
            real numbers. Then
                \begin{equation}
                    \alpha \int_E f + \beta \int_E g = \int_E \alpha f + \beta g
                \end{equation}
            \end{theorem}
            \begin{proof}
                We first prove that $$\int_E \alpha f = \alpha \int_E f$$, then prove $$\int_E f + \int_E g = \int_E f+g $$
            Suppose $\alpha>0$(the argument for $\alpha <0$ is similar), by simple approximate lemma, given $\epsilon$, there exist simple
            functions $\varphi, \ \psi$, such that $\psi - \varphi < \frac{\epsilon}{\alpha \cdot m(E)}$
                \begin{equation*}
                    \begin{aligned}
                        \varphi \leq &f \leq \psi \\
                        \alpha \varphi \leq \alpha&f \leq \alpha\psi \\
                    \end{aligned}
                \end{equation*}
            By the upper theorem, $\alpha f$ is integralbe. And by the definition of the integral, the linearity of the integral of simple functions,
            \begin{equation*}
                \begin{aligned}
                    \int_E \alpha \varphi \leq &\int_E \alpha f \leq \int_E \alpha \psi\\
                    \Rightarrow \alpha \int_E \varphi \leq &\int_E \alpha f \leq \alpha \int_E \psi\\ 
                \end{aligned}
            \end{equation*}
            Notice that we also have 
            $$\alpha \int_E \varphi \leq \alpha \int_E f \leq \alpha \int_E \psi$$
            So 
            \begin{equation*}
                \begin{aligned}
                    |\alpha \int_E f - \int \alpha f| &\leq \alpha \int_E \psi - \alpha \int_E \varphi \\
                    &\leq \alpha \int_E \psi - \varphi \\
                    &\leq \alpha \int_E \frac{\epsilon}{\alpha \cdot m(E)}\\
                    &\leq \alpha \cdot m(E) \cdot \frac{\epsilon}{\alpha \cdot m(E)} = \epsilon 
                \end{aligned}
            \end{equation*}
            Thus the first part of the theorem is proved. \par 
                Since the sum of measurable function is still measurable, $f+g$ is integralbe over $E$. Suppose 
            $\varphi_1, \ \varphi_2, \ \psi_1, \ \psi_2$ are simple functions such that 
            $$\varphi_1 \leq f \leq \psi_1, \ \varphi_2 \leq g \leq \psi_2$$
            We can see that 
            \begin{equation*}
                \begin{aligned}
                    \int_E f+ g &= \sup \int_E f+g \\
                    &\leq \int_E \psi_1 + \psi_2\\
                    &= \int_E \psi_1 + \int_E \psi_2\\
                \end{aligned}
            \end{equation*}
            Since the above inequality is true for all $\psi_1 > f$, $\psi_2 > g$, we have 
            \begin{equation*}
                \begin{aligned}
                    \int_E f+ g &\leq \inf_{\psi_1 \geq f}\int_E \psi_1 +  \inf_{\psi_2 \geq g}\int_E \psi_2\\
                    &= \int_E f +\int_E g 
                \end{aligned}
            \end{equation*}
            On the other hand, 
            \begin{equation*}
                \begin{aligned}
                    \int_E f+ g &= \inf \int_E f+g \\
                    &\geq \int_E \varphi_1 + \varphi_2\\
                    &= \int_E \varphi_1 + \int_E \varphi_2\\
                \end{aligned}
            \end{equation*}
            This inequality is also true for all $\varphi_1 > f$, $\varphi_2 > g$, thus 
            \begin{equation*}
                \begin{aligned}
                    \int_E f+ g &\geq \sup_{\varphi_1 \leq f}\int_E \varphi_1 +  \sup_{\varphi_2 \leq g}\int_E \varphi_2\\
                    &= \int_E f +\int_E g 
                \end{aligned}
            \end{equation*}
            Together the two inequality, we have 
            $$\int_E f+ g = \int_E f +\int_E g $$.
            Thus we prove the linearity of the integration for bounded measurable functions.
            \end{proof}
            \begin{theorem}[Monotonicity]
                Let $E$ be a finite measure set, and $f, \ g$ be bounded measurable function define on $E$. If $f \leq g$ on $E$, then 
                \begin{equation}
                    \int_E f \leq \int_E g
                \end{equation}
            \end{theorem}
            \begin{proof}
                Let $h = g-f$, which is a non-negative measurable function. By linearity, 
                $$\int_E g - \int_E f  = \int_E g-f = \int_E h$$
                Since $h$ is non-negative, $h\geq 0$. Let $\psi = 0$, we can see that 
                $$\int_E h \geq \int \psi = 0 $$
            which shows that $$\int_E g - \int_E f  \geq 0$$
            Thus the monotonicity of the integration of bounded measurable functions is proved.
            \end{proof}
            By the Monotonicity, and the fact that $-|f|\leq f\leq |f|$we can get the following useful conclusion.
            \begin{corollary}[Absolute value]
                Let $f$ be a bounded measurable function defined on a set of finite measure. Then
                $$\int_E |f| \geq | \int_E f |$$ 
            \end{corollary}
            \begin{proposition}[The integral over measure zero set is zero]\label{The integral over measure zero set is zero}
                Suppose $f$ is bounded measurable function on a measure zero set $ E$, then 
                $$\int_E f = 0$$
            \end{proposition}
            \begin{proof}
                Since $f$ is bounded, there exist a $M$, such that $|f|< M$ on $E$. So by monotonicity of the integral, we have 
                \begin{equation*}
                    \begin{aligned}
                        \int_E |f| &\leq \int_E M\\
                        &= M \cdot m(E) 
                        &= 0 
                    \end{aligned}
                \end{equation*}
                By previous corollary, the proposition is proved 
            \end{proof}
            Now we proved the last basic property of the integral: the additivity over domain. 
            \begin{lemma}
                Let $f$ be a bounded measurable function defined on $E$, and $E_0$ be a measurable subset. Then   
                \begin{equation*}
                    \int_{E_0} f = \int_E f \cdot \chi_{E_0}
                \end{equation*}
            \end{lemma}
            \begin{proof}
                For all simple functions $\varphi \leq f$, we exetend $\varphi$ to a new simple function on $E$ by letting 
            $\varphi'(x) = 0$ for $x \in E-E_0$. Notice that the integral of two functions are the same: 
            $$\int_{E_0} \varphi = \int_E \varphi'$$
            For $x \in E_0$, $\varphi \leq f|_{E_0} = f \cdot \chi_{E_0}$; for $x \in E-E_0$, $\varphi = 0 \leq f\cdot \chi_{E_0}$. Thus  
            \begin{equation*}
                \begin{aligned}
                    \int_{E_0} f &= \inf_{\varphi \leq f} \int_{E_0} \varphi \\
                    &= \inf_{\varphi' \leq f} \int_E \varphi' \\
                    &\leq \int_E f \cdot \chi_{E_0}
                \end{aligned}
            \end{equation*}
                For simple functions $\psi \geq f$, the extension $\psi'$ on $E$ with $\psi'(x) = 0$ for $x \in E-E_0$, is also a simple 
            funcion. For $x \in E_0$, $\psi' \geq f|_{E_0} = f \cdot \chi_{E_0}$; for $x \in E-E_0$, $\psi' = 0 \geq f\cdot \chi_{E_0}$.   
            \begin{equation*}
                \begin{aligned}
                    \int_{E_0} f &= \sup_{\psi \geq f} \int_{E_0} \psi \\
                    &= \inf_{\psi' \geq f} \int_E \psi' \\
                    &\geq \int_E f \cdot \chi_{E_0}
                \end{aligned}
            \end{equation*}
            Together the two inequality, we proved the theorem. 
            \end{proof}
            \begin{theorem}[The additivity over domain]
                Let $f$ be bounded measurable functions over $E$, which is a finte measure set. Suppose $A$ and $B$ are disjoint subsets
            of $E$, then 
            $$\int_{A \cup B} f = \int_{A} f + \int_B f$$
            \end{theorem}
            \begin{proof}
                By the linearity of the integral, and the lemma above,
                \begin{equation*}
                    \begin{aligned}
                        \int_A f +\int_B f &= \int_{A \cup B} f \cdot \chi_A +\int_{A \cup B} f\cdot \chi_B\\
                        &= \int_{A \cup B} f \cdot \chi_A + f\cdot \chi_B\\
                        &= \int_{A \cup B} f \cdot \chi_{A\cup B}\\
                        &= \int_{A \cup B} f 
                    \end{aligned}
                \end{equation*}
            \end{proof}
            From the additivity over domain, we can derive a interesting result, which can enhance many propositions which involving integral.
            \begin{corollary}[Measure zero set does not influence the integral]
                Let $f, \ g$ be bounded measurable function on $E$, which is a finite measure set. If $f = g$ a.e. on $E$, 
                $$\int_E f = \int_E g$$
            \end{corollary}
            \begin{proof}
                Let $E_0$ be the set that $f \neq g$. By additivity over domian,
                \begin{equation*}
                    \begin{aligned}
                        \int_E f &= \int_{E_0} f + \int_{E-E_0} f\\
                        \int_E g &= \int_{E_0} g + \int_{E-E_0} g\\
                        \Rightarrow \int_E g - \int_E f &= \int_{E_0} g - \int_{E_0} f
                    \end{aligned}
                \end{equation*}
                Since the integral over measure zero set is measure zero, the corollary is proved.
            \end{proof}
        
        \section{The Lebesgue Integral for Non-negative Functions}
            In this section, we push the definition of integral further, to non-negative measurable functions over general measurable 
        sets. We allowed the function to take large value(not being bounded anymore, but don't take values at infinity), and define it on a larger space(without requiring
        $E$ to have finite measure).\par  
            To this purpose, it is convenient to establish the concept of \textbf{finite support functions}.
            \begin{definition}[Support]
                For a function $f:E \to \RR^*$, where $E$ is a subset of $\RR$, we say it's \textbf{support} $E_0$ is the closure of 
            $$\{x\in E: f(x) \neq 0\}$$
                If the support is finite, then we say $f$ is a function has finite support, or compact support(notice that $E_0$ is closed 
            and bounded, thus compact).
            \end{definition}\par
            It is not hard to see that the linear combinenation of finite support functions are still finite support.\par 
            We haven't defined the integral on a set which has infinite measure. Suppose $f:E\to \RR$ is bounded, measurable
        and has a finite support $E_0$, then $$\int_E f := \int_{E_0} f$$
        is a very natrual definition. Any thing we proved in the last section can be applied on this new definition. For non-negative 
        measurable function, we can define it's integral through the integral of bounded measurable function of finte support.
            \begin{definition}[Integral of non-negative measurable functions]
                Let $f:E \to \RR^*$ be a non-negative measurable function defined on a measurable set. The integral of $f$ is defined as 
            \begin{equation}
                \int_E f := \sup\{\int_E \varphi : \varphi \mbox{ is bounded measurable and with finite support }; 0\leq \varphi \leq f\}
            \end{equation}
            \end{definition}
            To prove the properties of the integral of non-negative functions, \textbf{Chebychev's inequality} is needed.
            \begin{lemma}[Chebychev's inequality]
                Let $f$ be a non-negative function on $E$. Then for any $\lambda >0$,
                \begin{equation}
                    m(\{x \in E: f(x) \geq \lambda\}) \leq \frac{1}{\lambda} \cdot \int_E f
                \end{equation}
            \end{lemma}
            \begin{proof}
                We denoted $\{x \in E: f(x) \geq \lambda\}$ as $E_{\lambda}$. Suppose $g$ is defined by the following:
            $$g(x) := \left\{ \begin{matrix}
                \lambda & \mbox{ if } x \in E_{\lambda}\\
                0 & \mbox{ if } x \in E- E_{\lambda}
            \end{matrix} \right.$$
            Then $g \leq f$. If $m(E_{lambda})$, $g$ is measurable bounded with finite support. By definition, 
            \begin{equation*}
                \begin{aligned}
                    \int_E g &\leq \int_E f\\
                    \lambda \cdot m(E_{\lambda}) &\leq \int_E f\\
                    m(E_{\lambda}) &\leq \frac{1}{\lambda}\int_E f\\
                \end{aligned}
            \end{equation*} 
            If $m(E_{\lambda}) = +\infty$, let $E_{\lambda, n} := E_{\lambda} \cap [-n,n]$. Then $\psi_n = \chi_{E_{\lambda, n}}$
            is a bounded measurable function with finite support. And we have $\psi_n \leq f$. By continuity of the measure,
            \begin{equation*}
                \begin{aligned}
                    \lambda \cdot m(E_{\lambda}) &= \lambda \lim_{n \to \infty} E_{\lambda, n}\\ &= \int_E \psi_n\\ &\leq \int_E f\\
                \end{aligned}
            \end{equation*}
            Times $\frac{1}{\lambda}$ on the both side of the inequality, then we obtained the desirable conclusion.                 
            \end{proof}
            \begin{proposition}[Non-negative function and measure zero set]\label{Non-negative function and measure zero set}
                Let $f$ be a non-negative measurable function on $E$, then 
                $$\int_E f = 0$$
                if and only if $f= 0$ a.e. on $E$.
            \end{proposition}
            \begin{proof}
                If $f = 0$ a.e. on $E$, it is bounded, measurable, finite support. Then by the conclusion in the last section, 
            $$\int_E f = 0$$\par
                Suppose $\int_E f = 0$, and there exist a $E' \subset E$ such that $m(E') > 0 $ and $f > 0$ on $E'$. We can see that 
            $$E' = \bigcup_{n=1}^{\infty} \{x\in E: f(x) \geq \frac{1}{n} \}$$
            Since countable union of measure zero set is still measure zero, there exist $N$ such that 
            $m(\{x\in E: f(x) \geq \frac{1}{n} \}) > 0 $. By Chebychev inequality, 
            $$m(\{x\in E: f(x) \geq \frac{1}{n} \}) \leq n \int_E f = 0$$
            we arrive at a contradiction.
            \end{proof}
            \begin{theorem}[Linearity]
                Let $f$ and $g$ be two non-negative measurable functions on a measurable set, say $E$. Let $\alpha$, $\beta$ be two non-negative
            real numbers. Then
            $$\int_E \alpha f + \beta g = \alpha\int_E f + \beta \int_E g$$
            \end{theorem}
            \begin{proof}
                With ordinary method, we first prove the part of scalar multiple, then the part of addition.\par
                $\int_E \alpha f = \alpha \int_E f$: Notice that for $\alpha > 0 $, $0 \leq h \leq f$ if and only if $0<\alpha h < \alpha f$.
            Thus by the linearity of finite support, 
            \begin{equation*}
                \begin{aligned}
                    \int_E \alpha f &= \sup_{h' < \alpha} \int_E h'\\
                    &=\sup_{h < f} \int_E \alpha h \\
                    &= \alpha \sup_{h < f} \int_E h \\
                    &=\alpha \int_E f 
                \end{aligned}
            \end{equation*}
            where $h$ is bounded and measurable, with finite support. Thus the part of scalar multiple is done.\par
            Let $h , \ k$ be bounded and measurable with finite support, then their sum is also bounded and measurable, with finite support.
            By the definition of the integral,
            \begin{equation*}
                \begin{aligned}
                    \int_E h + \int_E k &=\int_E h + k\\
                    &\leq \int_E f+g 
                \end{aligned}
            \end{equation*}
            Thus for the supremum of the left side, the inequality still holds.
            $$\int_E f +\int_E g \leq \int_E f+g$$
            It remains to prove the inequality in the opposite side. Since $\int_E f+g$ is the supremum of $\int_E l$, where $l \leq f+g$
            is a bounded measurable function with finite support. So it is surffice to  prove the inequality for all such $l$ in place of 
            $f+g$.\par
                For a such $l$, let $h = min(f, l), \ k= l-h$. Then if $l(x) \leq f(x)$, $h(x) = l(x)$ and $k(x)= 0$; if $l(x) > f(x)$,
            $h(x)=f(x)$ and $k(x) = l(x)-f(x) < g(x)$. Then we can see that $h \leq f$, $k \leq g$. Thus by the additivity of the integral 
            for bounded measurable fucntion of finite support,
            \begin{equation*}
                \begin{aligned}
                    \int_E l &= \int_E l + min(f, l) - min(f,l)\\
                    &=  \int_E min(f, l) + \int_E l - min(f,l)\\
                    &= \int_E h + \int_E k \\
                    &\leq \int_E f + \int_E g\\
                \end{aligned}
            \end{equation*}
            Thus proved the the part of additivity.
            \end{proof}
            \begin{theorem}[Monotonicity]
                Suppose $f$ and $g$ are non-negative measurable functions on measurable set, such that $ f \leq g$ on their domain $E$.
            Then
            $$\int_E f \leq \int_E g$$
            \end{theorem}
            \begin{proof}
                Notice the fact that for all bounded measurable functions $\varphi$ on $E$, which satisfy $\varphi \leq f$, we have
            $\varphi \leq g$. So we know that 
                \begin{equation*}
                    \begin{aligned}
                        \sup_{\varphi <f} \int_E \varphi &\leq \sup_{\psi <g} \int_E \psi\\ 
                        \int_E f &\leq \int_E g 
                    \end{aligned}
                \end{equation*}
                Thus the theorem is proved.
            \end{proof}
            \begin{theorem}[Additivity over domain]
                Let $f$ be a non-negative measurable function over a measurble set $E$. Suppose $E_1, E_2$ are disjoint measurable subsets,
            then we have
            $$\int_{E_1 \cup E_2 }f = \int_{E_1} f + \int_{E_2} f$$            
            \end{theorem}
            \begin{proof}
                Observing that $f = f\chi_{E_1} +f\chi_{E_2}$ on $E_1 \cup E_2$, by linearity of the integral, 
                \begin{equation*}
                    \begin{aligned}
                        \int_{E_1\cup E_2} f &= \int_{E_1\cup E_2} f\chi_{E_1} +f\chi_{E_2}\\
                        &= \int_{E_1\cup E_2} f\chi_{E_1} + \int_{E_1\cup E_2} f\chi_{E_2}\\
                    \end{aligned}
                \end{equation*}
                One can check that for all measurable bounded finite supported $\varphi $ on $E_1 \cup E_2$, $\varphi \leq f$ on $E_1 
            \cup E_2$ if and only if $\varphi \chi_{E_1} \leq f \chi_{E_1}$ on $E_1$ and $\varphi \chi_{E_2} \leq f \chi_{E_2}$ on $E_2$. So
            \begin{equation*}
                \begin{aligned}
                    \int_{E_1\cup E_2} f\chi_{E_1} + \int_{E_1\cup E_2} f\chi_{E_2} &= \sup_{\varphi<f} \int_{E_1} \varphi \chi_{E_1} +\sup_{\varphi<f} \int_{E_2} \varphi \chi_{E_2}\\
                    &= \sup_{\varphi<f} \int_{E_1} \varphi +\sup_{\varphi<f} \int_{E_2} \varphi \\
                    &= \int_{E_1} f + \int_{E_2} f \\
                \end{aligned}
            \end{equation*}
            \end{proof}
            
        \section{The Lebesgue Integral in General}
            For measurable function $f$ in general, our mean to construct it's integral is to express $f$ as the difference of
        two non-negative functions. In the chapter of Lebesgue measure functions, we have a conclusion that maximum and minimumu of two 
        measurable functions is still measurable, which allowe us to split $f$ into two part.
            \begin{definition}[Positive part and negative part]
                Let $f$ be a measurable function on $E$. We define it positive part $f^+$ and negative part $f^-$ as the following:
                \begin{equation*}
                    \begin{aligned}
                        f^+ = max(f, 0)\\
                        f^- = -min(f, 0) 
                    \end{aligned}
                \end{equation*} 
            \end{definition}
            It is not hard to find that $f^+$, $f^-$ are non-negative functions. And the relation 
            $$f = f^+ - f^-$$
        is also clear. But $\int f^+,  \ \int f^-$ might be infinity at the same time, then $\infty - \infty$ is not a well defined value. To fix this
        problem, we introduce the concept of integralbe for non-negative measurable functions.
            \begin{definition}[Integralbe]
                Let $f$ be a non-negative function on $E$, $f$ is integralbe on $E$ if and only if $$\int_E f$$
            is a finite value.
            \end{definition}
            \begin{proposition}\label{IntegralbeFiniteAE}
                A non-negative function $f$ is integrable over $E$, then it is finite a.e. on $E$.
            \end{proposition}
            \begin{proof}
                If $f$ is integrable, by Chebychev inequality,
                $$m(\{x \in E: f(x) \geq \infty\}) \leq m(\{x \in E: f(x) \geq \lambda\}) \frac{1}{\lambda} \cdot \int_E f$$
                Let $\lambda \to \infty$, we see that the measure of the subset where $f=\infty$ is zero.
            \end{proof}
            \begin{proposition}
                 For a measurable function $f$ on $E$, $f^+$ and $f^-$ are integralbe on $E$ if and only if $|f|$ is integralbe on $E$. 
            \end{proposition}
            \begin{proof}
                Suppose $|f|$ is integralbe on $E$. Notice that $|f|= f^+ + f^-$, suppose $f^+$ is not integralbe 
                \begin{equation*}
                    \begin{aligned}
                        +\infty &= \int_E f^+ \\
                        &\leq \int_E f^+ \int_E f^-\\
                        &= \int_E |f|
                    \end{aligned}
                \end{equation*} 
            Contradicting with the fact that $|f|$ is integralbe, $f^+$ is integralbe. One can show that $f^-$ is integralbe in the exact
            same way. Then we proved one direction of the proposition.\par
                Suppose $f^+$ and $f^-$ are integrable, which means their integrals are finite. Then 
                \begin{equation*}
                    \begin{aligned}
                        \int_E |f| &= \int_E (f^+ +f^-)\\ 
                        &= \int_E f^+ + \int_E f^-
                    \end{aligned}
                \end{equation*}
            So $|f|$ is integralbe.
            \end{proof}
        \begin{definition}[Lebesgue integral for measurable functions]
            Let $f$ be a measurable function on $E$. If $|f|$ is integrable on $E$, then we say $f$ is integrable, and define the integral of $f$ on $E$ as 
            \begin{equation}
                \int_E f = \int_E f^+ - \int_E f^-
            \end{equation}
        \end{definition}
        \begin{proposition}
            Let $f$ be integrable over $E$, then $f$ is finite a.e. on $E$, and 
            \begin{equation}\label{GLIMeasureZeroSet}
                \int_E f = \int_{E-E_0} f
            \end{equation} 
            where $E_0$ is a measure zero subset of $E$.
        \end{proposition}
        \begin{proof}
            If $f$ is not finite almost every where, then $|f| = \infty$ on $F \in E$ where $m(F) > 0$. 
        Then by the additivity over domain, and the monotonicity of integration,
        \begin{equation*}
            \begin{aligned}
                \int_E |f| &= \int_{E- F} f + \int_{F} f\\
                &\geq \int_{F} f\\
                &= \infty \cdot m(F)\\
                &= \infty
            \end{aligned}
        \end{equation*}
        which contradict with the fact that $f$ is integral.\par
        Now we prove \eqref{GLIMeasureZeroSet}. Notice that since the proposition \ref{The integral over measure zero set is zero}, being 
        the supremum of $\{0\}$, $\int_{E_0} f^+ = \int_{E_0} f^- = 0$.
        By the definition of the integral,
        \begin{equation*}
            \begin{aligned}
                \int_{E} f &= \int_{E-E_0} f^+ - \int_{E-E_0} f^-\\
                &= \int_{E-E_0} f^+ + \int_{E_0} f^+ - \int_{E-E_0} f^- - \int_{E_0} f^-\\
                &= \int_{E-E_0} f^+ - \int_{E-E_0} f^-\\ 
                &= \int_{E-E_0} f
            \end{aligned}
        \end{equation*}
        Thus the proposition is proved.
        \end{proof}
        Now we prove the familiar basic properties of integration in the most general extend. The function will be the general measurable functions, 
    which can take value at infinity. Although we can't even defined the sum of two measurable functions if they have too much value of 
    infinite, but if we ask them to be integrable, every thing turns out to be nice.
        \begin{theorem}[Linearity]
            Let $f , \ g$ be two measurable functions which are integrable over $E$, $\alpha$ and $\beta$ are real numbers. Then 
        $\alpha f+\beta g$ is integrable, and 
            $$\int_E \alpha f+\beta g = \int_E \alpha f + \int_E \beta g $$
        \end{theorem}
        \begin{proof}
            As usual, we first prove $\alpha f$ is integrable and $\int_E \alpha f = \alpha \int_E f$. Since $|\alpha f| = |\alpha| |f|$,
        by the linerarity of the inftegral of non-negative functions,
            \begin{equation*}
                \begin{aligned}
                    \int_E |\alpha f | = \int_E |\alpha| | f | = |\alpha| \int_E | f |\\ 
                \end{aligned}
            \end{equation*}
            Since both $ |\alpha|$ and $\int_E | f |$ are finite,  $\alpha f $ is integrable. If $\alpha \geq 0$,
            \begin{equation*}
                \begin{aligned}
                    \int_E \alpha f &= \int_E (\alpha f)^+ + \int_E (\alpha f)^- \\
                    &= \alpha \int_E f^+ +  \alpha \int_E f^-\\
                    &= \alpha \int_E f
                \end{aligned}
            \end{equation*}
            If $\alpha < 0$, 
            \begin{equation*}
                \begin{aligned}
                    \int_E \alpha f &= \int_E (\alpha f)^+ + \int_E (\alpha f)^- \\
                    &= \alpha \int_E f^+ +  \alpha \int_E f^-\\
                    &= \alpha \int_E f
                \end{aligned}
            \end{equation*}
            By the proposition above, $f$ and $g$ are finite a.e. on $E$, thus their sum is well defined. By triangle inequality of real 
        numbers, $|f+g| \leq |f|+|g|$. Then by the monotonicity of the integral of non-negative functions, one can see that $\int_E|f+g|$
        is finite, which indecated that $f+g$ is integrable. \par
            We have the following relationship:
            \begin{equation*}
                (f+g)^+ - (f+g)^- = f+g = (f^+ - f^-) + (g^+ -g^-)
            \end{equation*}
            Since the value on a measure zero set doesn't influence the integral, one can assuming that the above functions take finite 
        value on $E$, so 
            \begin{equation*}
                (f+g)^+ + f^- +g^-  = (f+g)^- + f^+  + g^+ \mbox{ on } E
            \end{equation*}
            From the linearity of non-negative functions(notice that both side of the equality above is non-negative), 
            \begin{equation*}
                \begin{aligned}
                    \int_E [(f+g)^+ + f^- +g^-]  &= \int_E [(f+g)^- + f^+  + g^+] \\
                    \int_E (f+g)^+ + \int_E f^- +\int_E g^-  &= \int_E (f+g)^- + \int_E f^+  + \int_E g^+\\
                \end{aligned}
            \end{equation*}
            Since $f$, $g$, $f+g$ are all integrable, the integral of non-negative function above are all finite. So it is allowed to 
        move some of them to the other side.
        \begin{equation*}
            \begin{aligned}
                \int_E (f+g)^+ - \int_E (f+g)^-  &= \int_E f^+ - \int_E f^- + \int_E g^+   - \int_E g^-\\
                \int_E f+g &= \int_E f + \int_E g 
            \end{aligned}
        \end{equation*} 
        \end{proof} 
        \begin{theorem}[Monotonicity]
            Let $f$ and $g$ be integralbe over $E$. If $f\leq g$ on $E$, then 
            \begin{equation*}
                \int_E g \geq \int_E f
            \end{equation*}
        \end{theorem}
        \begin{proof}
            Since $g \geq f$, it is not hard to see $g^+ \geq f^+$, and $g^- \leq f^-$. By monotonicity of the integral of non-negative
        function, 
        $$\int_E g^+ \geq \int_E f^+, \ \ \ \int_E g^- \leq \int_E f^-$$
        which indicated 
        \begin{equation*}
            \begin{aligned}
                \int_E g^+ - \int_E g^- &\geq \int_E f^+ - \int_E f^-\\
                \int_E g &\geq \int_E f
            \end{aligned}
        \end{equation*} 
        \end{proof}
        \begin{theorem}[Additivity over domain]
            Let $f$ be a integralbe function over $E$, and $E_1, \ E_2$ be measurable subset of $E$. Then $f$ is integrable over $E_1$ and 
        $E_2$, and we have 
        $$\int_{E_1 \cup E_2} f = \int_{E_1} f + \int_{E_2} f$$
        \end{theorem}
        \begin{proof}
            By the symmetry of the proposition, we only need to prove that $f$ is integralbe $E_1$. By the additivity over domian of the 
        integral of non-negative functions, and the fact that the integral of any non-negative function is non-negative,
        \begin{equation*}
            \begin{aligned}
                \int_E |f| &= \int_{E-E_1} |f| +\int_{E_1} |f| \\
                &\geq \int_{E_1} f \\
            \end{aligned}
        \end{equation*}
        So it is clear that $f$ is integrable over $E_1$. \par 
            By the additivity over domain of non-negative functions,
        \begin{equation*}
            \begin{aligned}
                \int_{E_1\cup E_2} f &= \int_{E_1\cup E_2} f^+ - \int_{E_1\cup E_2} f^-\\
                &= \int_{E_1} f^+ + \int_{E_2} f^+ - \int_{E_1} f^- - \int_{E_2} f^-\\
                &= \int_{E_1} f + \int_{E_2} f
            \end{aligned}
        \end{equation*}
        \end{proof}
            Here we sum up the property of Lebesgue integral:
            \begin{equation*}
                \begin{aligned}
                    \mbox{Measure zero sets and integral: }& m(E_0) = 0 \Rightarrow \int_E f = \int_{E-E_0} f\\
                    \mbox{Linearity: }& \int_E f + \int_E g = \int_E (f+g)\\
                    \mbox{Monotonicity: }& f \leq g \Rightarrow \int_E f \leq \int_E g\\
                    \mbox{Additivity over domain: }& \int_{E_1} f + \int_{E_2} f = \int_{E_1 \cup E_2} f\\ 
                \end{aligned}
            \end{equation*}
            The method of constructing the Lebesgue integral is very inspiring. When dealing with a rather gengeral mathematic concept,
        one can first consider some special case, and extend the proposition or property to the more general case.  
    \section{Lebesgue vs Riemann}
        In this section we give some example, and discuss some difference between Lebesgue integral and Riemann integral. In the section of 
    the integral for bounded measurabe function, we have show that any Riemann integralbe function is Lebesgue integralbe, but didn't show 
    the converse is false. Now we give an counter-example.
        \begin{example}
            Let $f = \chi_{\RR -\QQ} \cap [0,1]$(the Dirichlet function). No matter how fine a Riemann partition is, the small interval
        determined by the partition will contains a rational and a irrational, which means the upper Riemann integral will be $1$ and 
        the lower will be $0$. So $f$ is not Riemann integralbe. \par
            But with Lebesgue integral, $f$ is just a simple function, so is integralbe and it's integral is $1$.
        \end{example}

        \begin{theorem}[The Lebesgue condition for Riemann integral funciton]
            
        \end{theorem}
        By the upper theorem, a Riemann integrable function on a closed bounded interval is continuous a.e. on it's domain; hbounded 
    functions on intervals is Lebesgue integrable if and only if it is measurable. By Lusin's theorem, given any $\epsilon>0$, it is 
    continouos on it's domain but a set whose measure is less than $\epsilon$. 

\chapter{Limit and Itergral}
        In the first section, we promised that Lebesgue integral will has better properties compare to Riemann integral. Now we have 
    seen that Lebsegue integral have defined on more functions and sets. In this chapter, we will focus on the issue 
    that under what condition, the following equation will hold: 
    \begin{equation*}
        \lim_{n \to \infty} \int_E f_n(x) = \int_E f(x) 
    \end{equation*}
    where $E$ is a measurable set, and $\{f_n\} \subset $\lmf{E} which converge to $f$.  
        
    \begin{theorem}[Bounded convergence theorem]
        Let $\{f_n\}$ be a sequence of measurable function on $E$, where $m(E)<+\infty$. Suppose $f_n$ is uniform bounded on $E$. 
    If $\{f_n\} \to f$ pointwisely a.e. on $E$, then 
    $$\lim_{n \to \infty} \int_E f_n(x) = \int_E f(x) $$
    \end{theorem}\par
        Although we need to show that $f$ is integrable, this part is rather trivial, so is it left to the reader to check.
        Our basic idea of the prove is by using Egoroff's theorem, to show that $\{f_n\}$ coverge uniformly on a very large subset.
    Then for the part on that subset, the integral and the limit sign could commute. Out of the subset, the difference of integral 
    will be small enough.
    \begin{proof}
        Since our result is the equation between integrals, we can assume $\{f_n\}$ converge pointwisely on $E$. Assuming $|f_n| < M$ 
    for all possible $n$. Since $m(E)< +\infty$, Egoroff's theorem is available. Thus given $\epsilon >0 $, 
    we can find a close subset of $E$, say $F$, such that $m(E - F) < \frac{\epsilon}{ 4M \cdot m(E) }$  and $f_n$ converge uniform to 
    $f$. Also because the uniform convergence, for the same $\epsilon$ we can find a $N$ such that for $n\geq N$, 
    $$|\int_F f_n -\int_F f| \leq \frac{\epsilon}{2}$$
    Together with the above estimation and properties of integral,
        \begin{equation*}
            \begin{aligned}
                |\int_E f_n - f| &= |\int_{E-F} f_n - \int_{E-F} f + \int_{F} f_n -\int_{F} f|\\
                &\leq |\int_{E-F} f_n - \int_{E-F} f| + |\int_{F} f_n -\int_{F} f|\\
                &\leq \int_{E-F} 2M  + \frac{\epsilon}{2}\\
                &\leq 2M \cdot m(E-F) +\frac{\epsilon}{2}\\
                &\leq 2M \cdot \frac{\epsilon}{ 4M \cdot m(E) } +\frac{\epsilon}{2}\\
                &= \epsilon
            \end{aligned}
        \end{equation*}
    Thus we show that the limit sign could go though the integral sign.
    \end{proof}
        Uniform bounded is weaker than uniform converge, but we can prove much more strong converge theorem under the context 
    of Lebesgue integral. To prove the stronger converge theorems, it is conveneint to establish the following lemma.
    \begin{lemma}[Fatou's lemma]
        Let $\{f_n\}$ be a sequence of non-negative measurable functions on $E$. If $\{f_n\}$ converge pointwisely a.e. over $E$.
    then 
    $$\int_E f \leq \lim \inf \int_E f_n$$
    \end{lemma}
    Remember that $\lim \inf $ of a sequence $\{a_n\}$ is defined as $\sup \{\inf \{a_n\}_{n=k}^{\infty} : 1\leq i < \infty \}$.
    \begin{proof}
        First, measure zero sets doesn't influence the integral, thus one can assume $\{f_n\}$ converge pointwise on $E$.
    Since the integral of non-negative functions is defined as the supremum of the integral of bounded measurable function of finite 
    support, we only need to show that for all such function $\varphi \leq f$, 
    $$\int_E \varphi \leq \lim \inf \int_E f_n$$\par
        Let $\varphi_n = min(\varphi, f_n)$, then $\{\varphi_n\}$ coverge to $\varphi$ pointwisely. There exist a set of finite measure 
    $E_0$ on which $\varphi$ and $\varphi_n$ support on. So by bounded converge theorem,
    \begin{equation*}
        \begin{aligned}
            \int_E \varphi = \int_{E_0} \lim_{n \to \infty} h_n = \lim_{n \to \infty} \int_{E_0} h_n
        \end{aligned}
    \end{equation*}
    By the basic properties of limit and integral, and the fact that $h_n \leq f_n$we have 
    \begin{equation*}
        \begin{aligned}
            \int_E \varphi &= \lim_{n \to \infty} \int_E h_n\\
            &\leq \lim \inf \int_E f_n
        \end{aligned}
    \end{equation*}
    \end{proof}
        The prove above illustrate that if you want to prove a proposition of non-egative function, you can first prove it for bounded 
    measurable function of finite support. Also if yoou want to prove something for measurable function, it is always helpful to prove it 
    for simple functions. 
    \begin{theorem}[Monotone converge theorem]
        Let $\{f_n\}$ be a sequence of nonegative measurable functions on $E$. If $f_m \leq f_n$ for $m<n$, and $\{f_n\} \to f$ pointwisely a.e. on
    $E$, then 
    $$\lim_{n \to \infty} \int_E f_n(x) = \int_E f(x) $$
    \end{theorem}
    \begin{proof}
        We assume $\{f_n\}$ converge to $f$ pointwisely on $E$. Since the sequence is increasing, by monotoniciy, if $m<n$, we have
        $$\int_E f_m \leq \int_E f_n$$
        Thus by Fatou's lemma 
        $$\int_E f \leq \lim \inf \int_E f_n = \lim_{n\to \infty}\int_E f_n$$
    The last equality comes from the fact that $\{\int_E f_n\}$ is a increasing sequence. It remains to show the inequality from the other
    side. Since $\{f_n\}$ converge to $f$ increasingly,$f> f_n$ for all $n$. Thus by the monotonicity of the integration,
        $$\int_E f_n \leq \int_E f$$
        $$\Rightarrow \lim_{n \to \infty} \int_E f_n \leq \int_E f$$
    \end{proof}
    Notice that the existence of the limit of the integration of the sequence of function is guaranteened because the general limit exist 
    for monotonic sequences.
    \begin{corollary}
        Let $\{f_n\}$ be non-negative measurable function on $E$. If $f = \sum_{n=1}^{\infty} f_n$ converge pointwisely a.e. on $E$,
        $$\int_E f = \sum_{n=1}^{\infty} \int_E f_n$$
    \end{corollary}
    \begin{proof}
        Notice that the partial sum is coverge increasingly(since $f_n$ are non-negative), so we can apply the monotone converge theorem.
    Let $F_k = \sum_{n=1}^{k} \int_E f_n$.
        \begin{equation*}
            \begin{aligned}
                \int_E f &= \lim_{n \to \infty} \int_E F_k\\
                &= \lim_{n \to \infty} \int_E \sum_{n=1}^{k} \int_E f_n\\
                &= \sum_{n=1}^{\infty} \int_E f_n
            \end{aligned}
        \end{equation*}
    \end{proof}
    \begin{lemma}[Beppo Levi's Lemma]
        Let $\{f_n\}$ be a sequence of non-negative measurable function on $E$. If $\{\int_E f_n\}$ is bounded, then $\{f_n\}$ converge 
    pointwisely to a $f$ on $E$, which is finite a.e on $E$, and 
    $$\lim{n \to \infty} \int_E f_n = \int_E f < \infty$$ 
    \end{lemma}
    \begin{proof}
        Since every monotone sequence of real number converge to a extended real number, so for every $x\in E$, we can define 
        $$f(x) = \lim_{n\to \infty} f_n (x)$$
    Then it is obvious that $\{f_n\}$ converge to $f$ pointwisely, and increasingly. Thus we can use monotone converge theorem, which shows
        $$\lim_{n\to \infty} \int_E f_n = \int_E f$$
        Since $\{\int_E f_n\}$ is bounded, $\int_E f$ is a finite value. So $f$ is integralbe, which indicated $f$ is finite a.e. on $E$
        (proposition \ref{IntegralbeFiniteAE}).
    \end{proof}
    \begin{theorem}[Legesgue dominated converge theorem]
        Let $\{f_n\}$ be a sequence of measurable functions on $E$, which converge pointwisely a.e. to $f$ on $E$. If $f_n$ is dominated 
    by a integrable function $g$, in the sense of $|f_n|\leq g$, then
        $$\lim_{n \to \infty} \int_E f_n = \int_E f$$
    \end{theorem}
    \begin{proof}
        Again, the \textsl{a.e.} can be ignored. Notice that $f$ is also dominated by $g$. By monotonicity of the integral,
        $$\int_E |f| \leq \int_E g < \infty$$
    $f$ is integralbe.
        $g-f_n$ and $g-f$ are all non-negative functions, and $(g-f_n) \to (g-f)$ pointwisely as $n \to \infty$. Thus we can apply 
    Fatou's lemma.
        \begin{equation*}
            \begin{aligned}
                \int_E g-f &\leq \lim \inf \int_E g -f_n \\            
                &= \int_E g - \lim \sup \int_E f_n\\
                \Rightarrow \int_E g- \int_E f &\leq \int_E g - \lim \sup \int_E f_n\\
                \Rightarrow \int_E f &\geq \lim \sup \int_E f_n\\
            \end{aligned}
        \end{equation*}
    Observing that $g+ f_n$ and $g+f$ and also non-negative, and $(g+f_n) \to (g+f)$ pointwisely as $n \to \infty$,
    using the same trick,
        \begin{equation*}
            \begin{aligned}
                \int_E g+f &\leq \lim \inf \int_E g +f_n \\            
                &= \int_E g + \lim inf \int_E f_n\\
                \Rightarrow \int_E g+ \int_E f &\leq \int_E g + \lim \inf \int_E f_n\\
                \Rightarrow \int_E f &\leq \lim \inf \int_E f_n\\
            \end{aligned}
        \end{equation*}
    Together the two inequality, we have 
        $$\int_E f \leq \lim \inf \int_E f_n \leq \lim \sup \int_E f_n \leq \int_E f$$
    Thus $\lim_{n\to \infty} f_n$ exist and 
    $$\int_E f = \lim_{n\to \infty} f_n$$ 
    \end{proof}
    \begin{theorem}[General dominated converge theorem]

    \end{theorem}
    \begin{definition}[Uniform integral]
        
    \end{definition}

    \begin{theorem}[Vetali's converge theorem]
        
    \end{theorem}

    \begin{theorem}[General Vetali's converge theorem]
        
    \end{theorem}
    
\chapter{Lp Space}
    \section{Definition}

    \section{Important Inequality}

    \section{?}

\chapter{Appendix 1: Non-measurable Sets, Non-measurable Functions}
    For the purpose to illustrate the theory in a more clear way, we didn't talk about the existence of non-measurable sets, non-measurable functions and many 
other counter examples. Counter-examples are important. Without showing there exsistence, our theory is meaningless(mabey all the sets are 
measurable). In fact, many of them are quite hard to construct, which is the reason why we didn't mention them in the main body.   
\end{document} 